
% Au cours de cette thèse, la généralisation sur de nombreux sites de prélèvement et le
% développement de la méthodologie PMF a permis la quantification de nouvelles sources
% d'émissions de PM et une estimation de l'incertitude des profils chimiques associée.
% Notamment, l'ajout de traceurs moléculaire spécifiques permet l'identification d'une
% source biologique primaire ubiquitaire en France, pour une grande diversité
% d'environnements.
Le travail en amont des PMF par sur la provenance du nitrate d'ammonium par mesure
isotopique, origine géographique du MSA par PSCF ou des polyols grâce à l'établissement
d'une base de données harmonisées et mesures génétiques a permis le renforcement des
connaissances des processus d'émissions des PM et l'identification de nouvelles sources
potentielles.
La caractérisation de l'aérosol organique secondaire a également été rendu possible par
l'ajout d'espèces chimiques comme le MSA, l'acide pinique et 3-MBTCA, permettant la mise
en lumière de 2 facteurs d'oxydation secondaires pouvant contribuer à une part importante
de la matière organique totale, particulièrement en période estivale.

Fort de ces connaissances, une étude harmonisée à large échelle spatiale regroupant des
mesures de terrain de 15 sites de prélèvements en France a permis une phénoménologie des
\PMdix{} à large échelle spatiale. La similitude géochimique des profils de d'émission ou
provenant de processus secondaires a été estimé et la variabilité inter-site évaluée grâce
aux méthodologies proposées par le groupe FAIREMODE du JRC. L'utilisation de traceurs
spécifiques, y compris d'espèces organiques comme les polyols, MSA ou lévoglucosan,
conduit à l'identification de 8 facteurs présent sur la quasi-totalité des sites de
mesures : combustion de biomasse domestique, trafic routier, poussières minérales,
biologique primaire, sel marin agée, \textit{MSA-rich}, \textit{nitrate-rich} et \textit{suflate-rich}.
Cette méthodologie harmonisée ne masque pas les spécificités locales à chacun des sites
par une trop grande standardisation, mais permet une généralisation et une diminution de
la subjectivité de l'expérimentateur.

Enfin, la mesure sur site du potentiel oxydant par deux tests a-cellulaires (\POAA{} et
\PODTT) suivant un protocole standardisé permettant une comparabilité des échantillons et
la prise en compte de la bio-accessibilité permet la première climatologie grande échelle
de cette nouvelle métrique d'intérêt sanitaire.
L'utilisation de méthode de déconvolution des sources de PO suivant un modèle simple de
régression linéaire multiple grâce aux études PMF quantifie avec une performance
satistiques très satisfaisante un PO intrinsèque pour chaque facteur déterminé.
En plus d'être statistiquement valable, la généralisation à large échelle spatiale
identifie le PO intrinsèque des différentes sources avec une bonne homogénéité, permettant
l'attribution robuste d'un PO intrinsèque par typologie de source.

Suivant cette méthodologie, l'importance des différentes sources d'émissions varient
considérablement selon que la métrique d'observation est la concentration massiques de
\PMdix{} ou leurs \POv{} mesurés par acide ascorbique ou DTT. Notamment, les pics de
pollutions printaniers, dont la source majoritaire est le \textit{nitrate-rich} ne semble
pas présenté de réactivité particulière au regard du potentiel oxydant. En revanche, la
combustion de biomasse et les espèces organiques associées présente un PO intrinsèque
relativement élevé et l'importance de cette source en hiver en fait l'une des sources
prépondérante du \PODTTv{} mais surtout \POAAv{} en moyenne annuelle.
Seulement, l'exposition à de plus faible concentration mais tout au long de l'année aux
émissions véhiculaires du trafic routier associé au plus fort PO intrinsèque parmis toutes
les sources identifiées fait du trafic routier la source principale de l'exposition
chronique. Associé à la présence de cette source notament en milieu urbain et donc
densément peuplé, cela fait des émissions liées au trafic routier l'un des leviers
prépondarant d'amélioration de la qualité de l'air.

En revanche, la raison du \POv{} élevé du trafic routier n'est peut-être pas suffisament
connue.
Nous ne sommes pas capable en l'état actuelle des connaissances de séparer les émissions à
l'échappement des émissions hors-échappement (usure des pneus et freins, resuspension de
bitume, etc). 
La séparation entre ces 2 types d'émissions liées au trafic routier à l'heure du
remplacement du parc automobile par des voitures ``propre'' (comprendre ``sans combustion
de pétrole lors du déplacement'') présentant des émissions à l'échappement quasi nulle
est donc une lacune importante de cette méthodologie pour laquelle des améliorations
serait bienvenue.
Cependant, il est très peu probable que les raffinenement futures des connaisances des sources de
PM changent drastiquement la vision que l'on a actuellement et l'identification de nouveaux
facteurs ne viendrait a priori que subdiviser une typologie de source déjà identifiée.

Cette connaissance générale ne doit cependant pas occulter les particularités locales des
différents environnements.
La recherche de nouveaux traceurs spécifiques pourra permettre l'identification de sources
ponctuelles et statistiquement peu présentes spatiallement et temporellement
(industrielle, portuaire, volcanique, etc.), et présente donc un intérêt scientifique et
politique certains.
Bien que nécessaire, cette connaissance et diversité locale ne devrait cependant pas
remettre en cause la connaisance générale établit jusuq'àlors.

En définitive, cette thèse démontre l'intérêt d'une nouvelle métrique de caractérisation
de l'aérosol à travers la mesure du potentiel oxydant pour la quantification de l'impact
sanitaire de la qualité de l'air. 
Ces travaux présentent une assise suffissante à l'implémentation du
potentiel oxydant dans les modèles déterministe et de prévision de la qualité de l'air, et
mes travaux futurs y seront consacrés.

Cependant, bien que la mesure du potentiel oxydant soit très prometteuse et présente une
étape importante, il est important de se souvenir que la diversité des polluants de l'air
ne sont cependant pas tous quantifiés par cete métrique.  Par exemple, certains polluants
émergents comme les pesticides n'étant pas redox-actifs et ne contribuant que très peu à
la masse des aérosols restent ``invisibles'' par ces différentes métriques. D'autres
familles de polluant pourraientt également ne pas nécessairement être sensible à métrique
sanitaire, comme les perturbateurs endocriniens ou microplastique.

Cette thèse participe donc humblement à l'évaluation et quantification des sources
du potentiel oxydant des particules comme étape vers une meilleure prise en compte de
l'exposition et impact de la qualité de l'air. Cependant, de nombreuses questions restent
ouvertes et non adressées dans cette recherche concernant les laisions entre chimie de
l'aérosol, toxicologie et impacts sanitaires.
