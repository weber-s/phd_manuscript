Cette thèse s'est attachée à la quantification rigoureuse des sources d'émissions des \PMdix.
Le travail en amont des PMF sur la provenance du nitrate d'ammonium par mesure
isotopique, origine géographique du MSA par PSCF ou des polyols grâce à l'établissement
d'une base de données harmonisées et mesures génétiques a permis le renforcement des
connaissances des processus d'émissions des PM et l'identification de nouvelles sources
potentielles.
La caractérisation de l'aérosol organique secondaire a également été rendue possible par
l'ajout d'espèces chimiques comme le MSA, l'acide pinique et 3-MBTCA, permettant la mise
en lumière de 2 facteurs d'oxydation secondaires pouvant contribuer à une part importante
de la matière organique totale, particulièrement en période estivale.
Le raffinement des solutions PMF par ajout de ces traceurs organiques permet une
estimation plus précise des sources de \PMdix{} et peut mettre en lumière des nouveaux
processus ou sources jusqu'alors ignorés.
Cependant, il s'agit bien d'une amélioration et non d'une remise en cause des études
précédentes. Les nouveaux facteurs identifiés viennent principalement modifier à la marge ceux
préexistants, tout en diminuant les incertitudes qui leur était associées.

Fort de ces connaissances, une étude harmonisée à large échelle spatiale regroupant des
mesures de terrain de 15 sites de prélèvements en France a permis une phénoménologie des
\PMdix{} à large échelle spatiale. La similitude géochimique des profils d'émission ou
des processus secondaires a été estimée et la variabilité inter-site évaluée grâce
à la méthodologie deltaTool proposée par le groupe FAIREMODE du JRC. L'utilisation de traceurs
spécifiques, y compris d'espèces organiques comme les polyols, MSA ou lévoglucosan,
conduit à l'identification de 8 facteurs présents sur la quasi-totalité des sites de
mesures : combustion de biomasse domestique, trafic routier, poussières minérales,
biologique primaire, sel marin âgé, \textit{MSA-rich}, \textit{nitrate-rich} et \textit{suflate-rich}.
Cette méthodologie harmonisée permet une généralisation des résultats et une diminution de
la subjectivité de l'expérimentateur. Une trop grande homogénéisation qui aurait pu être
liée à cette standardisation n'est pas observée et cette méthodologie permet bien la prise
en compte des spécificités locales à chacun des sites.

Enfin, la mesure sur site du potentiel oxydant par deux tests a-cellulaires (\POAA{} et
\PODTT) suivant un protocole identique pour tous les échantillons (mesure à iso-masse et
dans un surfactant pulmonaire pour la prise en compte de la bio-accessibilité) permet une
comparabilité des prélèvements. Ainsi, nous avons pu établir la première climatologie grande échelle
de cette nouvelle métrique d'intérêt sanitaire.
L'utilisation de méthode de déconvolution des sources de PO suivant un modèle simple de
régression linéaire multiple grâce aux études PMF quantifie avec une performance
statistique très satisfaisante un PO intrinsèque pour chaque facteur déterminé.
En plus d'être statistiquement valable, la généralisation à large échelle spatiale
attribue un PO intrinsèque similaire à chaque typologie de source, quel que soit le site
considéré pour les sources de PM majoritaires.

Suivant cette méthodologie, l'importance des différentes sources d'émissions varient
considérablement selon que la métrique d'observation est la concentration massique de
\PMdix{} ou leurs \POv{} mesurés par acide ascorbique ou DTT. Notamment, les pics de
pollution printaniers, dont la source majoritaire est le \textit{nitrate-rich}, ne
semblent
pas présenter de réactivité particulière au regard des tests de potentiel oxydant. En
revanche, la combustion de biomasse pour le chauffage résidentiel présente un PO
intrinsèque relativement élevé et l'importante activité de cette source en hiver en fait l'une des
sources prépondérante du \PODTTv{} mais surtout du \POAAv{} en moyenne annuelle.  Seulement,
le fort PO intrinsèque des PM issues des émissions primaire du trafic routier associé à
l'exposition à certes de plus faible concentration mais présente tout au long de l'année
fait du trafic routier la source principale de l'exposition chronique aux deux tests de
\POv.  La présence de cette source notamment en milieu urbain et donc densément
peuplé fait de cette source l'un des leviers prépondérant d'amélioration de la qualité de
l'air.

En revanche, la raison du \POv{} élevé du trafic routier n'est peut-être pas suffisamment
connue.
Nous n'avons pas réussi, en l'état actuel des connaissances, à séparer les émissions à
l'échappement des émissions hors-échappement (usure des pneus et freins, re-suspension de
bitume, etc). 
La séparation entre ces 2 types d'émissions liées au trafic routier à l'heure du
remplacement du parc automobile par des voitures ``propres'' (comprendre ``sans combustion
de pétrole lors du déplacement'') qui ne présentent pas d'émission à l'échappement 
est donc une lacune importante de cette méthodologie pour laquelle des améliorations
seraient bienvenues.
Cependant, il est très peu probable que les raffinements futurs des connaissances des sources de
PM changent considérablement la vision générale obtenue dans cette étude. L'identification de nouveaux
facteurs ne viendrait a priori que subdiviser une typologie de source déjà identifiée.

Nous pensons que les études présentées dans cette thèse permettent de répondre en partie à
la question de l'exposition générale en Europe de l'Ouest, mais la variabilité temporelle
et spatiale se limite cependant au passé et  à des milieux majoritairement urbains.
Cette connaissance générale ne doit cependant pas occulter les particularités locales des
différents environnements.
La recherche de nouveaux traceurs spécifiques pourra permettre l'identification de sources
ponctuelles et statistiquement peu présentes spatialement et temporellement
(industrielle, portuaire, volcanique, etc.) et présente donc un intérêt scientifique et
politique certains.
De même, l'utilisation de modèle déterministe CTM sera une étape importante pour la
prévision du potentiel oxydant mais également son utilisation par des études
épidémiologiques caractérisant les liens entre exposition et impacts sanitaires.

Cependant, bien que la mesure du potentiel oxydant soit très prometteuse et présente une
étape importante, il est important de se souvenir que la diversité des polluants de l'air
n'est pas entièrement quantifiée par cette métrique.  Par exemple, certains ``polluants
émergents'' comme les pesticides ne sont pas redox-actifs et ne contribuent que très peu à
la masse des aérosols. Ils restent ``invisibles'' par ces différentes métriques. D'autres
familles de polluants comme les perturbateurs endocriniens ou les micro-plastiques pourraient
également ne pas nécessairement être visibles par ces mesures.

Cette thèse participe donc à l'évaluation et quantification des sources
du potentiel oxydant des particules comme étape vers une meilleure prise en compte de
l'exposition et impacts de la qualité de l'air. Cependant, de nombreuses questions restent
ouvertes ou non adressées dans cette recherche. Méthodologiquement, le modèle simple de
régression linéaire n'est peut-être pas adapté, et d'autres types de modèle apporteront
peut-être une vision complémentaire. Aussi, les liaisons entre chimie de l'aérosol,
toxicologie et impacts sanitaires sont encore peu connues. Or, l'utilisation du potentiel
oxydant comme indicateur sanitaire passera nécessairement par une confrontation des
différents tests de mesures aux effets délétères de la qualité de l'air. Cette
confrontation permettra certainement également de répondre à la question essentielle du
``meilleur'' tests de potentiel oxydant, qui reste pour l'instant sans réponse.
