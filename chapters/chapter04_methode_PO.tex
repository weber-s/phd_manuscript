\section{Introduction}%
\label{sec:intro_PO}

\subsection{Quelle métrique pour l'impact sanitaire ?}%
\label{sub:quelle_métrique_pour_l_impact_sanitaire_}


Bien que dans les chapitres précédents nous avons vu qu'il était possible d'estimer de
façon fiable la contribution des différentes sources de PM grâce aux PMF et d'établir une
phénoménologie des sources de \PMdix, la question de leur effet sanitaire n'est toujours
pas répondu. En effet, l'ordre de contribution à la concentration moyenne annuelle des
\PMdix{} présentée par \cite[(figure 3)]{weberComparison2019} ne préjugage pas de leurs
impacts sanitaires.

En effet, comme détaillé dans en introduction
section~\ref{sec:le_potentiel_oxydant_des_aerosols}, la mesure de la masse des PM n'est
certainement pas l'indicateur le plus adapté pour évaluer leur toxicité, car la composition
chimique, forme, surface réactive, etc. n'est pas prise en compte par cette métrique.
C'est pourquoi le potentiel oxydant (PO ou OP en anglais), mesurant indirectement les
espèces réactives de l'oxygène (ERO ou ROS en anglais) apportées ou induites par les PM,
est proposé comme nouvel indicateur de l'exposition des populations à la toxicité des PM. 
Il est maintenant bien documenté que les différents tests de PO présentent une information
différente de la concentration massique (voir par exemple,
\cite{choRedox2005,vermaReactive2014,batesReactive2015,fangOxidative2016,fangAmbient2017,calasSeasonal2019},
ou la revue détaillée récente de \cite{batesReview2019}), comme rappelé dans le
tableau~\ref{tab:calas_2018_spearman} sur une étude à Chamonix par
\cite{calasComparison2018} présentant la corrélation entre la masse des \PMdix{} et 5
mesures de PO.

Il est également connu que les différentes espèces chimiques constitutives des PM ne
réagissent pas à un même test de PO de la même manière. Notamment, les métaux de
transitions ainsi que certaines quinones conduisent à la formation d'un grand nombre de
\ce{HO^.} et sont donc des espèces très réactives à la mesure du PO
\autocite{charrierRates2015,calasImportance2017}.

\begin{table}[ht]
    \begin{ThreePartTable}
        \centering
        \caption{Corrélation de Spearman entre 5 tests de PO et la masse des \PMdix{} sur
            le site de Chamonix (2013) séparer en période chaude (triangle bas) et froide
            (triangle haut).\\
            Source : \cite[Table 3]{calasComparison2018}
        }
        \label{tab:calas_2018_spearman}
        \footnotesize
        \begin{tabular}{lSSSSSS}
            \toprule
             & {\PMdix} & {OP DTTv\tnote{1}} & {OP AAv\tnote{1}} & {OP ESRv\tnote{2}} & {OP GSHv\tnote{1}} & {OP ASCv\tnote{3}}\\
             \midrule
            \PMdix  &           & 0.91{***} & 0.91{***} & 0.59{***} & 0.87{***} & 0.90{***}\\
            OP DTTv & 0.71{***} &           & 0.89{***} & 0.61{***} & 0.79{***} & 0.72{***}\\
            OP AAv  & 0.43{*}   & 0.65{***} &           & 0.54{***} & 0.85{***} & 0.79{***}\\
            OP ESRv & 0.088     & 0.17      & 0.36      &           & 0.56{**}  & 0.59{**}\\
            OP GSHv & 0.44{*}   & 0.29      & 0.36      & 0.63{*}   &           & 0.92{***}\\
            OP ASCv & 0.38{*}   & 0.37{*}   & -0.072    & -0.29     & 0.17      & \\
            \bottomrule
        \end{tabular}
        \begin{tablenotes}
        \item[] *** p < 0.001 level, ** p < 0.01 level, * p < 0.05 level
        \item[1] n = 30 (cold period), n = 29 (warm period)
        \item[2] n = 30 (cold period), n = 14 (warm period)
        \item[3] n = 27 (cold period), n = 29 (warm period)
        \end{tablenotes}
    \end{ThreePartTable}
\end{table}

\subsection{Un PO par espèces chimiques ?}%
\label{sub:un_po_par_espèces_chimiques_}

Cependant, lorsque l'ensemble des PM est solubilisé, analysé et confronté aux PO, de
fortes corrélations sont bien observés avec les espèces présentant une forte réactivité aux
PO, mais certaines autres espèces sont parfois fortement corrélées alors qu'une mesure
directe de leurs PO montre aucune réactivité --c'est notamment le cas du nitrate,
de l'ammonium ou de lévoglucosan \autocite{vermaRedox2009,calasComparison2018,calasSeasonal2019}.

Ces corrélations ne reflètent donc pas de réelles causalité mais des co-corrélations. Le
lévoglucosan est émis en même temps que certaines quinones lors de la combustion de bois,
qui elles sont redox-active, et le nitrate est émis temporellement en fin d'hiver-début
printemps, où la combustion de biomasse est toujours présente. 
Ainsi, ces corrélations sont un premier pas vers l'identification des sources majoritaires
de PO, mais s'avère insuffisant.

\subsection{Du PO par espèces chimiques au PO par sources}

L'attribution d'un PO intrinsèque par espèces chimiques est donc estimée par mesure
directe ou par regression linéaire
multiple~\autocite{calasImportance2017,borlazaOxidative2018}. La première solution permet
la mesure exacte du PO de l'espèce mais présente des limitations analytiques évidentes
(composé pur, grand nombre d'espèces et de gamme de concentration à étudier), et la
deuxième peut présenter des ``faux positifs'' dans le cas où toutes les espèces chimiques
ne sont pas mesurées --ce qui est toujours le cas-- du fait de l'émission conjointe
d'espèce redox-active et non-redox-active (quinones et lévoglucosan par exemple).

Aussi, d'un point de vue réglementaire, il est souvent plus compréhensible d'agir sur
un secteur d'émission que sur une espèce chimique en particulier.

Pour ces raisons, l'agrégation de l'information géochimique par source d'émission plutôt
que par espèce chimique est intéressante :
\begin{enumerate}
    \item le nombre d'inconnues diminue drastiquement (plusieurs milliers d'espèces
        chimique à une dizaine de sources d'émission);
    \item pas besoin d'identifier chaque espèces chimiques du moment que les sources
        majoritaires sont bien représentées;
    \item la contribution des sources aux potentiels oxydants est davantage transmissible
        en termes de politiques publiques que la contribution d'espèces chimiques;
\end{enumerate}

Pour estimer un PO par sources, plusieurs méthodes sont possibles : 1) introduire le PO
comme variable dans une étude de source (CMB, PMF, etc.) 2) faire une étude de source
(CMB, PMF, etc.) grâce à l'information chimique et ensuite établir un modèle d'inversion
entre les sources identifiées et les PO.

\cite{vermaReactive2014} ont utilisé les 2 méthodes en introduisant le
\PODTT{} dans une PMF avec comme autres espèces WSOC\footnote{WSOC: Water soluble
organic carbon}, BrnC\footnote{BrnC: Brown Carbon}, EC, \SOq, \NHt, K, Ca, Mn, Fe, Cu et
Zn, mais également en établissant une régression linéaire entre le \PODTT{} et les sources
issues du CMB.
Entre 2014 et 2017, \cite{batesReactive2015} puis \cite{fangOxidative2016}, de la même
équipe de recherche, ont repris ce principe et l'ont appliqué à plus grande échelle
spatiale et également au \POAAv.

\begin{tcolorbox}[colback=red!5!white,colframe=Melon,title=Note]
    Lors du commencement de ma thèse, à ma connaissance, seules ces 3 études étaient
    disponibles dans la littérature. Ce sujet s'est vu largement développés au cours des 3
    dernières années, et sera discutée dans la suite.
\end{tcolorbox}

\begin{figure}[ht]
    \centering
    \includegraphics[width=1.0\linewidth]{figures/chapter04/verma_2014_fig8.pdf}
    \caption{Première estimation de la contribution des sources de PM au \PODTT{} par
        \cite[][figure 8]{vermaReactive2014}: \textbf{a)} par ajout du DTT dans la PMF et
        \textbf{b)} par régression linéaire entre \PODTT{} et CMB.
    }%
    \label{fig:figures/chapter04/verma_2014_fig8}
\end{figure}

Seulement, nous avons vu que les PMF sont très sensible aux variables utilisées. Or, leurs
études PMF inclut un ensemble d'espèce chimiques restreint (une dizaine de métaux et le
WSOC), en plus du PO. Seuls 4 facteurs sont identifiés par \cite{fangOxidative2016}, ce
qui semble peu au regard des PMF utilisant un jeu d'espèce chimique plus conséquente (voir
notamment tous le chapitre précédent). En plus, ce faible nombre d'espèces chimiques
associés à une nouvelle variable, le PO, dont nous ne connaissons a priori que peu de chose de ses
sources, peut conduire à des solutions instables statistiquement.

La solution utilisant la régression linéaire à partir du CMB hérite des limitations
propres au CMB : peu de prise en compte des particularités locales, connaissance a priori
forte, etc (voir section
~\ref{ssub:atouts_et_limitations_des_différents_modèles_récepteurs}).
Ainsi, \cite{batesSource2018} utilise un modèle CMB pour prédire à large échelle spatiale
(Amérique du Nord) le PO, mais le modèle établit présente de faible performance
statistique ($r^2 = 0.36$ et intercepte non nul), notamment du fait de l'oubli potentiel de
sources importantes (bioaérosols \autocite{samakeUnexpected2017}, etc).

\subsection{Couplage de PMF avancée avec l'estimation du PO}%
\label{sub:couplage_de_pmf_avancée_avec_l_estimation_du_po}

Dans ce chapitre, je propose tout d'abord une méthodologie permettant de coupler les
connaissances acquises sur les PMF présentées
dans le chapitre~\ref{cha:approfondissement_des_connaissances_des_sources_des_pm}
précédent, en n'ajoutant pas le PO comme variable explicative pour ne pas perturber le
modèle, puis d'évaluer le PO intrinsèque (i.e. par microgramme) de chacune des sources
identifiées, illustré par la figure~\ref{fig:workflow_inversion}.

Le site de Chamonix entre 2013 et 2014 a été choisi pour tester cette méthode du fait
d'une étude PMF approfondie par \cite{chevrierChauffage2016} et par les mesures de PO sur
ce site effectué lors de la thèse de \cite{calasPollution2017}, aussi bien pour le \POAA{}
que le \PODTT, et a fait l'objet d'une publication~\autocite{weberApportionment2018}
présentée dans la section~\ref{sec:weber_et_al_2018}.

Cette méthode ayant présenté des résultats très encourageants, son application à un
ensemble de 15 séries de prélèvement réparties sur 14 sites en France métropolitaine et
discutée dans la section~\ref{sec:synthèse_grande_échelle}.
Cette partie reprend un article en cours de soumission~\autocite{weberSourceinprep.} et a
déjà été présentée à l'EAC de 2019 à Göteburg \autocite{weberSources2019}.

\begin{figure}[ht]
    \centering
    \includegraphics[width=0.8\linewidth]{figures/chapter04/flowchart_inversion.pdf}
    \caption{Processus suivi afin d'estimer la contribution des sources de PM aux
        potentiels oxydants. La méthode d'inversion utilisée dans ce chapitre est une
        régréssion linéaire multiple, permettant d'attribuer un PO par microgramme de source
        (\textit{PO intrinsèque}), estimant la ``toxicité de la source'', et la contribution de
    chacune des sources aux PO, estimant l'exposition de la population à cette source.}%
    \label{fig:workflow_inversion}
\end{figure}


\section{Climatologie du PO}%
\label{sec:climatologie_du_po}

\todo{\cite{calasSeasonal2019}}

\section{Développement méthodologique à Chamonix}%
\label{ssub:développement_méthodologique_à_chamonix}

\clearpage
\subsection{An apportionment method for the oxidative potential of atmospheric particulate
matter sources: application to a one-year study in Chamonix, France}
\label{sec:weber_et_al_2018}

Article paru dans le journal \textit{Atmospheric Chemistry and Physics} le 4 juin 2019 :

\begin{quote}
    Samuël Weber, Gaëlle Uzu, Aude Calas, Florie Chevrier, Jean-Luc Besombes,
    Aurélie Charron, Dalia Salameh, Irena Ježek, Griša Močnik, et Jean-Luc Jaffrezo. 2018.
    \textit{An Apportionment Method for the Oxidative Potential of Atmospheric Particulate
    Matter Sources: Application to a One-Year Study in Chamonix, France}. Atmospheric
    Chemistry and Physics 18(13), pp. 9617‑9629.
    \textsc{doi} : \href{https://doi.org/10.5194/acp-18-9617-2018}{10.5194/acp-18-9617-2018},
    \textsc{url} : \url{https://www.atmos-chem-phys.net/18/9617/2018/}
\end{quote}

Les profils PMF, la corrélation entre espèces chimiques et PO et sources et
PO à titre de comparaison avec les études antérieures, sont présentés en complément de
l'article, repris en annexe~\ref{annexe:deconvol_OP_SI}.

\includepdf[pages=-,scale=0.95,pagecommand={\pagestyle{fancy}}]{chapters/deconvol_OP.pdf}


\subsection{Conclusion}

Ce chapitre prouve tout d'abord qu'il est possible de déterminer les sources de PO grâce à
une étude couplée entre une PMF avancée utilisant différents traceurs organiques résultant
en 8 facteurs distincts et une régression linéaire multiple prenant en compte
la mesure du PO à iso-masse utilisant des conditions de bioaccessibilité proche du milieu
pulmonaire et ses incertitudes, avec une très bonne performance statistique.

Cette méthode permet de différencier très nettement les sources de PM contribuant aux PO
et apporte une vue nouvelle de l'aérosol en redistribuant l'importance de la contribution
des sources. Conformément aux résultats des 3 études précédentes faites aux États-Unis
\autocite{vermaReactive2014,batesReactive2015,fangOxidative2016}, la combustion de
biomasse domestique et le transport routier représentent les 2 sources principales de
\POAAv{} et \PODTTv.

L'utilisation de deux tests de PO nous permet aussi d'affirmer que ces 2 tests ne portent
pas en eux exactement les mêmes informations géochimiques, car les sources de PM les
expliquant diffèrent --bien que la même tendance générale est observée. En l'absence
d'étude approfondie quant aux liens toxicologie-PO ou épidémiologie-PO, le maintient de
différents tests de PO est donc recommandé.

Aussi, certaines corrélations observées, aussi bien avec des espèces chimiques que des
facteurs PMF, s'avère comme attendue trompeuse. Le facteur nitrate-rich était corrélé au
\POAAv{} mais présent en réalité un PO intrinsèque quasi-nul. Inversement, le facteur
secondaire biogénique, tracé par le MSA, est négativement corrélé aux deux \OPv{} alors
qu'il présente le 2\ieme{} \PODTT{} intrinsèque le plus élevé. Il est donc nécessaire
d'utiliser des méthodes statistiques plus avancées que la régression uni-variée lorsque
l'on cherche à estimer les sources de PO afin de s'affranchir des tendances saisonnières et
des covariations entre espèces.

Cette méthode est donc validée comme outil de déconvolution des sources de PO, et le
prochain chapitre traitera de son application à un large panel d'environnement et année
d'étude, afin d'établir à la manière du chapitre précédent, une phénoménologie non pas des
sources de la masse \PMdix, mais des sources de PO.


\section{Pertinence géochimique à grande échelle des sources de PO}%
\label{sec:synthèse_grande_échelle}

\subsection{Introduction}%
\label{sub:introduction_synthèse_nationale}

La méthode d'estimation des sources de PO établie dans la section précédente permet donc
d'estimer efficacement les PO intrinsèques des différentes sources de PM déterminer par
PMF, avec de meilleures performances statistiques que les études similaires précédentes.
Cela tient certainement au fait de la meilleure prise en compte des différentes sources de
PM grâce à une PMF plus détaillée que celles présentées dans les études précédentes et non
perturbée par l'ajout des PO dans le système d'équation de la PMF.

La question suivante est donc naturellement la généralisation ou non de ce résultat.
Est-il possible d'estimer correctement un PO intrinsèque pour chacune des sources de PM
estimer par PMF, indépendamment du lieu de prélèvement ? Si oui, est-ce que chaque source
de PM présente un PO intrinsèque similaire de site en site, avec quelle variabilité ?

Pour répondre à cette question, un ensemble de 14 sites de prélèvement à minima annuel, couvrant les
années 2013 à 2018 et différents types d'environnements (urbain, trafic, vallée alpine),
a été choisi pour estimer de manière standardisée leurs sources de PM grâce à la
méthodologie développée dans le cadre du programme SOURCES présentée
section~\ref{sub:article_SOURCES}. En effet, la PMF présentée précédemment sur le site de
Chamonix n'est pas facilement généralisable, car les espèces chimiques utilisées ne sont
pas disponibles pour un grand nombre de site (hopanes, aethalometre, methoxyphénol).

Ainsi, les sites du programme SOURCES sur lesquels suffisamment de filtres étaient
disponibles ont été analysées en \POAA{} et \PODTT{}. Afin d'enrichir et améliorer la
représentativité spatiale de l'étude, j'ai également pris en compte de nouvelles études
PMF similaire à la méthodologie SOURCES, mais faite dans le cadre de cette étude, pour les
sites de Passy, Marnaz, GRE-fr (2017), GRE-cb, et Vif.
Aussi, le site de Rouen présentant un résultat PMF indiquant de potentiel mélanges de
facteurs a été écarté de cette étude \autocite{weberComparison2019}.

C'est donc plus de 1700 filtres répartis sur 14 sites de prélèvements et 15 séries
annuelles complètes (voir tableau 1 de l'article suivant) analysées analytiquement suivant
un protocol standard à tous les échantillons et statistiquement par des PMF harmonisées
qui permettront, ou non, la généralisation de cette méthode de déconvolution des sources
de PO.

\begin{tcolorbox}[colback=red!5!white,colframe=Melon,title=Note]
    Devant la quantité de résultat à présenter (résultat PMF, similitude géochimique des
    facteurs entre sites, séries temporelles des PO, performance statistiques,
    contribution saisonnières des sources aux PO, etc.), le choix a été pris de rendre
    disponible l'ensemble des données et de facilité leur visualisation à travers le site
    \url{http://getopstandop.u-ga.fr}.
    Les résultats agrégés sont présentés dans l'article suivant, mais le détail de chacun
    des sites et facteur PMF peut être retrouvé sur le site.
\end{tcolorbox}

\subsection{Article}%
\label{sub:article}

Article actuellement en préparation pour \textit{Atmospheric Chemistry and Physics} :

\begin{quote}
    Samuël Weber, Gaëlle Uzu, Aude Calas, Dalia Salameh, Florie Chevrier, Julie Allard,
    Jean-Luc Besombes, Olivier Favez, (Les représentants des différentes AASQA), et
    Jean-Luc Jaffrezo. in prep.
    \textit{Source apportionment of the oxidative potential of aerosols at 15 French
    sites for yearly time series of observation}.
\end{quote}

\includepdf[pages=-,scale=0.95,pagecommand={\pagestyle{fancy}}]{chapters/article_allOP/article_nuage.pdf}

\subsection{Conclusion}%
\label{sec:conclusion_synthèse_OP}

\subsubsection{Une méthodologie statistiquement et géochimiquement validée}%
\label{ssub:une_méthodologie_statistiquement_et_géochimiquement_validée}

La méthode d'estimation des sources de \POAA{} et \PODTT{} des \PMdix{} couplant des
analyses PMF avancées et un modèle d'inversion linéaire prenant en compte les incertitudes
des mesures de PO présente des résultats statistiques très satisfaisant pour l'ensemble
des sites étudiés ($r^2>0.7$ et intercept faible, à l'exception notable de VIF (\PODTT) et
STG-cle (\POAA)).

Les mêmes sources de \PMdix{} retrouvées grâces à des PMF harmonisées présentes un potentiel
oxydant intrinsèque (i.e. par microgramme de PM) similaire, et ce, pour différents types
d'environnements (urbain, industriel, trafic et vallée alpine). Une étude à si large
échelle spatiale et temporelle, utilisant à la fois le \POAA{} et \PODTT{} nous permet
donc de conclure que les PO sont bien distincts selon la source --et donc géochimie--
considérés, et chaque type de source présent un PO intrinsèque déterminé et indépendant du
site d'étude.

Ce résultat était un préalable nécessaire à la modélisation du PO par les modèles
déterministes CTM afin d'établir une couverture spatiale et temporelle beaucoup plus
large que les études sur sites, permettant à terme des études épidémiologiques sur la
qualité de l'air à travers la métrique du PO.

\subsection{Une redistribution de l'importance des sources de PM}%
\label{sub:une_redistribution_de_l_importance_des_sources_de_pm}

Sans attendre la modélisation CTM, il est d'ors et déjà possible de dire que les
différentes sources de PM présentes des PO intrinsèques distincts et qu'il y a une
redistribution complète de l'importance de la contribution des sources aux \PMdix{} selon
la métrique d'observation choisi : masse, \POAAv{} ou \PODTTv. Les sources inorganiques
secondaires (nitrate-rich et sulfate-rich) contribuant significativement à la masse de PM
ne contriue en définitive que peu aux PO. Ainsi, sous l'hypothèse que
la métrique du PO permet une meilleure estimation de la toxicité des PM, les sources
d'émissions d'importances sanitaires présentes sur l'ensemble de la France se trouvent être
principalement 2 sources d'émissions primaires anthropiques : le trafic routier et la
combustion de biomasse domestique. Ces résultats concordent avec les études précédentes de
\cite{batesReactive2015,fangOxidative2016}, et celles plus récentes de
\cite{paraskevopoulouYearlong2019,cesariSource2019}. Il est également intéressant de
voir que l'étude de \cite{cesariSource2019} a confronté les approches 2 approches ``PO comme variable
dans la PMF'' ou ``PO estimé par régression linéaire des résultats PMF'', et lors
de l'ajout du \PODTTv{} dans la PMF, la source traffic se voit attribuer davantage de Fe
et Cr et beaucoup moins de nitrate et ammonium.

Il est a noter que ces 2 sources ont cependant des dynamiques très différentes. L'émission
primaires du traffic présente un fort PO intrinsèque pour les deux tests de PO.
En revanche, si la combustion de biomasse domestique présente un \POAA{} intrinsèque du
même ordre que le traffic routier, son \PODTT{} intrinsèque est près de 2 fois plus
faible. Aussi, l'importance de la combustion de biomasse vient principalement du fait des
fortes concentrations de cette source durant l'hiver. L'exposition à cette source est donc
importante en hiver mais absente en dehors de cette saison à l'inverse de l'émission
primaire du traffic présentant de plus faible concentrations mais tout au long de l'année.
Selon l'importance d'une exposition aigüe ou chronique aux PO, il conviendra donc de
cibler d'abord l'une ou l'autre de ces sources.

Parmi les sources naturelles, les émissions primaires biogéniques et les poussières
crustales présentes également une activité rédox importante pour le test au \PODTT, mais
pas pour le \POAA, retrouvant ainsi en condition ambiantes les résultats de mesure de PO
sur les bioaérosols de \cite{samakeUnexpected2017}.

Cependant, les processus secondaires semblent également pouvoir jouer un rôle important
dans le PO ambiant, mais l'incertitude liée à cette source est trop grande pour en tirer
des conclusions avec les connaissances dont nous disposonti actuellmeent. Cette
incertitude est très certainements liée à la difficulté de la prise en compte des sources
ou processus conduisant à la formation d'aerosols organiques secondaires par les PMF avec
spéciation chimique sur filtres.
\todo{une figure avec PO avec acide orga ?}


\section{Conclusion}%
\label{sec:conclusion_chap4}

\subsection{Une nouvelle vision de l'aérosol, plus proche de l'impact sanitaire}%
\label{sub:une_nouvelle_vision_de_l_aérosol_plus_proche_de_l_impact_sanitaire}

Grâce à la mise en place et l'harmonisation des nombreuses mesures de chimie et de PO
ainsi que des résultats issus PMF au sein d'une base de donnée unique, il a été possible
d'établir une phénoménologie des sources de \POAA{} et \PODTT{} à grande échelle spatiale.

Au cours des dernières années, cette base de donnée s'est considérablement enrichie et
d'autres groupes de recherches ont commencé à documenter à large échelle spatiale des
résultats similaires en Italie~\autocite{pietrograndeReview2019} ou en
Suisse\footnote{Mesures effectuées par l'IGE} grâce au
PSIr~\autocite{daellenbachSourcessubmitted} ou l'EMPA et des travaux sont également en
cours pour une collaboration avec le TNO au Pays-bas et l'IDAEA en Espagne dans le cadre
de l'ANR GetOP-StandOP portée par Gaëlle Uzu.
\todo{est-ce qu'on peut dire ca?}

À travers ces travaux et l'établissement de méthode de déconvolution des sources de PO, il
est montré que la vision que l'on a de l'aérosols à travers la métrique de la
concentration massique doit être repensé lorsque l'on s'intéresse à l'impacte sanitaire
des sources de PM. Les sources contribuants majoritairement à la masse des PM n'étant pas
nécessairement celles contribuant le plus aux \POAA{} ou \PODTT, et inversement.
Ainsi, au cours des dernières années, les études aux \POAA{} et \PODTT{} de
\cite{vermaReactive2014,batesReactive2015,fangOxidative2016,weberApportionment2018,cesariSource2019,daellenbachSourcessubmitted,weberSourceinprep.}
semblent indiquer la prédominance de 2 sources anthropiques majoritaire pour l'exposition
aux PO : le traffic routier et la combustion de biomasse domestique.

\subsection{Limitation et idées pour le futur}%
\label{sub:limitation_et_idées_pour_le_futur}
\todo{trouver un meilleur titre}

Seulement, 


Limitation des méthodes linéaires alors qu'on sait le PO pas linéaire

Nécessité de descendre en infra journalier.

