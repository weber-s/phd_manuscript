\section*{Introduction}%
\label{sec:introduction}
\addcontentsline{toc}{section}{Introduction}


Les travaux présentés dans ce chapitres ont été menés afin d'approfondir les connaissances
des processus d'émissions de différents composés des PM. En effet, le modèle PMF utilisé
pour tracer l'origine des sources d'aérosol nécessite une base de connaissance préalable à
son utilisation car comme nous l'avons vu, plusieurs paramètres doivent être choisi par
l'expérimentateur, notamment les variables servant d'entrainement au modèle.

Un modèle étant nécessairement une représentation simplifiée de la réalité, il convient de
trouver le bon équilibre entre un modèle trop complexe, qui serait difficilement
interprétable, et un modèle trop simple, qui n'apporterait pas de nouvelles informations.
Le cas de la PMF, et le \textit{machine learning} en général, n'échappe pas à cette règle.
Il faut en effet choisir soigneusement les variables d'entrée du modèle pour lui permettre
d'extraire des informations géochimiquement pertinentes d'un ensemble de donnée. Cela se
traduit par la détermination d'espèce traceuse de processus d'émissions ou de
transformations dans l'atmosphère, puis de leur utilisation conjointes dans le modèle PMF,
permettant alors d'isoler de nouveaux facteurs et de rafiner la contribution des autres
espèces aux facteurs restants.

Aussi, un modèle procède toujours à une étape de validation. Dans notre cas, nous n'avons
pas de référence pouvant servir de témoin positif. Nous pouvons cependant nous servir de 2
procédés : 1) la confrontations à d'autres méthodes de \textit{sources-apportionment},
indépendantes de la PMF (carbone 14 (\textcite{bonvalotEstimating2016} et
\textcite{chevrierChauffage2016}), Aethalometre, etc) ou 2) estimer la fiabilité des
résultats PMF par cohérence géophysique, par exemple en retrouvant l'origine géographique
des sources d'émissions.

Les travaux de cette thèse ont ainsi conduit à différents développement ou applications
méthododologiques et techniques.
\begin{enumerate}
    \item Nous verrons dans un premier temps que l'origine géographique des masses d'air
        par méthode PSCF a permis de consolider les solutions obtenues par méthodologie
        PMF, mais a également apportée une vision nouvelle de la provenance du MSA,
        supposé jusqu'alors d'origine exclusivement marine~\autocite{gollyOrganic2019}.
    \item Dans un second temps, nous expliquerons en quoi le recensement et
        l'harmonisation de la base de donnée de filtre de prélèvement ambiant a été
        utilisé afin de généraliser des observations à un ensemble de 28 sites de
        prélèvements et a conduit à la quantification des processus d'émissions
        biogéniques primaires~\autocite{samakePolyols2019,samakeArabitol2019}.
    \item Enfin, nous présenterons un travail en cours sur la variabilité fine échelle des
        sources de PM et l'importances des facteurs d'oxydations secondaires, identifiés
        par l'ajout de nouveaux traceurs organiques dans la
        PMF~\autocite{borlazaSourceinprep}.
\end{enumerate}
\todo{Et INACS dans tout ca ?}


\section{Provenance géographique des composés chimiques}%
\label{sec:provenance_géographique_des_composés_chimiques}


\section{Généralisation d'observations}%
\label{sec:généralisation_dobservations}

\subsection{Mise en place d'une base de donnée harmonisée}%
\label{sub:mise_en_place_d_une_base_de_donnée_harmonisée}



\section{Amélioration des solutions PMF grâce à de nouveaux traceurs organiques}%
\label{sec:amélioration_des_solutions_pmf_grâce_à_de_nouveaux_traceurs_organiques}


\printbibliography[segment=\therefsegment,heading=subbibliography]
