
\section{Stratégie de la thèse (et du groupe CHIANTI)}%
\label{sec:stratégie__du_groupe_chianti}

\subsection{Place de ma thèse dans les problématiques de CHIANTI}%
\label{sub:place_de_ma_thèse_dans_les_problématiques_de_chianti}

\todo{Dessein inteligent, grand manitou, etc.}
Le groupe de recherche de chimie atmosphérique, neige, transfert et impacts (CHIANTI) au
sein duquel j'ai effectué ma thèse s'emploie ``à identifier les sources, les puits et
mécanismes de transformations des espèces chimiques générées par les activités humaines
(en particulier) afin de déterminer leur impact sur le climat, la composition et la
qualité de l’air, et les écosystèmes
enneigés''\footnote{\url{http://www.ige-grenoble.fr/-Chimie-atmospherique-CHIANTI-}}.
Le cadre de ma thèse concerne donc une sous partie de ces recherches, portant sur l'étude
de la qualité de l'air.

Une partie des travaux concerne un aspect fondamental de compréhension des processus
conduisant à la présence d'aérosols dans l'atmosphère, par le biais de prélèvement sur le
terrain et d'analyse sur filtre par le plateau analytique Air-O-Sol à travers notamment
l'analyse de la fraction organique des PM (voir
section~\ref{sec:methodologie_de_prélèvement_et_d_analyse}).
Les mesures enregistrés dans la ``filtrothèque'' porte actuellement sur \num{18000}
filtres répartis sur plus de 80 sites différents et ont été possible grâce à différents
partenariats.

Ces partenariats avec les AASQUA ou d'autres instituts (LCSQA, INERIS, ADEME, ANSES...),
constitue également un volet important des recherches portant sur la reglementation et sur
la connaissance des processus et la quantification des sources d'émissions conduisant aux
concentrations de PM en air ambiant.
Différents programme de recherche ont ainsi été élaboré en vue d'une amélioration de la
qualité de l'air (PRIMEQUAL, DECOMBIO, etc.) à travers les plans de protections de
l'atmosphère (PPA). À titre d'exemple, l'importance
de la combustion de biomasse domestique en vallée alpine a pu être démontré, conduisant
entre autre à la mise en place de la prime Air-bois, incitatif au remplacement des vieux
poele à bois. Les conséquences de cette action sont suivies et
analysés au sein du laboratoire depuis plusieurs années maintenant.

Finallement, nous essayons aussi de nous rapprocher de l'impacte sanitaire des PM à
travers le développement de méthode d'analyse du potentiel oxydant (voir
section~\ref{sub:potentiels_oxydants}. En plus du développement méthodologique de leurs
mesures, l'analyse sur des PO sur plus de 45 sites de prélèvement différents, pour un
total de plus de 6500 échantillons dont le PO a été mesuré.
L'avantage majeur de la mesure du PO sur filtre consiste en la concomitance des mesures de
PO et de chimie, permettant des recherches couplé sur la chimie de l'aérosols et de son
potentiel oxydant et à terme de mieux comprendre la variabilité inter-analyse des différentes
méthodes de quantification du PO et leurs liens avec d'autres variables métrologiques (par
exemple les connexions avec les drivers géochimiques). Parallellement, la pertinence de la
mesure du PO pour comme métrique sanitaire et sa comparaison avec la variable
réglementaire en vigueur (la concentration des PM) n'est pas encore totalement établit.
Les travaux du groupe son donc également focalisé sur ce point, notamment par des travaux
pluridisciplinaires associant géochimistes et épidémiologiste.

\subsection{Méthodologie générale}%
\label{sub:méthodologie_general}

Ma thèse s'inscrit dans la continuité logique du groupe de recherche et est transverse aux
différentes thématique abordée, à savoir la détermination des sources d'émission d'intérêt
sanitaire.

Pour ce faire, la méthodologie générale suivante est donc adoptée et résumée sur le schéma
synthétique~\ref{fig:chapter01/workflow} :

\begin{enumerate}
    \item Grâce aux mesures de chimie et le modèle PMF, retrouver les sources d'émissions
        de PM;
    \item Coupler les données issues de la PMF et les mesures de PO pour construire un
        modèle d'inversion permettant l'estimation de la contribution des différentes
        sources aux potentiels oxydants ;
    \item Estimer la variabilité géographique du potentiel oxydant des sources de PM.
\end{enumerate}

Chacun de ces items seront abordés dans les chapitres dédiés.

\begin{figure}[ht]
    \centering
    \includegraphics[width=0.5\linewidth]{chapter01/workflow.png}
    \caption{Méthodologie générale suivie au cours de cette thèse.}%
    \label{fig:chapter01/workflow}
\end{figure}


\subsubsection{Choix du modèle d'attribution de source}%
\label{ssub:choix_du_modèle_d_attribution_de_source}

Le choix du modèle source-récepteur PMF est motivé par son aptitude à reproduire les
concentrations locales sans préjugés a priori de la situation du site de prélèvement. La
capacité du ME-2 à introduire des contraintes géochimiques est également un atout
important justifiant l'usage du modèle PMF.
Enfin, les précédents travaux au sein de l'équipe CHIANTI d'Antoine Waked, Benjamin Golly,
Flory Chevrier et Dallia Salameh sur le modèle PMF représentent également une expertise
appréciable dans le choix de cette méthode.

\subsubsection{Échantillons et analyses}%
\label{ssub:échantillons_et_analyses}

Les prélèvements, mesures de concentrations des différentes espèces chimiques et
mesures du PO s'appuieront sur les travaux préalables et en cours de l'équipe CHIANTI et
particulièrement du plateau analytique Air-O-Sol ainsi que la base de donnée accumulée
depuis plusieurs année et en construction permanente.

\subsubsection{PO et chimie ou PO et sources ?}%
\label{ssub:chimie_ou_sources_}

Le choix a été fait de ne pas utiliser directement les espèces chimiques comme prédicteur
des PO. En effet, sur la myriade d'espèces chimiques présente sur les aérosols, seule une
partie infime est effectivement mesurée. Il n'est donc pas possible d'avoir une vue
exhaustive de la chimie des PM et l'attribution d'un PO intrinsèque par espèce chimique
présentera nécessairement des biais du fait de corrélation entre une espèce mesurée mais
non redox-active (i.e. le lévoglucosan) et des espèces co-émise mais non mesurée (i.e. les
quinones). Quand bien même toutes les espèces seraient mesurées, on se retrouverait alors
avec un système d'équation a beaucoup trop d'inconnue par rapport au nombre
d'observation (i.e jours de prélèvement), rendant caduque sa résolution.
À contrario, les facteurs PMF aggrègent toutes cette chimie en quelques variables, tout en
n'ayant besoin que de quelques espèces traceuses des émissions ou processus. Par exemple,
la combustion de biomasse est déterminée à partir de la présence de lévoglucosan. On pourra
donc attribuer un PO à la source \textit{combustion de biomasse}, avec concentration en PM
connue et un profile chimique déterminé (et partiellement inconnue), tout en ne sachant
pas exactement quelles espèces chimiques de cette source sont responsables de son PO.



\section{Methodologie de prélèvement et d'analyse}%
\label{sec:methodologie_de_prélèvement_et_d_analyse}


\subsection{Un filtre pour les gouverner tous}%
\label{sub:un_filtre_pour_les_gouverner_tous}

Les données utilisées pour cette thèse proviennent d'analyse faite en laboratoire à
partir de prélèvement de terrain (mesure \textit{off-line}) via des préleveurs
automatiques haut-volume, selon les recommandations de
l'EN~16450:2017~\autocite{cenAmbient2017a}, chaque prélèvement correspondant à une
journée d'échantillage. 

L'entiéreté des analyses des différents composé chimique et de PO se fera sur ce même
filtre, donc différent morceaux seront prélevés, comme le montre la
figure~\ref{fig:chapter02/filter_speciation_en}.

\begin{figure}[ht]
    \centering
    \includegraphics[width=1.0\linewidth]{chapter02/filter_speciation_en.pdf}
    \caption{Séparation typique des différentes surfaces pour préparation et
    analyses conduitent à l'IGE ou dans les laboratoires collaborateurs.}%
    \label{fig:chapter02/filter_speciation_en}
\end{figure}

\subsection{Mesure des composés chimiques}%
\label{sub:analyses_des_composés}

\subsubsection{Préparation des filtres}%
\label{sub:préparation_des_filtres}

Les prélèvements étant réalisé sur des filtres en fibre de quartz
pré-chauffée à \SI{500}{\degreeCelsius} pendant 8 heures pour se prévenir de la présence en espèces
ioniques et organiques. Ces filtres sont chargés dans les porte-filtres des préleveurs
haut-débit, préalablement nettoyés par nos soins, puis envoyé sous emballage fermé.
Le prélèvement se fait grâce à un préleveur haut-débit (\SI{30}{\cubic\m\per\hour}) DA80
digitel.

La fraction prélevée correspond toujours aux \PMdix{}.

\subsubsection{Matière carbonnée (EC et OC) : méthode thermo-optique}%
\label{ssub:matière_carbonnée_ec_et_oc_}

L’analyse de la matière carbonée (carbone organique (OC) et carbone élémentaire (EC)) est
réalisée directement sur un poinçon issu du filtre, à l’aide d’un analyseur
thermo-optique~\autocite[Sunset Lab. Analyser]{birchElemental1996}, par le protocole
EUSAAR-2~\autocite{cavalliStandardised2010,cenAmbient2017a}

Le principe de mesure est basé sur la détection par détecteur FID du CH4 issue de la
combustion puis réduction de la fraction carbonée présente dans l’échantillon. Une
fraction d'échantillon (1 ou \SI{1.5}{\centi\m\squared}) est placée dans un four à quartz
et soumise à différents plateaux de température et sous des atmosphères plus ou moins
oxydantes. Une calibration journalière est effectuée. L’IGE participe annuellement à des
exercices d’intercomparaison dans le cadre du programme européen ACTRIS.

\subsubsection{Espèces ioniques : chromatographie ionique}%
\label{ssub:espèces_ioniques_par_chromatographie}

L'analyse de la fraction ionique des aérosols et des acides organiques légers est réalisée
sur la phase aqueuse par chromatographie
ionique~\autocite{jaffrezoSeasonal2005,cenAmbient2017b} (modèle Dionex ICS 3000) avec une
colonne
CS16 pour l’analyse des cations et colonne AS11 HC pour l’analyse des anions. L’analyse
des anions permet la quantification des ions chlorures \ce{Cl-}, nitrate \ce{NO3-} \&
sulfate \ce{SO4^2-}. L’analyse des cations permet la quantification du sodium \ce{Na+}, de
l'ammonium \ce{NH4+}, du potassium \ce{K+}, du magnésium \ce{Mg^2+} et du calcium
\ce{Ca^2+}. 
Les concentrations en oxalate et en acide méthane sulfonique (MSA) sont aussi accessibles.
La calibration est réalisée tous les jours à partir de solutions standards certifiées.

\subsubsection{Sucres et autres polyols : HPLC-PAD}%
\label{ssub:sucres_et_autres_polyols_hplc-pad}

L’analyse de la fraction soluble des sucres et des polyols est réalisée sur la phase
aqueuse, par une méthode HPLC avec détection par PAD (Pulsed Amperometric Detection)
(modèle Thermo 5000+) avec des colonnes Metrosep (Carb 1 – Guard + A Supp 15 – 150 + Carb
1 – 150)~\autocite{piotQuantification2012,wakedSource2014}.
Cette analyse permet la quantification des saccharides anhydrides (lévoglucosan,
mannosan, galactosan) et des polyols (xylitol, arabitol, sorbitol, mannitol) et sucres
(glucose). La calibration est réalisée tous les jours à partir de solutions standards.
L’IGE a participé dernièrement à une intercomparaison européenne des mesures de
lévoglucosan.

\subsubsection{Cellulose : digestion enzymatique et HPLC-PAD}%
\label{ssub:cellulose_}

The concentration of atmospheric cellulose was quantified using improvements of the
procedure proposed by Kunit and Puxbaum (1996). The principle is to extract the cellulose
from the filter in an aqueous solution, then to process this extract in several solutions
of enzymes in order to break-down the cellulose into glucose units, and finally to
quantify the glucose concentrations using an HPLC-PAD technique. In brief, a 21 mm
diameter punch of each quartz filter sample is extracted for 40 minutes using an
ultrasound bath in 3 ml of an aqueous solution with thymol buffer (pH 4.8). Then two
enzymes solutions (cellulase (Sigma Aldrich, C2730) with 20 µl of an aqueous solution at
70 units g-1) and glucosidase (Sigma Aldrich, 49291), with 60 µl of an aqueous solution at
5 units g-1) are added into the solution. The solution is then incubated at 50 °C for 24
hours for the hydrolysis to occur. The hydrolysis is stopped by placing the solution in an
oven at 100 °C for 45 minutes. The solution is then centrifuged (7000 rpm) for 15 minutes,
and carefully extracted out using a syringe before being analysed with an HPLC-PAD
instrument. The procedural blanks are greatly improved when the enzymes stock solution are
filtered to lower their glucose content. This is performed with a series of cleaning steps
(n=10) by tangential ultrafiltration in a Vivaspin 15R tube at 7000 rpm in Milli-Q water.

The HPLC-PAD (Dionex DX500) is equipped with a Methrom column (250 mm long, 4 mm
diameter), with an isocratic run of 40 minutes with the eluents A (50%, 18mM NaOH), B
(25%, 100 mM NaOH + 150mM NaAc), and C (25%, 220 mM NaOH). Column temperature is
maintained at 30 °C. Eluent flow rate is 1 ml min-1, and injection volume is 250 μl. Each
analytical batch also includes standard glucose solutions as well as standard cellulose
solutions (using 20 µm beads, Sigma Aldrich, S3504) that have been processed like the real
samples in order to determine the specific efficiency of the cellulose-to-glucose
enzymatic conversion for each batch. The final calculation of the atmospheric
concentration of the free cellulose takes this conversion efficiency into account. It
varied according to the batch, generally ranging from 65–80%. The calculation of the
cellulose concentration also takes into account the initial concentrations of atmospheric
glucose of each sample, determined in parallel with the HPLC-PAD analysis of sugars and
polyols as described above. Finally, field and procedural blanks are also taken into
account.
\todo{retravailler cette partie}
\subsubsection{Acides organiques légers : HPLC-MS}%
\label{ssub:acides_organiques_légers_hplc_ms}

L’analyse de la fraction soluble d’une vingtaine d’acides organiques légers (C2 – C7) est
réalisée sur la phase aqueuse, par une méthode HPLC avec détection par Spectrométrie de
Masse (Dionex DX500 + MS LCQ Fleet Thermo) avec séparation sur une colonne Waters
Carotenoid. Cette analyse permet la quantification d’une assez large série d'acides
organiques (depuis l’oxalate jusqu’à l’acide benzoïque). Ces acides sont issus de
processus d’oxydation de matière organique d’origine biogénique ou anthropique. La
calibration est réalisée tous les jours à partir de solutions standards.
\todo{ref}

% Analyses des HULIS par TOC
% L’analyses de la fraction soluble dans l’eau des HULIS (Humic-Like Substances ;
% composés macromoléculaires polyfonctionnels polyacides) est réalisée par une séparation
% avec filtration sur résine DEAE, permettant de ne retenir que les HULIS de
% caractéristiques précises, tout en éliminant les autres espèces solubles de la matière
% carbonée. Ces HULIS sont ensuite élués de cette résine, l’échantillon récupéré étant
% ensuite analysé pour son contenu en carbone total avec un analyseur Shimadzu TOC 5000.
% Les protocoles détaillés ont été mis au point dans le cadre de la thèse de C Baduel (cf
% Hal CNRS).

% Préparation et analyses de traceurs organiques par GC-MS :
% Ces analyses sont réalisées au LCME (Chambéry). Pour réaliser la spéciation
% moléculaire, une fraction de filtre est extraite par extraction solide/liquide
% filtre/solvants organiques (acétone/dichlorométhane 1:1) à 100°C et 100 bar à l’aide
% d’un Accelerated Solvent Extractor (ASE 200 – Dionex). L’extrait est ensuite concentré
% grâce à un évaporateur (TurboVap II – Zimark) puis filtré à 0,1 µm (Anotop 10 –
% Whatman), Cet extrait est ensuite divisé en fractions pour l’analyse de 15 HAP, de
% composés polaires (entre autres de nombreux dérivés de la combustion de la biomasse,
% etc), et de composés apolaires (alcanes C11 à C40 et 10 hopanes), soit environ une
% grosse centaine d’espèces organiques.

% L’analyse des deux dernières séries de composés est réalisée par GC-MS (chromatographie
% gazeuse couplée à un spectromètre de masse ; GC Clarus 500 associée à un MS 560 –
% Perkin Elmer). Pour les composés polaires, avant l’analyse par GC-MS, l’extrait est
% dérivé par ajout du NO-Bis(trimethylsilyl)trifluoroacetamide (BSTFA) + 1% de
% trimethylchlorosilane (TMCS) en proportion 1:1 v/v et sous agitateur chauffant à 50°C
% pendant 2 heures. La quantification est réalisée sur des fragments caractéristiques, la
% calibration étant réalisée à partir de solutions standards et de la méthode à l’étalon
% interne deutéré (n-dodecane d26 pour les composés apolaires et levoglucosan d7 pour les
% composés polaires). La bibliothèque de spectre utilisée est la NIST 2001.

% Analyses des HAP :
% L’analyse des HAP est réalisée par HPLC-fluorescence (chromatographie liquide couplée à
% un détecteur par fluorescence ; HPLC de type Series 200 couplée à un détecteur de type
% Series 200a – Perkin Elmer). La colonne de séparation est une colonne de type phase
% inverse C18 (NUCLEOSIL 100-5 C18 PAH, 25cm × 4,6 cm) éluée avec une phase mobile formée
% d’un mélange méthanol/eau en mode gradient. Cette méthode permet l’analyse des 13 HAP
% classés prioritaires par l’US-EPA (Phénanthrène, Anthracène, Fluoranthène, Pyrène,
% Benzo[a]anthracène, Chrysène, Benzo[e]pyrène, Benzo[b]fluoranthène,
% Benzo[k]fluoranthène, Benzo[a]pyrène, Benzo[ghi]pérylène, Dibenzo[a,h]anthracène,
% Indéno[1,2,3-cd]pyrène) ainsi que du rétène et du coronène. Le LCME a participé aux
% dernères intercomparaisons mises en place par le LCSQA.

\subsubsection{Métaux et traces : IPC-MS ou AES}%
\label{ssub:métaux_et_traces}

Les analyses sont réalisées dans un laboratoire qui
participe aux exercices d’intercomparaison organisés par les Mines de Douai. Ces analyses
sont réalisées par ICP-MS après digestion acide (ou ICP-AES selon les sites
d'études)~\autocite{allemanPM102010,mbengueSizedistributed2014,cenAmbient2005}.
Les métaux et éléments traces analysés sont compris dans la liste suivante : Al, Ag, As,
Ba, Be, Br, Bi, Ca, Cd, Ce, Co, Cr, Cs, Cu, Fe, K, La, Li, Mg, Mn, Mo, Na, Ni, Pb, Pd, Pt,
Rb, Sb, Sc, Se, Sn, Sr, Ti, V, Zn et Zr.

\subsection{Mesure des potentiels oxydants}%
\label{sub:potentiels_oxydants}

Le PO acellulaire des aérosols est déterminé par la capacité intrinsèque des particules
atmosphériques à oxyder le milieu pulmonaire en générant/induisant la formation de ROS.
Les différents filtres seront soumis à deux tests de PO complémentaires, choisis pour
leur représentativité (dénommés PO DTT, et PO AA en fonction du substitut de
l’antioxydant pulmonaire impliqué dans le test, le Dithiotréitol (DTT) et l’acide
ascorbique (AA)). 

Pour chacun des 2 tests présentés ci-après, des triplicats de mesures sont effectués, un
témoin positif à la 1,4-naphtoquinone est utilisé et les blancs de laboratoire sont
soustré aux échantillons.

\subsubsection{Prendre en compte la bioaccessibilité: SLF}%
\label{sub:prendre_en_compte_la_bioaccessibilite_slf}

Étant donné que l'on cherche à reproduire le comportement des PM en milieu biologique
(les poumons), il est important de procéder à l'extraction et l'analyse du PO dans un
fluide reproduissant au plus fidèlement possible ce milieu, et non dans de l'eau mili-Q
directement.

Il existe différents types de fluides mimiquant les surfactants pulmonaires (\textit{simulated
lung fluid} (SLF)). Les analyses de cette thèse provennant du protocole de
\textcite{calasPollution2017}, il est fait usage d'un mélange de solution de Gamble+DPPC.
\textcite{calasImportance2017} a pu montrer que différent SLF produisent des mesures de PO
sensiblement différentes. Notamment, l'ALF tend à produire des mesures de PO
systèmatiquement inférieure à la mesure faite dans du Gamble+DPPC (2 à 3 fois plus
faible), qui elle-même est légèrement inférieure à la mesure faite dans de l'eau mili-Q.
\todo{est-ce que je justifie plus en avant ce choix ?}

\subsubsection{Mesure au dithiothréitol (DTT)}%
\label{ssub:mesure_au_dtt}

L’objectif du test DTT est de suivre la cinétique de consommation du dithiothréitol (DTT)
par des espèces redox-actives portées ou générées en présence de PM. Le dithiothréitol,
réactif principal du test, qui a donné son nom à l’expérience par extension, est un thiol
qui présente un fort pouvoir réducteur (E°=\SI{0.33}{\V}) équivalent à celui d’agents réducteurs
biologiques. Il est considéré comme un substitut aux antioxydants pulmonaires.

Le suivit cinétique se fait par titration à interval régulier (0, 15 et 30 minutes) du DTT
restant (i.e. non oxydé), par réaction avec le DTNB (acide
5-5'-dithiobis-2-nitrobenzoïque). Le produit de réaction, le thionitrobenzoate (TNB) de
couleur jaune, permet la quantification par suivit d'absorbance à \SI{465}{nm}.

Initialement utilisé par~\textcite{choRedox2005} pour mesurer une concentration simulée de
\ce{O2^{.-}} engendrée par la présence de composé organique, il a été ensuite montré par
\textcite{beiReaction2014} que le DTT est potentiellement sensible à une plus grande
variété de ROS qu'imaginé dans le schéma réactionnel originel. C'est maintenant le test le
plus largement utilisé pour mesurer le PO des PM, notamment par la simplification et
la semie automatisation de ce test par \textcite{fangSemiautomated2015}.

Il a été montré également que ce réactif n'est pas sensible qu'aux composés organiques
mais également aux métaux~\autocite{charrierDithiothreitol2012,linGeneration2011}

Ce test est cependant limité du fait que le DTT n'est pas capable de mesurer
\ce{HO^{.}}, qui est pourtant l'un des ROS prépondérants~\autocite{xiongRethinking2017}.

\subsubsection{Mesure à l'acide ascorbique (AA)}%
\label{ssub:mesure_à_l_aa}

Ce test, initié par \textcite{zielinskiModeling1999} pour mesurer la capacité oxydante de
particule fine issue de la combustion de diesel, puis adapté par
\textcite{mudwayVitro2004,ayresEvaluating2008}, utilise l'acide ascorbique (AA) comme
anti-oxydant.
L'acide ascorbique ou vitamine-A est un anti-oxydant naturellement présent au niveau du
surfactant pulmonaire. De la même manière que pour le test au DTT, l'AA est oxydé par
les ROS des PM.

Le suivit cinétique de la consommation de AA se fait pas absorbance à \SI{265}{\nano\m}
toutes les 4 minutes à partir de la deuxième minute, pendant 30 minutes au total.

Le test AA est reconnu pour sa forte réactivité vis-à-vis des métaux de transition. 


% \subsubsection{Mesure par DCFH}%
% \label{ssub:mesure_par_dcfh}
%
% Le test DCFH mesure de façon acellulaire le potentiel oxydant des particules en suivant
% par fluorescence l’oxydation d’un composé par les ROS produits ou véhiculés par les PM en
% présence.
%
% L’oxydation de la DCFH (dichlorofluorescéine) conduit à la DCF (2’,7’dichlorofluorescéine)
% qui est fluorescente. Le dosage de la dichlorofluorescéine est communément utilisé en
% biologie pour visualiser la génération de ROS (reactive oxygen species) mais aussi de RNS
% (reactive nitrogen species) à un niveau intracellulaire et acellulaire. C’est donc un test
% qui va réagir avec plus d’espèces et qui est moins spécifique que les tests DTT ou AA qui
% ne mesurent que les ROS. Il est néanmoins très complémentaire.
%

\subsubsection{Unitées de mesures}%
\label{ssub:unitees_de_mesures}

La mesure du PO par DTT ou AA fournit une consommation de réactant en \si{\nmol\per\min}.
Cependant, cette mesure est nécessairement dépendante de la quantité de PM introduite dans
la réaction.
Il convient donc de normaliser cette consommation d'antioxydant par la masse des réactifs.
Seulement, cette mesure n'est pas fidèle à l'exposition des personnes car ne prend pas en
compte la concentration en particule. De ce fait, deux normalisations de la réactivité des
particules coexistent.

\paragraph{Normalisation par la masse}%
\label{par:normalisation_par_la_masse}
Pour exprimer la réactivité "intrinsèque" d'une particule, la cinétique de déplétion
d'antioxydant ($cAO$, en \si{\nmol\per\min}) est normalisée par la masse de réactif
introduit (en \si{\ug}). On obtient ainsi le PO par \si{\ug}, noté \OPm{} en
\si{\opm}:
\begin{align}
    \label{eq:opm}
    OP_m &= \frac{cAO - cAO_{blank}}{M}
\end{align}
où $cAO$ est la consommation d'antioxydant dans l'échantillon et $cAO_{blank}$
est la consommation d'antioxydant dans les blancs, tous deux en \si{\nmol\per\min}, et $M$
la masse de réactif introduit (PM ou élement chimique), en \si{\ug}.

\paragraph{Normalisation par le volume}%
\label{par:normalisation_par_le_volume}

Seulement, la réactivité intrinsèque n'est pas un indicateur propice à l'exposition. En
effet, il faut également tenir compte de la concentration ambiante des particules. Le PO
est donc aussi exprimé par unité de volume, notée \OPv{} en \si{\opv} et parfois appelé PO
"extrinsèque", calculé par:
\begin{align}
    \label{eq:opv}
    OP_v &= OP_m \times [\text{PM}]
\end{align}
où [PM] correspond à la concentration en particule en \si{\ugm}. Cette expression se
retrouve également écrite~\autocite{fangSemiautomated2015} sous la forme
\begin{align}
    \label{eq:opvalt}
    OP_v &= \frac{cAO - cAO_{blank}}{V}
\end{align}
où $V$ est le volume d'air analysé. Ces deux formes sont cependant équivalentes, car en
replacant $V = \frac{M}{[PM]}$ dans l'Eq.~\ref{eq:opvalt}, on retrouve bien
l'Eq.~\ref{eq:opv}:
\begin{align}
    \label{eq:opvopvalt}
    OP_v &= \frac{cAO -cAO_{blank}}{V} = \frac{cAO -cAO_{blank}}{M}\times [\text{PM}] = OP_m \times [\text{PM}].
\end{align}

\paragraph{Comparabilité des échantillons}%
\label{par:comparabilité_des_échantillons}

Le fait de se ramener à un microgramme de PM en divisant par la masse introduite fait
l'hypothèse d'une linéarité du PO en fonction de la masse. Or, il a été montré par
\textcite{charrierDithiothreitol2012,charrierBias2016,calasComparison2018} que cette
hypothèse est fausse dans le test au DTT et qu'il présente en réalité une réponse
pseudo-logarithmique en fonction de la masse du réactif.

Ainsi, 2 analyses du même filtres ne présenteront pas la même valeur de \PODTTm{} si la
concentration en PM n'est pas identique, les analyses faites à ''fortes masses''
présenteront toujours un \PODTTm{} plus faible que celles à ''faibles masses''.
La comparabilité des échantillons n'est donc possible que si les analyses sont faites à
masses constantes, ou artificiellement corrigées, soit par interpolation, soit par
estimation grâce à la concentration de Cu et Mg, comme le propose
\textcite{charrierBias2016}.

Il faut également noter que ce biais se propage au \PODTTv{} du fait de sa linéarité avec le
\PODTTm. Aussi, un deuxième biais apparaît ici. En effet, la non-linéarité du DTT fait
que la prolongation linéaire entre 1 microgramme et X microgrammes par mètre cube est
fausse. Le \PODTTv{} se comportant alors ''comme si'' les X microgrammes de PM de ce mètre
cube d'air se comportaient comme X fois 1/X 1 microgramme de PM, ce qui est faux de part
la non-linéarité du DTT.

Les analyses de \PODTT{} de cette thèse suivent donc le protocole défini lors de la thèse
de \textcite{calasPollution2017} : analyse à masse constante et faible de PM. Les
échantillons seront donc comparables entre eux. Le parti pris garder le biais
d'extrapolation de la ''faible masse'' à un volume d'air potentiellement charger en PM
pour le \PODTTv{} se justifie biologiquement par les faibles volumes d'air présent dans les
poumons et donc la faible masse de PM en contacte avec les surfactants pulmonaires.

Une telles non-linéarité n'est pas observée pour les tests à l'acide ascorbique ou à la
DCFH.\todo{C'est vrai ca pour la DCFH ? On a checké ?}
Les résultats de cette thèse issus de ces tests sont donc à priori comparables entre les
différentes études.


\section{Signature chimique des facteurs PMF}%
\label{sec:signature_chimique_des_facteurs_PMF}

Chaque source d'émission n'émettant pas les mêmes composés, une analyse chimique
des PM permet l'estimation des contributions des différentes sources aux PM observés.
Seulement, il n'est pas possible de mesurer l'entièreté des plus de 1000 espèces chimiques
présentes dans les aérosols.
Il convient donc de connaître au mieux la signature chimique des différentes sources
d'émissions, notamment via des prélèvements à l'émission en condition
réelle ou en chambre d'étude. La connaissance des espèces spécifiquement émises ou des
ratios d'émissions entre différentes espèces des différentes sources permetront par la
suite l'attribution de la contribution de ces sources aux PM observés.

\subsection{Profile chimique des sources d'émissions courantes}%
\label{sub:profile_chimique_des_sources_d_émissions_courantes}

\begin{description}
    \item[Combustion de biomasse] provenant en Europe majoritairement du chauffage
        domestique. Le type de combustible et les conditions de combustion influence
        grandement sur les espèces chimiques émises. On retrouve cependant toujours de
        l'OC en grande quantité, ainsi que de l'EC. Aussi, si le BC est mesuré, il est
        possible d'estimer la part du BC provenant de cette source de combustion \BCwb{}
        (\textit{BC wood burning}).
        Provennant de la pyrolyse de la cellulose, le lévoglucosan, mannosan et
        galactosan forment 3 proxy spécifique de cette source, car il n'existe pas d'autre
        procédé connu conduisant à leur présence dans l'atmosphère que la combustion de
        bois~\autocite{jordanLevoglucosan2006,puxbaumLevoglucosan2007}.
        Enfin, du potassium \ce{K+} et du rubidium Rb sont également associés à cette
        source~\autocite{navaBiomass2015,gollyEtude2014,chevrierChauffage2016}.

    \item[Traffic routier] émettant différents métaux (Ca, Cu, Mo, Pb, Sb, Sn, Fe, Zr,
        Ti) et de la matière carbonée --notamment de l'EC-- et des composés organiques
        (hopanes, alcanes)~\autocite{schauerCharacterization2006,charronIdentification2019}. Les émissions
        véhiculaires sont parfois séparées en 2 sous-types : émissions à l'échappement
        provenant de la combustion~\autocite{allenSize2001,huMetals2009,vianaSource2008},
        et émissions hors-échappement (abrasion de la route, usure des freins et pneus,
        etc.)~\autocite{grigoratosBrake2015,sandersAirborne2003,sternbeckMetal2002}.

    \item[Émission primaire biogénique] directement émise par les micro-organismes du sol
        et des plantes, qui, de par leur activité biologique, émettent notamment
        différents polyols : arabitol, mannitol,
        sorbitol~\autocite{bauerSignificant2008,yttriAmbient2007,samakePolyols2019}.

    \item[Poussière crustale] remise en suspension par le vent ou les activités humaines
        (mine, carrière, etc). Ce profil chimique reflète celui de la croute continentale
        et présente donc du \ce{Mg^2+}, \ce{Ca^2+}, Ti, Mn et Fe, mais très peu de composés
        organiques~\autocite{almeidaSource2005,dallostoHourly2013,morenoVariations2011,putaudSizesegregated2004}.

    \item[Sel de mer] issue des embruns marins, présentant donc une grande proportion de
        \ce{Na+} et \ce{Cl-}, mais également de
        \ce{Mg^2+}~\autocite{belisCritical2013,odowdMarine1997,pioClimatology2007}. Il
        est à noter que le sel de route, notamment en vallées alpines, présente la même
        composition chimique~\autocite{airrhone-alpesInfluence2012}.

    \item[Industrie] Les profils chimiques des sources industrielles sont très variables
        car dépendent directement du type d'industrie. Cependant, il est courrant d'y
        retrouver des métaux, du \SOq et certains composés organiques spécifiques (HAP et
        HAPs, hopanes...) selon les procédés
        industriels~\autocite{elhaddadPrimary2011,sylvestreComprehensive2017}.

    \item[Port et bateau] proche des côtes notamment. Marqué par la présence de V et Ni,
        provenant de la combustion de fuel lourd, mais également de composés organiques
        comme les hopanes.

    \item[Agriculture] émetant notamment du nitrate et de l'ammoniac à travers les
        fertilisants de synthèse ou de fumier. Cette source d'émission est notamment
        responsable des pics de pollution printanier, lors des épandages agricole à cette
        période.

    \item[Volcan] beaucoup plus ponctuel --surtout en Europe-- mais néanmoins possible
        (par exemple l'Eyjafjöll en 2010), les volcans émettent de grande quantité de
        poussère minérale et d'éléments traces, mais également de soufre et donc de \SOq.

\end{description}

\subsection{``Sources'' ou processus secondaires}%
\label{sub:_sources_secondaires}

Aux sources primaires précédemment listées, il faut ajouter les processus secondaires liés
à la photochimie ou au vieillissement des aérosols, qui conduisent à la formation de
nouvelles espèces chimiques.

\paragraph{Aérosol organique secondaire}%
\label{par:aérosol_organique_secondaire}

Notamment, la biologie est connue pour émettre du dimethyl sulfate (DMS), s'oxydant en
MSA dans l'atmosphère. Sa présence dans les PM trace donc le vieillissement d'un aérosol
d'origine organique.  L'oxalate est également un composé issu de réaction photochimique,
mais la diversité des composés primaires et voie de réaction rend extrêmement difficile
l'identification de la source originelle.
On nomme alors ces aérosols \textit{secondary organic aerosol} (SOA), dont le
dénominateur commun est de présenter une fraction importante de matière organique
oxygéné. Il est compliqué avec des analyses moléculaires d'en déterminer l'origine et des
travaux sont en cours en ce sens et présenté dans le
chapitre~\ref{cha:approfondissement_des_connaissances_des_sources_des_pm}.

Il faut noter que d'autres techniques d'analyses, notamment l'ACSM, permettent une
analyse plus avancée de cette fraction organique de l'aérosol, sans pour autant pouvoir
caractériser moléculairement les éléments retrouvés.

\paragraph{Aérosols inorganique secondaire}%
\label{par:aérosols_inorganique_secondaire}

Mais la partie organique n'est pas la seule à réagire dans l'atmopshère. Ainsi, les NOx,
gazeux, émis notamment par le transport routier, peuvent condensés en phase particulaire
lorsqu'ils s'associent à l'ammoniac, provenant du secteur agricole, pour former du
nitrate d'ammonium \ce{NO3NH4}. Il en est de même pour le \ce{SO2} émis par différentes
sources ou des fonctions sulfatés de la matière organique, et l'ammoniac, condensant en
sulfate d'ammonium \ce{SO4(NH4)2}. Cependant, une part de cet aérosols inorganique tracé
par le sulfate d'ammonium présente une part non-négligeable de OC mais aussi du Sélenium
\ce{Se}, en faisant un facteur mix inorganique et organique. Des travaux sur ce facteur
sont présentés dans le
chapitre~\ref{cha:approfondissement_des_connaissances_des_sources_des_pm}.

\paragraph{Vieillissement et volatilité}%
\label{par:vieillissement_et_volatilité}

Aussi, la partie la plus volatile de la phase particulaire tend à s'évaporer et rejoindre
la phase gaseuse au cours du temps. C'est ainsi qu'après plusieurs jours de transport
dans l'atmosphère, le sel de mer peut avoir ''dégazer'' le chlore présent dans son réseau
cristallin, et présenté des substitutions avec du sulfate ou du nitrate. De même que la
sodium peut avoir été remplacé par d'autres cations comme le potassium ou le magnésium.
Souvent, ce procédé est associé a une augmentation de la matière organique liée à cet
aérosol.


\section{Harmonisation et gestion de base de donnée}%
\label{sec:harmonisation_et_gestion_de_base_de_donnée}

\subsection{Problématisation : des quantités de plus en plus conséquentes}%
\label{sub:problématisation_des_quantités_de_plus_en_plus_conséquentes}

Pour approfondir la compréhension de la chimie des PM, la base de donnée accumulée à l'IGE
est extrèmement précieuse par son ampleur et diversité de lieu et chimie analysée.
Seulement, jusqu'à présent, la généralisation d'observation était rendu compliquée par une
difficulté technique : chaque série d'analyse était enregistré sur des support différents,
sans standardisation et avec des méta-donnée partielles. Cela ne posait pas de problème
tant que la personne en charge de l'étude de ces séries de mesure avait une connaissance
implicite du jeu de donnée et tant que la quantité d'analyse était restreinte.

Cependant, l'accumulation de mesure provennant de différent programme de recherche et la
diversité des personnes ayant à traiter ces résultats ainsi qu'une volontée de synthèse
demande un travail de fond sur la gestion de cette base de donnée afin d'avoir des données
homogènes (même unité, même nomenclature, même format de date…) et intercomparables
(métadonnée complète : fraction des PM analysée, site de prélèvement, date et heure du
début et fin de prélèvement…).

Aussi, le volume de donnée généré, sans être à proprement parler du \textit{big data}
\footnote{Bien qu'il n'y ait pas de définition formelle de \textit{big data}…  Une
    définition que j'apprécie stipule que le \textit{big data} commence lorsque tout
    n'est plus visible sur un tableur. Dans ce cas, nous pouvons catégoriser cette base de
donnée de \textit{big data}.}
, demande un traitement automatique de diverses vérifications (balance ionique, detection
d'erreur de mesure…) et pré-calcul de variables d'intérets (bilan de masse des PM,
fraction crustale…).

La quantité de donnée créer un besoin de visualisation rapide de ces mesures et
d'outils de statistique exploratoire uni- ou multi-variée rapide d'utilisation
(corrélation, dispersion saisonière, comparaison entre site…) tant pour la vérification
des mesures (analyse de points sortant de l'ordinaire…) que pour des recherches
préliminaires (cycle saison de tel espèce chimique sur site cotier ou alpin, évolution
inter-annuelle, ratio lévoglucosan/OC…).

Enfin, il n'existe pas pour l'instant de gestion de sauvegarde ou de suivi de l'édition
de chacun de ces fichiers, rendant ces mesures dépendantes d'une seul support. Ainsi la perte
d'une clef USB, un fichier corrompu, un virus informatique, un incendie… mais également
des erreurs liés à l'utilisateur via un glisser/déposer un peu trop rapide ou une
suppression de cellule pourrait conduire à la perte entière de série annuelle d'analyse.

\subsection{Mise en place et gestion d'une base de donnée pour la filtrothèque}%
\label{sub:mise_en_place_et_gestion_d_une_base_de_donnée_pour_la_filtrothèque}

Ma thèse ayant remobilisée directement dans mes travaux ou dans des travaux associés de
très nombreuses mesures, une des première étape a consisté en la mise en place d'une base
de donnée centralisée et homogène de la filtrothèque, pouvant ainsi être traité de manière
automatisée.

\paragraph{Différents usages et utilisateurs}%
\label{par:différents_usages_et_utilisateurs}

Les contraintes majeures suivantes devaient être respectées :
\begin{enumerate}
    \item facilité d'utilisation pour la diversité des techniciens et techniciennes du
        plateau analytique,
    \item suivit, sauvegardes et historisation des mises à jours,
    \item standardisation, structuration et interopérabilité vers différents logiciel ou
        language de traitement (python, R, QGIS…).
\end{enumerate}
Le point numéro 1 demande donc une interface haut niveau (copier/coller, cliquer et
éditer, commentaire, mise en couleur…) typique d'un tableur, alors le point 3 plaide
davantage pour des formats de base de donnée plus élaboré (SQL, HDF…).
Aussi, différents fichiers doivent pouvoir être mis en relation (lien site \& échantillon, méta-donnée du
filtre \& résultat d'analyse, résultat PMF \& échantillon…), ce qui est très bien géré
par le language SQL. Aussi, les \textit{dump} des bases de donnée SQL peuvent s'historiser
de farçons incrémentale tout en permettant de retrouver très exactement la base de donnée à
n'importe quelle date. 
Pour l'instant, le choix du stockage s'est donc porté sur \textit{sqlite} pour sa
simplicité d'utilisation\footnote{Cependant, le problème majeur de sqlite réside en son
    absence de serveur accessible depuis l'extérieur. Une évolution possible serait un
serveur postgres ou mysql.}.

\paragraph{Historisation et stockage}%
\label{par:historisation_et_stockage}

L'historisation sera quand à elle gérée par un gestionnaire de version. Ici, \textit{git}
est choisi. La sauvegarde se fera en locale sur les serveurs de l'IGE, mais également sur
un dépot git de l'université\footnote{Dépot git :
\url{https://gricad-gitlab.univ-grenoble-alpes.fr/pmall/bdd_aerosols}} (dépôt non public
car certaines données ne sont pas publiques).
Techniquement, l'historisation se fait par un versioning d'un dump de la base de donnée en
SQL --afin de pouvoir reconstuire l'intégralité de la base de donnée-- ainsi qu'en
l'export en CSV de chacune des séries de mesures par site d'étude.

\paragraph{Harmonisation et pré-calcul}%
\label{par:harmonisation_et_pré_calcul}

Plusieurs variables d'intérêts sont souvent demandées, telles que la part crustale, la
somme des certains polyols, l'estimation de la masse des PM reconstruites par bilan de
masse… Aussi, ces variables sont donc automatiquement calculées en fonction des données
disponibles. À titre d'exemple, la masse estimée des PM par bilan de masse est nécessaire
pour la mesure du PO à masse constante de PM. Seulement, les estimations des différentes
parties (crustale, sel marin, etc) peuvent dépendre des espèces mesurées, rendant complexe
leur estimation : 
\begin{align}
    \label{eq:pm_massbalance}
    \PMdix &= [\text{OM}] + [\text{EC}] + [\text{sea salt}] + [\ce{nss-SO_4^{2-}}] + [\ce{NO3-}] + [\ce{NH4+}] + [\text{dust}] + [\text{non dust}]
\end{align}
où EC, \ce{NO3-}, \ce{NH4+} sont directement mesurée, et où les autres fractions sont
estimés selon les relations suivantes :

\paragraph{OM} \autocite{putaudEuropean2010}(Lonati et al., 2005; Putaud et al., 2010; Sillanpää et al., 2006, Favez et al. 2011)
        Ici, on considère une valeur constante pour le ratio OC-OM prennant en compte la
        masse de H, N, O, S non mesurée :
        $[OM] = 1.8 \times [OC]$

\paragraph{nss-SO4} \autocite{seinfieldAtmospheric1998} sulfate non issue du sel marin
$[nss-SO_4^{2-}] = [SO_4^{2-}] - [ss-SO_4^{2-}]$ avec $[ss-SO4] = 0.252\times[Na^+]$ donc 
\begin{align*}
    [\ce{nss-SO_4^{2-}}] &= [\ce{SO_4^{2-}}] - 0.252\times[\ce{Na+}]
\end{align*}

\paragraph{sea salt} Deux équations sont utilisés, selon la présence ou non de chlore, et
en utilisant les ratio spécifique de l'eau de mer suivant :
\begin{align*}
    [\ce{ss-K+}]        &= 0.036\times[\ce{Na+}]\\
    [\ce{ss-Mg^{2+}}]   &= 0.129\times[\ce{Na+}]\\
    [\ce{ss-Ca^{2+}}]   &= 0.038\times[\ce{Na+}]\\
    [\ce{ss-Cl^-}]      &= 1.8\times[\ce{Na+}]\\
    [\ce{ss-SO_4^{2-}}] &= 0.252\times[\ce{Na+}]
\end{align*}

\subparagraph{Sans chlore} \autocite{putaudEuropean2010} (Grythe et al., 2014; Putaud et al., 2010):
\begin{align*}
    [\text{sea salt}] = [\ce{Na+}]\times(1 + [\ce{ss-K+}] + [\ce{ss-Mg^{2+}}] + [\ce{ss-Ca^{2+}}] + [\ce{ss-Cl-}] + [\ce{ss-SO_4^{2-}}])
\end{align*}
donc $[\text{sea salt}] = 3.252\times[\ce{Na+}]$.

\subparagraph{Avec chlore} \autocite{putaudEuropean2010}
\begin{align*}
    [sea~salt] = [\ce{Cl-}] + [\ce{Na+}]\times(1 + [\ce{ss-K+}] + [\ce{ss-Mg^{2+}}] + [\ce{ss-Ca^{2+}}] + [\ce{ss-SO_4^{2-}}])
\end{align*}
donc $[\text{sea salt}] = [\ce{Cl-}] + 1.47\times[\ce{Na+}]$.

\paragraph{Dust}
\subparagraph{Depuis le calcium soluble} \autocite{putaudEuropean2004}
$[dust] = 5.6 \times [nss-Ca^{2+}]$, avec $[nss-Ca^{2+}]$ estimé par $[Na^+]/[ss-Ca^{2+}]$
de l'eau de mer de 26, donc : $[nss-Ca^{2+}] = [Ca^{2+}] - [Na^+]/26$. Donc :
\begin{align*}
    [dust] &= 5.6 \times ([Ca^{2+}] - [Na^+]/26)
\end{align*}

\subparagraph{Depuis les métaux}
\begin{itemize}
    \item \autocite{malmSpatial1994}
      \begin{align*}
          [\text{dust}] &= 1.16 \times (1.90[Al] + 2.15[Si] + 1.41[Ca] + 2.09[Fe] + 1.67[Ti])
      \end{align*}
  \item \autocite{querolSource2002,perezCoarse2008}
      $[dust] = Al_2O_3 + SiO_2 + CO_3^{2-} + Ca + Fe + K + Mg + Mn + Ti + P$ avec
      $SiO_2$ et $CO_3^{2-}$ déterminé indirectement par les relations
      $[SiO_2] = 3 \times [Al_2O_3]$ ; $[Al_2O_3] = 1.89 \times [Al]$ et 
      $[CO_3^{2-}] = 1.5 \times [Ca]$, ainsi
      \begin{align*}
          [\text{dust}] &= 7.56 \times [Al] + 2.5 \times [Ca] + Fe + K + Mg + Mn + Ti + P
      \end{align*}
\end{itemize}

\paragraph{non dust} \autocite{salamehPM22015}
Les élements non-dust correspondent à la somme des métaux non crustaux communément mesurés :
\begin{align*}
    [\text{non dust}] &= [Cu] + [Ni] + [Pb] + [V] + [Zn]
\end{align*}


\begin{landscape}
\begin{figure}[ht]
    \centering
    \includegraphics[width=\linewidth]{chapter02/workflow_simple.pdf}
    \caption{Vie d'un échantillon et des données : reception, analyse et enregistrement
        dans la filtrothèque et visualisation, puis utilisation dans les différents
        programmes de recherche.}%
    \label{fig:bdd}
\end{figure}
\end{landscape}

