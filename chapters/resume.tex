\addcontentsline{toc}{chapter}{Résumé}
\begin{center}
    \Huge\textsc{Résumé}
\end{center}

\paragraph{Francais}%
\label{par:francais}

La qualité de l'air que nous respirons est responsable de nombreuses affections de santé allant de
l'asthme aux pathologies cardio-vasculaires ou cancers. À l'échelle mondiale, la mauvaise
qualité de l'air représente la
4\ieme{} cause de mortalité, principalement due à la
présence de particules fines (aérosols ou PM pour particulate matter) de propriétés physico-chimiques très
variées et
provenant de sources naturelles (forêts, poussières crustales, spray marin, etc.) ou
anthropiques (trafic routier, chauffage résidentiel, cuisine, industrie, agriculture,
etc.).

La diversité considérable de ces PM rend l'évaluation de leur toxicité complexe. Dans
cette thèse, 
une mesure intégratrice des différentes propriétés physico-chimiques des
aérosols (taille, forme, solubilité, composition chimique, etc.) est utilisée à travers la
métrique du potentiel oxydant (PO) des particules.
Cette mesure indirecte des espèces réactives de l'oxygène à travers les tests à l'acide
ascorbique (AA) et au dithiothreitol (DTT) permet une vision de l'aérosol plus proche des
impacts sanitaires.

Afin d'estimer les sources d'émissions responsables du PO, des études approfondies de la
géochimie des sources et de leur quantification à travers des mesures longue durée de
terrain et l'utilisation du modèle \textit{Positive Matrix Factorization} (PMF) sont tout
d'abord présentées. Les incertitudes des profils chimiques et des contributions
des sources de PM sont également quantifiées et la variabilité géochimique entre sites
estimée par la méthode deltaTool proposée par le groupe FAIRMODE. Une synthèse à grande
échelle spatiale portant sur quinze sites de prélèvements grâce à une étude harmonisée
montre la présence de 8 facteurs PMF présents sur l'ensemble du territoire : combustion de
biomasse, trafic routier, émissions biologiques primaires, poussières crustales, sel marin
agé, \textit{MSA-rich}, \textit{nitrate-rich} et \textit{sulfate-rich}. D'autres facteurs,
propres à la spécificité de chacun des sites, sont également bien déterminés.

La contribution de ces sources au potentiel oxydant est ensuite étudiée grâce à la mesure
conjointe du PO sur les filtres ayants permis les études PMF. Un modèle
d'inversion simple (régression linéaire multiple) entre les sources de PM et le PO permet
une estimation statistiquement satisfaisante de la contribution des sources au PO. La
pertinence géochimique de ce modèle est évaluée à travers son application sur quinze séries annuelles (
a minima) de mesure. Les sources présentant un PO intrinsèque les plus élevés sont le trafic routier, la combustion de biomasse, les
poussières crustales et dans une moindre mesure les émissions primaires biogéniques pour
le test au DTT et la combustion de biomasse et le trafic routier pour le test à l'AA.
Ainsi, certaines sources contribuant de façon importante à la masse des PM ne contribuent
pas ou peu à leur PO (notamment le \textit{nitrate-rich}) et l'on observe une redistribution de
l'importance des sources de PM selon que la métrique d'observation considérée est la concentration
massique ou le potentiel oxydant.

Les mesures sur des séries de prélèvements annuels et la grande base de données
utilisée regroupant plus de 1700 filtres permet également de mettre en évidence
l'importance de l'exposition chronique au potentiel oxydant. L'importance de la
combustion de biomasse en hiver, notamment en vallées alpines, fait de cette source la
source principale du PO en moyenne annuelle. Cependant, la présence à de plus faibles
concentrations mais relativement constantes au cours de l'année fait des émissions
primaire du trafic routier la contributrice majeure à l'exposition chronique.

Finalement, de par l'estimation robuste d'un potentiel oxydant intrinsèque par typologie
de source d'émission, cette thèse ouvre la voie à la prévision du PO par les modèles
déterministes et à l'amélioration de la prise en compte du potentiel oxydant dans les
études épidémiologiques, étapes importantes afin de permettre, à terme, l'utilisation de cette métrique pour
la réglementation de la qualité de l'air.

\paragraph{Anglais}%
\label{par:anglais}
\selectlanguage{english}

The quality of the air we breathe is responsible for many diseases ranging from asthma to
cardiovascular disease and cancer. Worldwide, poor air quality is the 4th leading cause
of death, mainly due to the presence of fine particles (aerosols or
PM) of very varied chemistry and physical properties, coming from natural sources
(forests, crustal dust, marine spray, etc.) or anthropogenic sources (road traffic,
residential heating, cooking, industry, agriculture, etc.).

The considerable diversity of these PMs makes the assessment of their toxicity complex.
In this thesis, an integrative measure of the different physico-chemical properties of
aerosols (size, shape, chemical composition, etc.) is used through the metric of the
oxidizing potential (OP) of the particles. This indirect measurement of reactive oxygen
species through the ascorbic acid (AA) and dithiothreitol (DTT) tests allows a vision of
the aerosol closer to health impacts.

In order to estimate the emission sources responsible for the OP, in-depth studies on the
geochemistry of the sources and on their quantification through long field measurements
and the use of the \textit{Positive Matrix Factorization} (PMF) model are first
presented.  Uncertainties in the chemical profiles and contributions of PM sources are
also quantified and the geochemical variability between sites estimated by the deltaTool
method proposed by the FAIRMODE group.  A large spatial scale synthesis of 15 sampling
sites through a harmonized study shows the presence of 8 PMF factors present over the
whole territory: biomass burning, road traffic, primary biological emissions,
crustal dust, aged sea salt, \textit{MSA-rich}, \textit{nitrate-rich} and
\textit{sulfate-rich}. Other factors, specific to the peculiarity of each site, are also
well determined.

The contribution of these sources to the oxidizing potential is then investigated using
the joint measurement of PO on the filters from the previous PMF studies. A simple
inversion model (multiple linear regression) between the PM sources and the OP provides a
statistically satisfactory estimate of the contribution of the sources to the OP. The
geochemical relevance of this model is evaluated through its application on fifteen
measurement series. The sources with a major intrinsic OP (i.e. per
microgram of PM) are road traffic, biomass burning, crustal dust and to a lesser
extent primary biogenic emissions for the DTT test and biomass burning and road
traffic for the AA test. Thus, some sources that contribute significantly to the mass of
PM contribute little to their PO (namely \textit{nitrate-rich}) and there is a
redistribution of the importance of PM sources according to whether the metric of
observation is the mass concentration or their oxidizing potential.

Measurements on annual sampling series and the large database of more than 1700 filters
also highlight the importance of chronic exposure to oxidizing potential. The importance
of biomass burning in winter, especially in alpine valleys, makes this source the main
source of OP on annual average. However, the presence at lower but relatively constant
concentrations during the year makes the road traffic source the major contributor to
chronic exposure.

Finally, due to the robust estimation of an intrinsic oxidizing
potential by emission source typology, this thesis opens the way
to the prediction of OP by deterministic models and to the 
improvement of the consideration of the oxidizing potential in 
epidemiological studies, important steps in order to eventually 
allow the use of this metric for air quality regulation.


\selectlanguage{french}
