
En plus de mon financement direct par l'ENS Ulm, les différents programmes de recherche
suivants ont permis la collecte et l'analyse des filtres utilisés pour cette thèse :

\begin{enumerate}
    \item Programme ADEME CAMERA : Caractérisation des aérosols pour le réseau MERA (2011-2013). (Révin, Peyrusse-Vielle, Verneuil, Dieulefit)
    \item Programme ADEME CORTEA (2012). INACS-1 et -2 (Isotopie du nitrate d’ammonium : Compréhension des sources) 2012-2015. 
    \item Convention cadre avec l’INERIS pour diverses études en collaboration avec le LCSQA dans le cadre de CARA. 2012 -2021. 
        \begin{itemize}
            \item Etude de la composition chimique détaillée des PM à Lens, Lyon, Bordeaux et Grenoble pour étude PMF (2012-2014) 
            \item Suivi des traceurs de combustion de la biomasse à Grenoble (2007 – en cours) 
        \end{itemize}
    \item Programme Primequal « Contribution à l’évaluation de l’opération pilote visant à
        réduire les émissions de particules fines du chauffage au bois individuel dans la
        zone du PPA de la vallée de l'Arve » : Détermination de la contribution de la
        combustion de la biomasse aux PM10 sur différents sites de la Vallée de l’Arve :
        mise en place et qualification d’un dispositif de suivi, et premières évaluations
        de l’évolution des apports (DECOMBIO) (2013 -- 2017) et DECOMBIO 2 (2018-2020). 
    \item Collaboration avec l’ANDRA (2012 -- en cours) Caractérisation chimiques des PM et de leurs sources sur le site de l’OPE. 
    \item Programme LEFE CHAT – EC2CO (2016). CAREMBIOS (Caractérisation des émissions biogéniques des sols). 
    \item CORTEA ADEME (2011-2013). PM-DRIVE : Emissions particulaires Directes et Indirectes du trafic routier 
    \item Collaboration avec Atmo Sud (2014--2018). Chimie des PM en Région PACA : études à Gardanne, Nice, Marseille, et Port de Bouc. 
    \item ADEME (2015-2017). SOURCES : synthèse sur les Sources des PM en France. 
    \item ADEME (2016-2019). QAMECS (Politiques publiques, qualité de l’air, impacts sanitaires et économiques, société). 
    \item IDEX – COMUE Grenoble. Cross Disciplinary Project (CDP). MobilAir (2018-2021).
    \item Collaboration avec l’EMPA (Suisse) (2018--2019) Caractérisation des PM en
        Suisse: étude de traceurs chimique et du PO des PM des 12 séries annuelles sur 6
        sites Suisses.
    \item LEFE-CHAT (2015) LEFE-PO. Le potentiel oxydant: une caractéristique chimique des PM atmosphériques utilisable comme proxy de l’impact sanitaire. 
    \item LEFE – CHAT (2017) MECEA : développement de la mesure de la cellulose atmosphérique.
    \item IDEX UGA Data Institute (2019). Financement pour un séjour de S Weber au TNO (Hollande) + Financement stage M2
    \item LEFE – CHAT (2019) MHYRIAM. Développement pour la mesure du PO \ce{OH^.}
    \item ANR JCJC (2019) Get OP – Stand OP (coord. G Uzu). 
    \item Collaboration avec le PSI (Suisse) pour des séries de prélèvements en Suisse et en Inde (2016-2019)
\end{enumerate}

