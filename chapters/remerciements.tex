\addcontentsline{toc}{chapter}{Remerciements}
\begin{center}
    \Huge\textsc{Remerciements}
\end{center}

Une thèse est un point d'étape dans une formation et un travail de recherche.
Sans les enseignants et enseignantes m'ayant aiguillé au fur et à mesure de mon parcours
vers cette voie, assurément, je n'aurai pas écrit cette page. Merci beaucoup à vous
toutes et tous.

Notamment, pour avoir entretenu mon enthousiasme pour les sciences, fait découvrir les ENS
et m'avoir permis d'y entrer, mais aussi d'avoir fait que je garde un pied dans la
biologie et fait sortir de mon laboratoire grâce aux sup2 et aux sorties géol' pendant
ma thèse, merci Alexandra !

Je remercie aussi grandement Emmanuel, Théo et Didier pour m'avoir fait enseigner
dans vos UE mais surtout pour la grande liberté que vous m'avez laissé dans vos cours la
confiance que vous m'y avez accordée.

L'ensemble de ma thèse n'aurait pas eu lieu non plus sans l'ENS, qui en plus de m'avoir
financé pendant ces trois ans de thèse, a considérablement ouvert mon esprit critique
grâce à son approche interdisciplinaire extrêmement riche, notamment en sciences
environnementales.

Merci également à toutes et tous les techniciens et techniciennes, chercheurs et
chercheuses, ingénieurs et ingénieures, stagiaires et personnels de laboratoire, pour
tous les prélèvements ou analyses et de
manière générale, le travail que vous effectuez sans lequel cette thèse n'existerait
simplement pas, et notamment les Fannys, Vincent, Coralie, Benjamin, Stéphan, Jean-Luc,
Auriane, Lisa, Anthony, Armelle, Céline, Rhabira, Claire, Kévin, Jean-baptiste et tous
les personnels des ASQAA dont je ne connais malheureusement pas le nom.

Bravo et merci aussi aux docteur·e·s chiantiesque ou assimilé Florie, Aude, Julie et
Abdoulaye, Valéria et Foteini. Et hop, passage de relais à Anouk !
Merci aussi à toute l'équipe chianti pour la bonne ambiance, la disponibilité et
les discussions en tout genre (mais parfois scientifique quand même) !
Notamment, merci à ma mentor PMF Dalia et à ma co-bureau du PO Lucille.

Pour m'avoir écouté pendant plusieurs heures parler du potentiel oxydant et de sources
de PM et leurs regards critiques et constructifs pendant ces trois années, merci
également à Aurélien et Rémy.

De toute évidence, merci aussi à Gaëlle pour sa foultitude d'idée et son énergie
débordante, et à Jean-Luc pour son dessin intelligent et à être en avant de la science et
pour la qualité de votre encadrement.
Ça a été un grand plaisir de faire cette thèse avec vous. Merci vraiment pour la
confiance que vous m'avez accordée.

Forcément, merci à mes parents, Alain et Isabelle, pour m'avoir poussé à être curieux, à
chercher à comprendre, à aller toujours un peu plus loin à chaque fois tout en me
laissant la liberté de mes choix et m'avoir tout le temps soutenu quels qu'ils soient.

Finalement, merci à toi Sarah, pour partager nos vies et m'accompagner depuis plus de
huit ans maintenant et pour je l'espère encore longtemps, le plus longtemps possible !



