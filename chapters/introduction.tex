Le rôle d'un travail scientifique est d'observer, d'interroger, de comprendre et d'expliquer. 
Dans ce contexte, la science permet de questionner l'état d'un système, de l'étudier et
éventuellement d'alerter sur ses probables évolutions.

Dans le domaine des sciences du climat ou des géosciences dans leur ensemble, la situation
de changement du climat terrestre a ainsi été observée, comprise et expliquée
par la communauté scientifique, de même que d'autres ``crises'' ou changements brutaux
actuels concernant le système Terre (trou de la couche d'ozone, diminution de la
biodiversité, qualité des eaux et des sols, etc).
Les débats scientifiques ne portent actuellement plus que sur l'affinement des théories et
modèles prédictifs, mais les généralités sont maintenant bien établies et acceptées. C'est
à la société dans son ensemble de trouver une réponse aux dangers auxquels nous faisons
face.

En revanche, l'impact de la qualité de l'air sur les écosystèmes et en particulier sur les
populations humaines est encore mal quantifié. Les observations disponibles sont
parcellaires et la compréhension physico-chimique des processus d'émissions et de
transformations dans l'atmosphère reste à étudier. Aussi, l'impact sur la santé
humaine de la pollution de l'air demande un travail interdisciplinaire important croisant
épidémiologie, toxicologie et géosciences.

La question de l'outil de l'observation de la qualité de l'air est un sujet complexe. La
composition de l'air que nous respirons est extrêmement vaste. Chimiquement, plusieurs
milliers de molécules gazeuses différentes pénètrent dans nos poumons à chaque inspiration.
Physiquement, des particules de tailles variant du nanomètre au centième de millimètre, de
formes et chimies très différentes sont également inspirées et expirées toutes les
secondes par notre organisme. Biologiquement, des pollens, bactéries, spores ou virus
évoluent ou vivent dans l'air que nous respirons.

Ainsi, plusieurs métriques d'observation et de quantification de l'impact sanitaire ont pu
être proposées : distribution en taille des particules, espèces chimiques présentes,
concentration massique.
Mais chacune de ces observations ne regarde qu'un aspect de la pollution. Il est donc
nécessaire de trouver une mesure intégratrice, permettant la prise en compte de la
diversité chimique, physique et biologique de l'air, tout en prenant en compte l'impact
sanitaire potentiel sur le système biologique humain.

L'un des mécanismes suspectés des maladies générées ou accentuées par la pollution de l'air
est imputable à la mise en place d'un état de stress oxydatif dans notre corps, à l'origine des
dysfonctionnements conduisant aux diverses pathologies observées (asthme, maladie
cardiovasculaire, etc). Ainsi, une mesure intégratrice prometteuse concerne la capacité de
l'air inspiré à déséquilibrer nos défenses anti-oxydantes, notamment pulmonaires.
La mesure de cette capacité oxydante de l'air, appelée potentiel oxydant, pourrait être
l'une de ces métriques recherchées.

Cette thèse s'inscrit dans cette démarche de recherches des déterminants de ce potentiel
oxydant des particules atmosphériques. Seulement, pour estimer les sources de potentiel
oxydant, il est nécessaire de se poser en amont la question des sources de particules
atmosphériques. L'utilisation de nombreuses mesures de terrain, collectées et analysées
lors de différents programmes de recherches antérieurs ou actuels, permettra dans un
premier temps la compréhension de différents processus d'émissions et le renforcement de
méthodologies de quantification des sources d'émissions. L'évaluation à grande échelle
spatiale de la contribution des différentes sources d'émission ainsi que leur apport au
potentiel oxydant sera traité dans un second temps. Finalement, quelques pistes de
travaux futurs seront explorés, toujours avec pour objectif principal la mise en place
d'un meilleur indicateur de la qualité de l'air d'intérêt sanitaire.


