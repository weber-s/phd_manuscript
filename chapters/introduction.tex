The Air Quality has become one of the most environmental issue in the past decades. It is
now well-known that the air quality cause dramatic health damage. The CAFE program (Clean
Air For Europe) estimates that the $\PM_{2.5}$ are responsible for a decrease of a
lifetime of 9 months in average in Europe. Another study
from~\textcite{amannBaseline2005}
indicates that around 42 000 premature death in France are due to the PM concentration in
the air.  This pollution has a health impact both during a pollution peak (asthma,
vulnerability to disease, etc) and with a chronicle exposition (cancer, pulmonary
infection, cardiovasculary disease, etc).

Nowadays, the regulation of the air quality aims to reduce the population exposition to
the air pollutant. For instance in France since the LAURE law (Loi sur l'Air et
l'Utilisation Rationelle de l'Énergie) in 1996, frequent measurements of the air quality
is mandatory.  To do so, the regulation focuses its guidelines on the total mass of the PM
present in the air.  In 2000 the ecology french minister setups several \emph{Association
Agrées de Surveillance de la Quality de l'Air} (AASQA) which monitor, among other things,
the $\PM_{10}$ and $\PM_{2.5}$ mass in \si{\um\per\cubic\m} in the air.  The current
french regulation imposes a daily maximum concentration of \SI{50}{\ug\per\cubic\m} for
$\PM_{10}$ and \SI{25}{\ug\per\cubic\m} for $\PM_{2.5}$. However, this threshold, and so
the population exposition, is regularly exceeded in some urban cities like Paris, Lyon or
Grenoble, or even in some other place like the Arve Valley in the Alps.
Moreover, the PM term regroup only chaptericle which share a size lower than few micrometers,
but they are highly different in term of chemical composition, shape or physical
properties. It results that some PM may be more biologically active than others.  The
metric of the mass of the PM is then not appropriate as \SI{10}{\um\per\cubic\m} of sugar
will not have the same effect as \SI{10}{\um\per\cubic\m} of mercury for instance.  The
fundamental question for human health is then: do all aerosols have the same health
impact? How can we measure it? 

In the past few years emerged the idea of a new metric: the Oxidant Potential (OP) of the
{PM}. The OP is indeed linked to all the physico-chemical characteristic of the PM: the
shape, the size and the chemical composition.  Moreover, the OP could be linked to a
health impact more easily than the mass.  Indeed, in the human lung anti-oxidant are
present to counteract the oxidative potential of the air (radicals, gases and also
aerosols).  In fact, the OP is a measure of the Reactive Oxygen Species (ROS) carried or
induced by aerosols. When there is an unbalance between the lung anti-oxidant defense and
the ROS, an inflammatory response is observed
\autocite{donaldsonOxidative2003,delfinoPotential2005,liAdjuvant2009}.  The OP of the
PM is then considered as a promising proxy for Air Quality purpose.  Recently,
\textcite{fangOxidative2016} coupled measurements of OP and large population
epidemiological analyses and shows a strong correlation between entry in emergency
dechapterment and the OP of the {PM}.
However, such studies are quite recent and present the necessity of further investigations
and rise new questions: how do we measure the {OP}? Could we do it standardly?  Is it
possible to easily measure it?  Will it be useful to understand the dynamic of the sources
and their respective environmental impact?


My internship will be focused on the comprehension of the link between the OP and the
chemistry of the PM and especially to an estimation of the OP contribution by sources of
{PM} in the frame of the ExPOSURE program funded by the ANSES\footnote{ANSES: Agence
Nationale de Sécurité Sanitaire de l'alimentation, de l'environnement et du travail}.
Such study is one of the first in the literature and undoubtedly the one associated with
the biggest sample in the world.  I will first explain some generalities of the
atmospheric sciences and aerosols (chapter~\ref{chapter:context}). Secondly, I will
explain the methodology of the study and the sample acquisition
(chapter~\ref{chapter:measure}).  The third chapter is focused on the results obtain
during this internship and is split in two main sections: the mathematical development of
an inversion framework for the attribution of the OP to the source
(section~\ref{sec:inversionMethod} chapter~\ref{chapter:results}) and the results obtained
with this new method (the rest of the chapter~\ref{chapter:results}).

% To do so, I will follow the workflow given in
% fig.~\ref{fig:workflow_intership}.  The internship is focused on the setup of
% the inversion procedure to attribute an intrinsic OP of the input variable
% (chemical species or emission sources), this is the \emph{inversion} box in the
% figure.  The goal is to attribute an intrinsic OP per \si{\ug} of species or per
% \si{\ug} of emission from a source.  
%
% The speciation of the aerosol chemistry and the OP measurement are discussed in
% the section~\ref{sec:analysis} in chapter~\ref{chapter:measure}.  The PMF model is
% briefly explain in the section~\ref{sec:PMF} in chapter~\ref{chapter:measure} and was
% run by several persons (Dalia Salameh for the site of the SOURCES program,
%     Alexandre Sylvestre for the sites of Nice, Florie Chevrier for the sites of
% the DECOMBIO program). The methodology of the inversion is detailed in the
% section~\ref{sec:inversionMethod} in chapter~\ref{chapter:results}.
%
% \printbibliography[heading=subbibliography]

\addcontentsline{toc}{section}{Bibliography}
\printbibliography[segment=1,heading=subbibliography]
