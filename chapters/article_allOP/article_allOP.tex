% Options for packages loaded elsewhere
\PassOptionsToPackage{unicode}{hyperref}
\PassOptionsToPackage{hyphens}{url}
%
\documentclass[
]{article}
\usepackage{lmodern}
\usepackage{amssymb,amsmath}
\usepackage{ifxetex,ifluatex}
\ifnum 0\ifxetex 1\fi\ifluatex 1\fi=0 % if pdftex
  \usepackage[T1]{fontenc}
  \usepackage[utf8]{inputenc}
  \usepackage{textcomp} % provide euro and other symbols
\else % if luatex or xetex
  \usepackage{unicode-math}
  \defaultfontfeatures{Scale=MatchLowercase}
  \defaultfontfeatures[\rmfamily]{Ligatures=TeX,Scale=1}
\fi
% Use upquote if available, for straight quotes in verbatim environments
\IfFileExists{upquote.sty}{\usepackage{upquote}}{}
\IfFileExists{microtype.sty}{% use microtype if available
  \usepackage[]{microtype}
  \UseMicrotypeSet[protrusion]{basicmath} % disable protrusion for tt fonts
}{}
\makeatletter
\@ifundefined{KOMAClassName}{% if non-KOMA class
  \IfFileExists{parskip.sty}{%
    \usepackage{parskip}
  }{% else
    \setlength{\parindent}{0pt}
    \setlength{\parskip}{6pt plus 2pt minus 1pt}}
}{% if KOMA class
  \KOMAoptions{parskip=half}}
\makeatother
\usepackage{xcolor}
\IfFileExists{xurl.sty}{\usepackage{xurl}}{} % add URL line breaks if available
\IfFileExists{bookmark.sty}{\usepackage{bookmark}}{\usepackage{hyperref}}
\hypersetup{
  pdftitle={ Source apportionment of the oxidative potential of aerosols at 15 French sites for yearly time series of observations },
  hidelinks,
  pdfcreator={LaTeX via pandoc}}
\urlstyle{same} % disable monospaced font for URLs
\usepackage{longtable,booktabs}
% Correct order of tables after \paragraph or \subparagraph
\usepackage{etoolbox}
\makeatletter
\patchcmd\longtable{\par}{\if@noskipsec\mbox{}\fi\par}{}{}
\makeatother
% Allow footnotes in longtable head/foot
\IfFileExists{footnotehyper.sty}{\usepackage{footnotehyper}}{\usepackage{footnote}}
\makesavenoteenv{longtable}
\usepackage{graphicx}
\makeatletter
\def\maxwidth{\ifdim\Gin@nat@width>\linewidth\linewidth\else\Gin@nat@width\fi}
\def\maxheight{\ifdim\Gin@nat@height>\textheight\textheight\else\Gin@nat@height\fi}
\makeatother
% Scale images if necessary, so that they will not overflow the page
% margins by default, and it is still possible to overwrite the defaults
% using explicit options in \includegraphics[width, height, ...]{}
\setkeys{Gin}{width=\maxwidth,height=\maxheight,keepaspectratio}
% Set default figure placement to htbp
\makeatletter
\def\fps@figure{htbp}
\makeatother
\setlength{\emergencystretch}{3em} % prevent overfull lines
\providecommand{\tightlist}{%
  \setlength{\itemsep}{0pt}\setlength{\parskip}{0pt}}
\setcounter{secnumdepth}{-\maxdimen} % remove section numbering

\title{ Source apportionment of the oxidative potential of aerosols at
15 French sites for yearly time series of observations}
\author{}
\date{}

\begin{document}
\maketitle
\begin{abstract}
\textbf{Abstract}

The reactive oxygen species (ROS) carried or induced by particulate
matter (PM) are suspected to induce oxidative stress in vivo, leading to
health impacts in Human populations. The oxidative potential (OP) of PM,
displaying the ability of PM to oxidize the lung environment is gaining
a strong interest to examine health risks associated to PM exposure. In
this study, OP was measured by two different acellular assays
(dithiothreitol, DTT and ascorbic acid, AA) on samples from yearly time
series of filters collected in 15 different sites in France between 2013
and 2016, including urban, traffic and valley typologies. The same
filters were characterized with an advanced chemical speciation allowing
source-apportionment of PM\textsubscript{10} using positive matrix
factorization PMF method for each series, for a total above 1700
samples. This study provides therefore a nation-wide synthesis on the
source-apportionment of OP using coupled PMF and multiple linear
regression model. The road traffic, biomass burning, dust, MSA-rich, and
primary biogenic sources have distinct positive redox-activity towards
the OP\textsuperscript{DTT} assay. The OP\textsuperscript{AA} assay only
presents significant activity for the biomass burning and road traffic
sources. The daily median source contribution to the total
OP\textsubscript{DTT} highlights the dominant influence of the road
traffic source. Both the biomass burning and the road traffic sources
contribute evenly to the observed OP\textsuperscript{AA}. Would the OP
being a good proxy of Human health impact, it appears that domestic
biomass burning and road traffic are the two main sources to target in
order to decrease significantly the OP over the French territory and to
lower the health risks from PM exposure.

\textbf{Introduction}
\end{abstract}

Air quality has become a major public health issue, being the fourth
global cause of mortality with 7 million premature deaths worldwide per
year due to both indoor and outdoor exposure (World Health Organization,
2016). Driving 90 \% of this health impact (Lelieveld et al., 2015),
particulate matter (PM) is one of the key pollutants in the air linked
to health outcomes, although the exact mechanism leading to toxicity is
not yet fully understood (Barraza-Villarreal et al., 2008; Beck-Speier
et al., 2012; Brauer et al., 2012; Goix et al., 2014; Goldberg, 2011;
Saleh et al., 2019). Many urbanized areas, mainly located in low- or
middle-incomes countries, are exposed to PM concentration far higher
than the recommendation guideline of the WHO.

Although PM are now monitored in many countries and large efforts are
observed to document ambient concentrations, the underlying processes
leading to the observed concentrations in the atmosphere, and
particularly the understanding of emissions sources, are still an active
field of research (Diémoz et al., 2019; El Haddad et al., 2011; Golly et
al., 2019; Hodshire et al., 2019; Jaffrezo et al., 2005; Jiang et al.,
2019; Marconi et al., 2014; Moreno et al., 2010; Piot et al., 2012;
Salameh et al., 2015; Samaké et al., 2019a; Waked et al., 2014). In
recent years, strong focus has been put worldwide on
source-apportionment methods in order to better understand the processes
leading to the airborne concentrations and the accumulation of PM in the
atmosphere. This includes direct modeling approaches such as Chemistry
Transport Model using tagged species~(Brandt et al., 2013; Kranenburg et
al., 2013; Mircea et al., 2020; Wagstrom et al., 2008; Wang et al.,
2009) or field studies coupled with receptor models (RM)~(Belis et al.,
2020; Pernigotti et al., 2016; Simon et al., 2010), notably Positive
Matrix Factorization (PMF). PMF can be based either on AMS time resolve
spectrum~(Bozzetti et al., 2017; Petit et al., 2014, 2015) or on filter
analysis (Amato et al., 2016; Bressi et al., 2014; Fang et al., 2015;
Jain et al., 2018; Liu et al., 2016; Petit et al., 2019; Salameh et al.,
2018; Srivastava et al., 2018; Waked et al., 2014) or a mix of these
different measurement techniques~(Costabile et al., 2017; Vlachou et
al., 2018, 2019). Score of results indicate that PM originate from a
wide variety of sources, not only from natural (volcano, sea spray, soil
dust, vegetation, bacteria, pollen\ldots) or anthropogenic (road
traffic, residential heating, industry\ldots) sources, but also is
formed as secondary product and condensed from the gaseous phase
(ammonium-nitrate and -sulfate\ldots). As a result, the chemistry, size
distribution or reactivity of PM widely vary from location to location
and season to season, which induces large changes in the health impacts
depending on all of these parameters~(Kelly and Fussell, 2012).

Faced with this diversity, it is understandable that epidemiological
studies have difficulty in finding clear links between the atmospheric
mass concentration of PM and the associated short-term health impacts.
Indeed, the mass of PM may not be the relevant metric when dealing with
health impacts of airborne particles since major properties (chemistry,
shape, size distribution, solubility, speciation) driving PM toxicity
are not taken into account within the mass. It is now believed that the
measurement of the reactive oxygen species (ROS) should be more closely
linked to the potential adverse health effects of atmospheric PM, since
oxidative stress is a key factor in the inflammatory response of the
organism, leading for instance to respiratory diseases or when exposed
for a long period of time, cardiovascular diseases or even
cancer~(Lelieveld et al., 2015; Li et al., 2003). Therefore, the
oxidizing potential (OP) of PM being an indirect measure of the ability
of the particles to induce ROS in a biological medium~(Ayres et al.,
2008; Cho et al., 2005; Li et al., 2009; Sauvain et al., 2008) has been
proposed as a potential proxy of the health impacts of atmospheric
exposure. Indeed, some recent studies already established associations
between OP and different possible health outcomes~(Costabile et al.,
2019; Karavalakis et al., 2017; Steenhof et al., 2011; Strak et al.,
2017b; Tuet et al., 2017; Weichenthal et al., 2016b, 2016a).

The demonstration of the OP to be a good proxy of health impact is still
needed. At this point, tThere is also no clear consensus toward a
standardized method to measure the OP of PM, and many assays and
protocols co-exist (DTT, GSH, AA, ESR, °OH or
H\textsubscript{2}O\textsubscript{2}, among others), with samples
extracted with different methods (water, simulated lung fluid (SLF),
etc.)., and not always with a constant mass of PM. However, the
dithiothreitol (DTT) and ascorbic-acid (AA) assay are widely used in
associations with health endpoints ~(Abrams et al., 2017; Atkinson et
al., 2016; Bates et al., 2015; Canova et al., 2014; Fang et al., 2016;
Janssen et al., 2015; Strak et al., 2017a; Weichenthal et al., 2016a;
Yang et al., 2016; Zhang et al., 2016) even if the exact methodologies
differ from one study to the other. The same is true for the seasonality
of OP based on these two assays~(Bates et al., 2015; Calas et al., 2019;
Cesari et al., 2019; Fang et al., 2016; Ma et al., 2018; Paraskevopoulou
et al., 2019; Perrone et al., 2016; Pietrogrande et al., 2018b; Verma et
al., 2014, 2015a; Weber et al., 2018; Zhou et al., 2019). Finally,
several studies also linked OP variabilities with the contributions of
emission sources of PM (references) Several studies have already shown
that different sources of PM have different reactivity to OP
tests~(Bates et al., 2015; Cesari et al., 2019; Fang et al., 2016;
Paraskevopoulou et al., 2019; Verma et al., 2014; Weber et al., 2018;
Zhou et al., 2019). In particular, sources with high concentrations of
transition metals, such as road traffic, appear to have a higher
intrinsic oxidizing potential than other sources of PM., even if the
exact methodologies differ from one study to the other.

Thus, it is now necessary to know if this new parameter of PM (oxidizing
potential) complements the usual metric (mass concentration per cubic
meter).

Several studies have already shown that different sources of PM have
different reactivity to OP tests~(Bates et al., 2015; Cesari et al.,
2019; Fang et al., 2016; Paraskevopoulou et al., 2019; Verma et al.,
2014; Weber et al., 2018; Zhou et al., 2019). In particular, sources
with high concentrations of transition metals, such as road traffic,
appear to have a higher intrinsic oxidizing potential than other sources
of PM. However, the studies in question are still rare and do not always
take into account complete seasonal cycles and therefore may not
encompass the variety of sources for a given site omitting some
important sources. Also, spatial variability at the country-scale is
currently unknown and requires homogeneous sampling and analysis
methodologies for all filters and time-series.

In order to address these questions, we gathered in this study an
extensive database of about 1 700 samples from 15 yearly time-series of
observations over continental France, collected during many research
programs conducted between 2013 and 2018. On each of these samples, we
concurrently measured the OP with the DTT and AA assays, together with
an extensive chemical characterization allowing PM source apportionment
using a harmonized PMF (Positive Matrix Factorization) approach~(Weber
et al., 2019). Then, we apportioned the OP measured by the DTT and AA
assay to the emission sources using a multilinear regression approach,
following~Weber et al. (2018). In this way, we can estimate the
oxidizing capacity of each microgram of PM from the different identified
emission sources but also the relative contribution of the different
sources to the OP\textsuperscript{DTT} and OP\textsuperscript{AA} on
seasonal and daily basis.

\hypertarget{material-and-method}{%
\section{Material and method}\label{material-and-method}}

\hypertarget{sites-description}{%
\subsection{Sites description}\label{sites-description}}

The selected sites had to fulfill three conditions: 1) a yearly sampling
period, 2) the required chemical analysis to perform a standardized PMF
study and 3) enough filter surface left to assess the OP measurements. A
total of 14 sites were included in this study (one being sampled 2 times
at 5 years interval) taken from different research programs. The chosen
sites reflect the diversity of typology we could encounter in the
western Europe: urban (NGT, TAL, AIX, MRS-5av), urban traffic (NIC),
urban alpine valley (GRE-cb, GRE-fr, VIF, CHAM, MNZ, PAS), industrial
(PdB) and traffic (RBX \& STG-ce) and are details in
Table~\protect\hyperlink{tab:tab1}{{[}tab:tab1{]}} and cover different
area of France (Figure~\protect\hyperlink{fig:fig1}{{[}fig:fig1{]}}).
All these sites are monitored by the local AASQA (Atmo Sud, Atmo AURA,
Atmo Aquitaine and Atmo HdF). We note, however, the absence of remote or
rural sites in our current dataset.

\includegraphics[width=5.83333in,height=4.08333in]{media/image1.png}

Figure 1. Location of the 15 sampling sites. Color codes denote the
typology of the site: red, urban; orange, urban valley; magenta,
industrial;blue, traffic.

Table 1. Sampling sites meta-data.

\begin{longtable}[]{@{}lllllll@{}}
\toprule
\endhead
Ville & Abbreviation & Typology & Coordinate & Elevation & \# sample &
Date\tabularnewline
Marseille & MRS-5av & Urban bgd & 43.3060 °N, 5.3957 °E & 64 m & 72 &
2015-01-11 → 2015-12-28\tabularnewline
Port-de-Bouc & PdB & Industrial & 43.4019 °N, 4.9819 °E & 1 m & 113 &
2014-06-01 → 2015-05-17\tabularnewline
Aix-en-provence & AIX & Urban bgd & 43.5302 °N, 5.4413 °E & 188 m & 56 &
2013-08-02 → 2014-07-13\tabularnewline
Nice & NIC & Urban traffic & 43.7020 °N, 7.2862 °E & 1 m & 105 &
2014-07-11 → 2015-05-26\tabularnewline
Talence & TAL & Urban bgd & 44.8004 °N, 0.5880 °W & 20 m & 120 &
2012-03-01 → 2013-03-19\tabularnewline
Nogent & NGT & Urban bgd & 49.2763 °N, 2.4821 °E & 28 m & 135 &
2013-01-02 → 2014-05-11\tabularnewline
Grenoble & GRE-fr\_2013 & Urban bgd & 45.1618 °N, 5.7356 °E & 214 m &
225 & 2013-01-02 → 2014-12-29\tabularnewline
Grenoble & GRE-fr\_2017 & Urban bgd & 45.1618 °N, 5.7356 °E & 214 m &
122 & 2017-02-28 → 2018-03-10\tabularnewline
Grenoble & GRE-cb & Urban bgd & 45.1833 °N, 5.7251 °E & 212 m & 124 &
2017-02-28 → 2018-03-10\tabularnewline
Vif & VIF & Urban bgd & 45.0580 °N, 5.6768 °E & 310 m & 125 & 2017-02-28
→ 2018-03-10\tabularnewline
Chamonix & CHAM & Urban valley & 45.9225 °N, 6.8699 °E & 1038 m & 93 &
2013-11-02 → 2014-10-31\tabularnewline
Marnaz & MNZ & Urban valley & 46.0577 °N, 6.5334 °E & 504 m & 89 &
2013-11-02 → 2014-10-31\tabularnewline
Passy & PAS & Urban valley & 45.9235 °N, 6.7136 °E & 588 m & 88 &
2013-11-02 → 2014-10-31\tabularnewline
Roubaix & RBX & Traffic & 50.7065 °N, 3.1806 °E & 10 m & 157 &
2013-01-20 → 2014-05-26\tabularnewline
Strasbourg & STG-cle & Traffic & 48.5903 °N, 7.7450 °E & 139 m & 76 &
2013-04-11 → 2014-04-08\tabularnewline
\bottomrule
\end{longtable}

\hypertarget{sample-analysis}{%
\subsection{Sample analysis}\label{sample-analysis}}

\hypertarget{chemical-speciation}{%
\subsubsection{Chemical speciation}\label{chemical-speciation}}

The PM\textsubscript{10} concentration was measured at each site by
means of an automatic analyzer, according to EN 16450:2017~(CEN, 2017a),
and daily (24~hours) filter samples were collected every third day by
employees of the corresponding regional air quality monitoring network.
Samplings were achieved on pre-heated quartz fiber filters using
high-volume sampler (DA80, Digitel), following EN 12341:2014
procedures~(CEN, 2014). Off-line chemical analysis performed on these
filters are fully described in the respective papers. Briefly, the
elemental and organic carbon fractions (EC and OC) were measured via
thermo-optical analysis (Sunset Lab. Analyzer~(Birch and Cary, 1996))
using the EUSAAR-2 protocol~(Cavalli et al., 2010; CEN, 2017b). Major
water-soluble inorganic contents (Cl\textsuperscript{-},
NO\textsubscript{3}\textsuperscript{-},
SO\textsubscript{4}\textsuperscript{2-} ,
NH\textsubscript{4}\textsuperscript{+}, Na\textsuperscript{+},
K\textsuperscript{+}, Mg\textsuperscript{2+}, and
Ca\textsuperscript{2+}) and methanesulfonic acid (MSA) were determined
using ion chromatography~(CEN, 2017c; Jaffrezo et al., 2005). Many
metals or trace elements (e.g., Al, Ca, Fe, K, As, Ba, Cd, Co, Cu, La,
Mn, Mo, Ni, Pb, Rb, Sb, Sr, V, and Zn) were measured by ICP-AES or
ICP-MS~(Alleman et al., 2010; CEN, 2005; Mbengue et al., 2014). Finally,
various anhydrosugars (including levoglucosan, mannosan, arabitol,
sorbitol, and mannitol) were analyzed using High Performance Liquid
Chromatography followed by pulsed amperometric detection
(HPLC-PAD)~(Waked et al., 2014).

\hypertarget{op-assays}{%
\subsubsection{OP assays}\label{op-assays}}

The same methodology was applied for all the OP measurements of the
collected filters (Calas et al., 2017, 2018, 2019). Shortly, we
performed the extraction of PM into a simulated lung fluid (SLF) to
simulate the bio-accessibility of PM and to closely simulate exposure
conditions. In order to take into account the non-linearity of the OP
with the mass and to have comparable results between them, the
extraction takes place at iso-mass (10 or 25 of PM, depending on the
site), by adjusting the area of filter extracted. The filter extraction
method includes both water soluble and insoluble species. After the SLF
extraction, particles removed from filter are not filtrated, the whole
extract is injected in the multi-wall plate. Samples were processed
using the AA and DTT assays. DTT depletion when in contact with PM
extracts was determined by dosing the remaining amount of DTT with DTNB
(dithionitrobenzoic acid) at different reaction times (0, 15, 30
minutes) and absorbency was measured at 412~using a plate
spectrophotometer (Tecan, M200 Infinite). The AA assay is a simplified
version of the synthetic respiratory tract lining fluid (RTFL)
assay~(Kelly and Mudway, 2003), where only AA is used. AA depletion is
read continuously for 30 minutes by absorbency at 265 nm~ (TECAN, M1000
Infinite). The maximum depletion rate of AA is determined by linear
regression of the linear section data. For both assays, the 96-wells
plate is auto shaken for 3 seconds before each measurement and kept at
37° C~. Three filter blanks (laboratory blanks) and three positive
controls (1,4 Napthoquinone, 24,7 µM~) are included in every plate
(OP\textsuperscript{AA} and OP\textsuperscript{DTT}) of the protocol.
The average values of these blanks are then subtracted from the sample
measurement of this plate. Detection limit (DL) value is defined as
three times of the standard deviation of laboratory blanks measurements
(blank filters in Gamble+DPPC solution).

The samples were stored from 1 to 4 years before they were analyzed for
the DECOMBIO and SOURCES sites. The MobilAir samples were analyzed in
the months following their collection. As mentioned in (Verma et al.,
2015), the OP activity may be impacted by such storage time. However, in
a previous program (ANSES ExPOSURE, 2017), still ongoing, we have been
measuring the same filter over time. Over one year (about one
measurement per month), OP results for DTT and AA assays display
respectively a coefficient of variation of 18 and 12 \%. Hereafter, the
OP\textsuperscript{DTT} and OP\textsuperscript{AA} normalized by air
volume are noted OP\textsubscript{v}\textsuperscript{DTT} and
OP\textsubscript{v}\textsuperscript{AA}, respectively, with unit .

\hypertarget{source-apportionment}{%
\subsection{Source apportionment}\label{source-apportionment}}

The source apportionment of the OP can be performed in two main ways: 1)
include the OP as an input variable for receptor-model (RM)~(Cesari et
al., 2019; Fang et al., 2016; Ma et al., 2018; Verma et al., 2014) or 2)
conduct source attribution to the PM mass and then, using a multiple
linear regression (MLR) model, assign OP to each of the sources from the
source-receptor model~(Bates et al., 2015; Cesari et al., 2019;
Paraskevopoulou et al., 2019; Verma et al., 2015b; Weber et al., 2018;
Zhou et al., 2019). We decided to use the second approach because the
inclusion of OP in the PMF could potentially destabilize the results but
also forces the OP to positive values (see below).

\hypertarget{pm-mass-apportionment-positive-matrix-factorization}{%
\subsubsection{PM mass apportionment: Positive Matrix
Factorization}\label{pm-mass-apportionment-positive-matrix-factorization}}

\hypertarget{methodological-background}{%
\paragraph{Methodological background}\label{methodological-background}}

The source apportionment at the 15 sites was conducted thanks to a
Positive Matrix Factorization (PMF), using the EPA PMF 5.0 software~(US
EPA, 2017) that makes use of the ME-2 solver from~Paatero (1999).
Briefly, the PMF was introduced by Paatero and Tapper (1994) and is now
a common tool for source-apportionment study. It aims at solving the
receptor model equation

\(\begin{matrix}
X & G \cdot F, \\
\end{matrix}\)

where \(X\) is the \emph{n×m} observation matrix, \(G\) is the
\emph{n×p} contribution matrix and \(F\) is the \emph{p×m} factor
profile (or \emph{source}, but some factor are not a proper emission
\emph{source} but may reflect secondary processes), with \emph{n} the
number of sample, \emph{m} the number of measured chemical specie and
\emph{p} the number of profile. Hereafter, the \(G\) matrix will be in
and the \(F\) matrix in of PM.

\hypertarget{pmf-set-up}{%
\paragraph{PMF set up}\label{pmf-set-up}}

Some of the PMFs were run during previous campaign, namely
SOURCES~(Favez et al., 2017; Weber et al., 2019), DECOMBIO~(Chevrier,
2016; Chevrier et al., 2016), MobilAir
(\url{https://mobilair.univ-grenoble-alpes.fr/},~Borlaza et al. (n.d.)).
For the aims of this study, all PMF have been rerun according to a
harmonized methodology, following the SOURCES program, in order to have
a common set of input species and constraints in the model and to have
comparable sources' profiles.

The input species are slightly site-dependent, but include carbonaceous
compound (OC \& EC), ions (SO\textsubscript{4}\textsuperscript{2-},
NO\textsubscript{3}\textsuperscript{-}, Cl\textsuperscript{-},
NH\textsubscript{4}\textsuperscript{+}, K\textsuperscript{+},
Mg\textsuperscript{2+}, Ca\textsuperscript{2+}), organic compounds
(levoglucosan, mannosan, arabitol, manitol (summed and referred to
polyols) and MSA) and metals for a total of about 30 species. The list
of metals is not exactly the same for each of the sites, due to too low
concentrations on some filters. The uncertainties were estimated thanks
to the method proposed by~Gianini et al. (2012) and was tripled if the
signal over noise ratio was below 2 (classified as ``weak'' in the PMF
software). Between 8 to 10 factors were identified for at the different
sites and are summarized in Table SI REF. On each of the PMF, the
possibility of adding constraints to the factors was used to better
disentangle possible mixing between factors and reduce the rotational
ambiguity, based on \emph{a priori} expert knowledge of the sources
geochemistry. A PMF solution was considered valid if it followed the
recommendation of the European guide on air pollution source
apportionment with receptor models~(Belis et al., 2019) as well as the
geochemical identification of the various factors. Estimation of the
precision of the PMF was obtained on both the base and constrained runs
thanks the both the bootstrap (BS) and displacement (DISP) functions of
the EPA PMF5.0.

\hypertarget{similarity-assessment-of-the-pmf-factors}{%
\subsection{Similarity assessment of the PMF
factors}\label{similarity-assessment-of-the-pmf-factors}}

Since PMF resolves sites-specific factor of PM, we may question if a
factor named by the user ``Primary traffic'' at CHAM display indeed a
similar chemistry as a ``Primary traffic'' at NGT for instance. In order
to identify the the chemical similarity of the PMF profiles, a
similarity assessment of all PMF factor profiles was run following the
DeltaTool approach~(Pernigotti and Belis, 2018), similarly to what we
presented recently in~(Weber et al., 2019). Shortly, the DeltaTool
approach compares a pair of factor profile based on it mass-normalized
chemical compounds thanks to 2 different metrics, the Pearson distance
(PD) and the standardized identity distance (SID) (see~Belis et al.
(2015)) for a detailed explanation of theses 2 metrics). The first one
is 1 minus the Pearson correlation coefficient
(\(\text{PD} = 1 - r^{2}\)) and so is strongly influenced by individual
extreme points (namely OC or EC in our dataset) whereas the second one,
SID, is more sensitive to every specie since it includes a normalization
term, expressed as follows

\(\begin{matrix}
\text{SID} = \frac{\sqrt{2}}{m}\sum_{j = 1}^{m}\frac{\left| x_{j} - y_{j} \right|}{x_{j} + y_{j}}, \\
\end{matrix}\)

where \emph{x} and \emph{y} are two different factors profile in
relative mass and \emph{m} the number of common specie in \emph{x} and
\emph{y}.

\hypertarget{op-apportionment}{%
\subsection{OP apportionment}\label{op-apportionment}}

\hypertarget{multiple-linear-inversion}{%
\subsubsection{Multiple linear
inversion}\label{multiple-linear-inversion}}

Similar to~Weber et al. (2018), a multiple linear regression (MLR) was
conducted independently at each sites with the two OP (DDT and AA) being
the dependent variables and the sources contribution obtained from the
PMF being the explanatory variables, following the equation:

\(\begin{matrix}
\text{OP}_{\text{obs}} = G \times \beta + \varepsilon, \\
\end{matrix}\)

with \emph{OP\textsubscript{obs}} a vector of size \emph{n×1} of the
observed OP\textsubscript{v}\textsuperscript{DTT} or
OP\textsubscript{v}\textsuperscript{AA} in , \(G\) is the matrix
(\emph{n×(p+1)}) of the mass contribution of PM sources determined from
the PMF in and a constant unit term for the intercept (no unit),
\emph{β} the coefficients (i.e. intrinsic OP of the source and the
intercept) of size (\emph{(p+1)×1}) in for the intrinsic OP and for the
intercept. The residual term \emph{ε} (\emph{n×1}) accounts for the
misfit between the observation and the model. We expressively did not
fix the intercept to zero. Indeed, if the system is well constrained the
intercept should be close to zero and conversely a non-zero intercept
would point out missing explanatory variables. A weighted least square
(WLS) were used in order to take into account the uncertainties of the
OP measurements.

In our previous study~(Weber et al., 2018), we used a stepwise backward
elimination of negative coefficients.The assumption was that air is an
oxidative environment, thus the PM oxidative potential cannot be
negative. However, the application of this method to our enlarged
dataset led, at some sites, to unrealistic exclusion of almost all
sources together with a bad reconstruction of OP. This observation tends
in favor of possible coating effect of some oxidative species in
presence of other chemical components or non-linear effects, seen in our
model as ``negative'' OP. We then decided to remove the positivity
constraint in this study. Moreover, the removal of this constraint lead
to more homogeneous intrinsic OP for the main PM sources.

A careful data-treatment was performed in order to remove from the
dataset highly specific samples (e.g. firework) since we focus in this
study to the dominant sources and processes leading to the population
exposure of OP.

The uncertainties of the coefficients β given by the MLR were estimated
by bootstrapping (BS) the solutions 500 times with randomly selected 70
\% of the samples in order to account for possible extremes events or
seasonal variations of the intrinsic OP per source. The uncertainty of
the PMF result G is not taken into account.

The computation was done thanks to the WLS method of the
\emph{statsmodels} package of python~(Seabold and Perktold, 2010).

\hypertarget{contribution-of-the-sources-to-the-op}{%
\subsubsection{Contribution of the sources to the
OP}\label{contribution-of-the-sources-to-the-op}}

The contribution G\textsuperscript{OP} in of the sources to the OP is
computed at each site independently and is simply the product of the
intrinsic OP β in from the MLR times the source contribution of the PMF
G in :

\(\begin{matrix}
G_{k}^{\text{OP}} = G_{k} \times \beta_{k} \\
\end{matrix}\)

where \emph{k} is the source considered. The uncertainties of
G\textsuperscript{OP} is computed thanks to the uncertainties of β
estimated from the BS. The uncertainties of the PMF G\textsubscript{k}
is not taken into account.

\hypertarget{results}{%
\section{Results}\label{results}}

Due to the amount of results, unlikely to be summarized into single
graph, we also present an interactive visualization of all the results
at \href{http://getopstandop.u-ga.fr/}{http://getopstandop.u-ga.fr}
where the reader can find all the details per station of the results
presented hereafter.

\hypertarget{pmf-results}{%
\subsection{PMF results}\label{pmf-results}}

A deep discussion on the PM mass source-apportionment results are given
in the respective document~(Borlaza et al., n.d.; Chevrier, 2016; Weber
et al., 2019). However, the PMF of these studies were rerun for the
present study to include a common set of species and validation
criteria, following the SOURCES methodology~(Favez et al., 2017; Weber
et al., 2019) in order to have comparable data across sites. Shortly, we
observed PMF factors from biomass burning (from residential heating,
mainly traced by levoglucosan), primary road traffic (identified by EC,
Cu, Fe, Sn and Ca\textsuperscript{2+}), mineral dust (thanks to Ti,
Ca\textsuperscript{2+}, and others crustal elements), secondary
inorganic (nitrate-rich (NO\textsubscript{3}\textsuperscript{-} and
NH\textsubscript{4}\textsuperscript{+}) and sulfate-rich
(SO\textsubscript{4}\textsuperscript{2-} and
NH\textsubscript{4}\textsuperscript{+}, together with some OC and Se),
salt (fresh (Cl\textsuperscript{-} and Na\textsuperscript{+}) and aged
(Na\textsuperscript{+}, Mg\textsuperscript{2+}) sea salt) as well as
primary biogenic (traced by the polyols) and MSA-rich (traced MSA). Some
other local sources were also identified at some sites, targeting either
some local heavy loaded metals sources with a very low contribution to
the total PM mass ---supposedly linked to industrial process--- or
factor related to shipping emission (namely heavy fuel oil, HFO) at some
coastal sites. The list of the identified factors at each sites is given
in SI REF.

\includegraphics[width=5.83333in,height=3.64583in]{media/image2.png}

Figure 2. Similarity profile in the PD-SID space for all the pairs of
PMF profile identified at all sites (see text for explanation). The mean
(dot) and standard deviation (lines) are reported in the figure. The
values in parenthesis indicate the number of pairs of profile considered
: with (N 2) the number of site where the profile is identified.

To assess if PMF factors identically named are indeed similar in terms
of geochemistry, we reported similarity metrics (PD: Pearson distance
and SID: standardized identity distance defined in~(Pernigotti and
Belis, 2018)) in Figure~\protect\hyperlink{fig:fig2}{{[}fig:fig2{]}} for
all the identified profiles. The figure reports the mean and standard
deviation of the PD and SID for all possible pairs of profile but the
details for each given pair of profile may be found in the website.
According to~Pernigotti and Belis (2018), two profiles are considered
similar if their PD and SID fall down the green rectangle delineating
the area with PD\textless0.4 and SID\textless1. We do observe a good
similarity (i.e. PD\textless0.4 and SID\textless1) for the main sources
of PM, namely biomass burning, nitrate-rich, primary biogenic, road
traffic, sulfate-rich. The dust, aged salt and MSA rich are often
identified and present acceptable SID, but do present important value
for the PD metric. As the PD is sensitive to "extreme points" that can
strongly affect the pearson correlation, this translates in our case
into the species contributing most to the PM mass (mainly OC and EC).
Then these profiles show differences in concentrations mainly for OC and
EC. The MSA-rich is found to be the most variable factor and a detail
analysis of the chemistry profile (see the website) indicates a lot of
differences for the concentration of EC but also
NO\textsubscript{3}\textsuperscript{-} and
NH\textsubscript{4}\textsuperscript{+} from site to site, where some
sites present contribution of theses species to the total PM mass
whereas other don't. Since it includes secondary organic species, its
variability may be explained by different formation or evolution
pathways (humidity, temperature, solar irradiation, etc.). We also point
out that the industrial source has a very diverse chemistry since it
gathers different local industrial processes. Nevertheless, the
geochemical stability of the majority of PMF factors on a regional scale
seems to us sufficient to consider that these emission sources have a
similar chemistry throughout France.

It is also interesting to note that the species contributing to
oxidizing potential have low concentration variability in the source
profiles emitting them. Indeed, previous studies pointed out the role of
transition metals in the OP of PM~(Calas et al., 2018; Verma et al.,
2015a). In our source-apportionment, most of the copper in apportioned
by the road traffic source, which is then suspected to play a key role
in the observed OP\textsuperscript{DTT} and moreover to the
OP\textsuperscript{AA} value since the copper (Cu) is apportioned
between 34 \% to 54 \% (first and third quantile) by the road traffic
factor, with a concentration ranging from 1.7~ to 3.1~ of PM from this
source. We also note that the concentration uncertainties of Cu into the
road traffic is rather low (see website) as well as the concentration
variation across sites, suggesting that the intrinsic OP of this source
may be similar at the regional scale. Also, the levoglucosan, which
shows strong correlation to the OP\textsuperscript{DTT} and
OP\textsuperscript{AA}, is quasi-exclusively apportioned by the
biomass-burning source. Moreover, a lot of OC is contributing to the PM
of this factor and some metals (notably the copper) are apportioned by
this factor (10 \% of the total copper). The biomass burning is then
also strongly suspected to contribute to both OP in wintertime.

\hypertarget{op-seasonality}{%
\subsection{OP seasonality}\label{op-seasonality}}

The 15 time-series for both the OPDTT and OPAA at each site are
presented in SI and the website. A monthly aggregated view is given in
Figure~\protect\hyperlink{fig:fig3}{{[}fig:fig3{]}}, for both the
OP\textsubscript{v}\textsuperscript{DTT} and
OP\textsubscript{v}\textsuperscript{AA}. It is noted that the dataset
covers complete seasonality and is, in this sense, representative of a
regional climatology of the OP since it includes different sources with
different seasonal activities. Similar to the previous work of~Calas et
al. (2018, 2019), extended by new yearly time-series, we observed a
seasonality of both the OP\textsubscript{v}\textsuperscript{DTT} and
OP\textsubscript{v}\textsuperscript{AA}, with higher values of OP during
the cold months compared to the warm months. We also note that during
the winter, the statistical distribution of OP values does not follow a
normal distribution and significant variability is observed. Notably,
and as shown in SI XX and already pointed out in~Calas et al. (2019),
the alpine sites present stronger seasonality compare to the others
(GRE-fr, GRE-cb, VIF, CHAM, MNZ, PAS). Also, some sites do not present
this seasonality, notably the traffic sites (RBX and STG-cle), the urban
traffic (NIC) or the industrial one (PdB). This seasonality could be
explained by the source of biomass combustion, which also exhibits this
type of seasonality, especially in the Alpine valleys, being the major
source of PM in these regions in winter. Also, as presented in the SI XX
and in~Calas et al. (2019), some rapid variation of the
OP\textsuperscript{DTT} and OP\textsuperscript{AA} are observed, with
drastic increase or decrease within the frame of few days, similarly to
the PM10 mass, suspected to be linked with atmospheric circulation and
vertical mixing of the boundary layer, notably due to inversion layer
that traps the pollutants in the valley.

\includegraphics[width=5.83333in,height=2.1875in]{media/image3.png}

Figure 3. Boxenplot of OPvDTT and OPvAA seasonal value. The numbers in
the x-axis indicate the number of observation. Each box represents one
decile and the black line indicates the median of the distribution. Some
values greater than 15.5 nmol min −1 m −3 are not displayed for
graphical purpose.

\hypertarget{correlation-op-sources}{%
\subsection{Correlation OP -- sources}\label{correlation-op-sources}}

This study focuses on the main drivers of OP at the regional scale. For
this reason, we decided to include in the main discussion only the PMF
factor identified at least two-thirds of the sites (i.e. 10 out of 15
sites), namely the aged salt, biomass burning, dust, MSA rich,
nitrate-rich, primary biogenic, primary road traffic and sulfate-rich
sources. The other local sources barely contributed to the total PM mass
and important uncertainties were often attached to them. The only
notable exception is the HFO profile identified at some coastal sites,
discussed hereafter in it's own section. We invite the reader to explore
the website to have the full view of the results.

In order to have a first estimate of the sources that can be associated
with the OP, the Spearman correlation between the source mass
apportionment from the PMF and the measured
OP\textsubscript{v}\textsuperscript{DTT} and
OP\textsubscript{v}\textsuperscript{AA} are presented in
Figure~\protect\hyperlink{fig:fig4}{{[}fig:fig4{]}}. This figure takes
into account all samples from all sites. First of all, the two OP shows
good correlation but do not carry the exact same signal
(r\textsubscript{OP DTT-OPAA}=0.61). Secondly, the only source that
strongly correlates to one OP (r\textgreater0.6) is the biomass burning
to the OP\textsubscript{v}\textsuperscript{AA}. Some mild correlation
(0.3\textless{} r\textless0.6) are found for the
OP\textsubscript{v}\textsuperscript{DTT} vs. road traffic, biomass
burning, nitrate-rich and dust and for the
OP\textsubscript{v}\textsuperscript{AA} vs. nitrate rich and road
traffic. This result is in line with previous study, either with the
source correlation~(Weber et al., 2018) or with the proxy of sources
(namely, levoglucosan for biomass burning and EC, iron, copper or PAH
for road traffic)~(Calas et al., 2019, 2018; Charrier and Anastasio,
2012; Cho et al., 2005; Hu et al., 2008; Janssen et al., 2014; Künzli et
al., 2006; Ntziachristos et al., 2007; Pietrogrande et al., 2018a; Verma
et al., 2014, 2009). However, the nitrate rich source, mildly correlated
to both OP, doesn't present any atmospheric compound that are known to
be redox-active (to our knowledge). This correlation may then not
reflect any causality and the multilinear regression should disentangle
possible co-variation of the nitrate-rich source and other redox-active
one.

\includegraphics[width=5.83333in,height=1.75in]{media/image4.png}

Figure 4. Spearman correlation coefficient between the OPvDTT and the
OPvAA to the 8 PM sources identified at at least two third of the sites.
All sites are grouped together in this figure, taking into account the
whole time-series of measurement (number of observation days: Aged salt
1430; Biomass burning 1700, Dust 1489, MSA rich 1595, Nitrate rich 1700,
Primary biogenic 1700, Road traffic 1587, Sulfate rich 1524).

\hypertarget{accuracy-of-the-mlr-model}{%
\subsection{Accuracy of the MLR model}\label{accuracy-of-the-mlr-model}}

The MLR statistical validation was carried out by a residual analysis
between the OP observed and the OP reconstructed by the model. For this
evaluation, the intrinsic OP of the sources was set to the mean of the
500 bootstrap values. Table~\protect\hyperlink{tab:tab2}{{[}tab:tab2{]}}
presents the fitted line between the modeled and observed
OP\textsubscript{v}\textsuperscript{DTT} and
OP\textsubscript{v}\textsuperscript{AA} together with the Pearson
correlation coefficient. Details and individual scatter plot are given
the SI XX. All but two sites present a very good correlation between
observed and reconstructed OP (r²\textgreater0.7) and a regression line
close to unity. We note that two models (STG-cle for AA and VIF for DTT)
present a clear lower correlation (r²=0.46 and 0.49, respectively)
together with the fitted line the farthest than the y=x line. However,
an in-depth analysis highlights that such a correlation is driven by few
days of observation on this regression (see SI XX). We therefore
consider our models valid for the rest of this study and each intrinsic
OP (i.e. coefficient of the regression) may be explored individually to
geochemically explain the observed OP.

\hypertarget{limit-of-the-linear-model}{%
\subsection{Limit of the linear model}\label{limit-of-the-linear-model}}

Despite our models being able to reproduce most of the observations, we
can also see in figure SI XX that even if the MLR produces normally
distributed residuals (observation minus model), it also tends to
underestimate the highest values and the residuals are often
heteroscedastics (i.e. the higher values, the higher the uncertainties).
Then, the underlying hypothesis of linearity between endogenous
variables (source PM concentration) and exogenous variables (OP) may be
deemed invalid. The reader should keep in mind that non-linear processes
are strongly suspected for the source-apportionment of OP, as already
noted by~Calas et al. (2018) or~Charrier et al. (2016). As a result,
future developments on OP apportionment models should focus on this
suspected non-linearity, either with co-variations terms or even
non-linear model such as neural-network for instance.

Observed vs reconstructed OP equation and Pearson correlation, taking
the mean value of the 500 bootstraps, for all the sites.

\begin{longtable}[]{@{}lll@{}}
\toprule
\endhead
& \textbf{DTTv} & \textbf{AAv}\tabularnewline
MRS-5av & 0.77x+0.60 r²=0.72 & 0.69x+0.10 r²=0.72\tabularnewline
PdB & 0.87x+0.24 r²=0.84 & 0.91x+0.07 r²=0.87\tabularnewline
AIX & 0.91x+0.18 r²=0.82 & 1.01x+0.04 r²=0.92\tabularnewline
NIC & 0.76x+0.52 r²=0.79 & 0.88x+0.12 r²=0.83\tabularnewline
TAL & 0.76x+0.34 r²=0.77 & 0.83x+0.07 r²=0.86\tabularnewline
NGT & 0.81x+0.43 r²=0.75 & 0.97x+0.05 r²=0.87\tabularnewline
GRE-fr\_2013 & 0.97x-0.14 r²=0.79 & 0.69x+0.20 r²=0.79\tabularnewline
GRE-fr\_2017 & 0.77x+0.20 r²=0.81 & 1.02x+0.01 r²=0.94\tabularnewline
GRE-cb & 0.70x+0.35 r²=0.72 & 0.81x+0.20 r²=0.86\tabularnewline
VIF & 0.62x+0.47 r²=0.49 & 0.78x+0.20 r²=0.93\tabularnewline
CHAM & 0.87x+0.21 r²=0.90 & 0.85x+0.14 r²=0.93\tabularnewline
MNZ & 0.87x+0.19 r²=0.95 & 0.90x-0.01 r²=0.96\tabularnewline
PAS & 0.76x+0.72 r²=0.90 & 0.96x-0.06 r²=0.82\tabularnewline
RBX & 0.75x+0.49 r²=0.72 & 0.86x+0.31 r²=0.75\tabularnewline
STG-cle & 0.75x+0.57 r²=0.73 & 0.55x+0.64 r²=0.46\tabularnewline
\bottomrule
\end{longtable}

\hypertarget{intrinsic-op-per-sources}{%
\subsection{Intrinsic OP per sources}\label{intrinsic-op-per-sources}}

Even if the models reproduce the observations correctly, this does not
guarantee that the geochemical meaning extracted is the same for each of
the models, i.e. the intrinsic OP of the sources may completely differ
from site to site. The question is then to identify if a given source
contributes similarly to the OP at all sites. In other words, does all
model extract any geochemical general information relative to the OP?
Figure 5 presents the intrinsic OP\textsuperscript{DTT} \textbf{A)} and
OP\textsuperscript{AA} \textbf{B)} for the selected subset of sources.
The values of all the N bootstraps for all the n stations where the
sources were identified (i.e. between \(500 \times 15 = 7500\) and
\(500 \times 10 = 5000\) values) are represented by the boxenplot to
display the value distribution, together with the mean and standard
deviation in white dots and lines, respectively. The exact values of
mean and standard deviation are also reported in
Table~\protect\hyperlink{tab:tab3}{{[}tab:tab3{]}} and details per
station for all sources are given in Table SI4.

\includegraphics[width=5.83333in,height=2.91667in]{media/image5.png}

Figure 5. Intrinsic OP A) DTT and B) AA for the sources identified at at
least two third of the site (i.e. 10 sites). The graphic displays the n
× N intrinsic OP in parenthesis (with n the number of site where the
source was identified and N = 500 bootstraps) and their statistical
distribution via boxenplot: each box delimit a decile. The mean and
standard deviation are also given in white dots and lines, respectively.

First of all, the mean values of the intrinsic OP are almost always
found positive when considering the whole dataset. The only case where a
small negative mean intrinsic OP found is for the MSA rich factor for
the AA assay (-0.018(152) ), but strong variance is attached to this
result. It then confirms, if still needed, that airborne particles are
an oxidant material.

We also observe a net distinction of the intrinsic OP for the different
sources of PM, with intrinsic OP ranging from 0.044(64) to 0.223(85) for
the DTT and -0.018(152) to 0.197(104) for the AA. We confirm the
previous studies that also found different reactivity (or intrinsic OP)
for different sources based on RM techniques~(Ayres et al., 2008; Bates
et al., 2015; Cesari et al., 2019; Costabile et al., 2019; Fang et al.,
2016; Paraskevopoulou et al., 2019; Perrone et al., 2019; Verma et al.,
2014; Weber et al., 2018; Zhou et al., 2019). Notably, the road traffic
source is the most reactive source toward the OP\textsuperscript{DTT},
with a value of 0.223(85) which is almost twice the value of the other
group of reactive source, namely 0.132(410), 0.129(65), 0.121(114) and
0.112(113) for the MSA rich, biomass burning, dust and primary biogenic
sources, respectively. Interestingly, the nitrate-rich factor is the
second most correlated to the OP\textsubscript{v}\textsuperscript{DTT}
but is the one associated with the lowest intrinsic
OP\textsuperscript{DTT} (0.044(65)). Concerning the intrinsic
OP\textsuperscript{AA}, less sources present redox activity. Notably,
only the biomass burning and road traffic present an intrinsic OP easily
distinguishable from 0, with respect to the uncertainties with intrinsic
OP\textsuperscript{AA} of 0.197(103) and 0.161(108), respectively.
Overall, it appears that the intrinsic OP of the different sources
(expect the MSA-rich) are similar at a regional level, paving the road
to the implementation of OP in CTM.

Overall, the OP\textsuperscript{DTT} is more balanced than the
OP\textsuperscript{AA} as already pointed by~Fang et al. (2016)
and~Weber et al. (2018) and seems to target all the sources containing
either metals and organic species but is not sensitive to ammonium
nitrate source. OP\textsuperscript{AA} mainly target biomass burning and
primary road traffic factor, as already pointed out in previous
studies~(Bates et al., 2019 and references therin). We then confirm what
previous studies found either by direct OP measurement at the source or
by source -apportionment. It is however hard to directly compare the
absolute value of our result to the literature since all protocols vary
from one group to another. The coefficient of variation (CV, standard
deviation over mean) of the intrinsic OP are the lowest for the
\textbf{biomass burning} and primary \textbf{road traffic} for the DTT
assay with values of 0.5 and 0.38, respectively, as well as for the AA
assay with value of 0.52 and 0.67, respectively. The variability of the
biomass burning intrinsic OP is more site dependent, with a low
uncertainty at any given site, but with slightly different intrinsic OP
between sites. It then suggests that the variability is not linked to
uncertainties of the model but from local variation of the chemistry of
this profile. However, the biomass burning is the most geochemically
stable profile in our dataset, with a PD \textless{} 0.1 and SID
\textless{} 0.7. Hence, the variability should come from species not
measured in our dataset. Namely, no PAH, oxy-PAH, OH-PAH nor quinone are
measured, although they are known to contribute to the OP~(Charrier and
Anastasio, 2012) and have short live time and being heavily influenced
by the climatic condition (Miersch et al., 2019). The road traffic
chemical profile is also similar at all sites (with the exception of RBX
and VIF where mixing effects are observed) with our given set of
species. Moreover, the uncertainty of the road traffic intrinsic OP at
each site lies in the uncertainties of the other sites. Hence, the low
variability for the OP\textsuperscript{DTT} indicates that the observed
species, similar at each sites, are the one that influence
OP\textsuperscript{DTT}. However, for the OP\textsuperscript{AA} the
variability is higher with some important difference from site to site,
without clear distinction by typology or groups of sites. Then, some
species that are not measured here may influence the
OP\textsuperscript{AA}, but not the OP\textsuperscript{DTT}.

The inorganic factor (\textbf{sulfate-rich} and \textbf{nitrate-rich})
presents a high CV. However, the CV may not be an accurate measure for
some sources with near 0 mean intrinsic OP. The standard deviation is
similar to the one of the biomass burning and road traffic for the OP
DTT, and are among the lowest variability for the OP\textsuperscript{AA}
(see table REF). The two inorganic factor are also very similar at each
site in term of chemical composition, as presented in the SID-PD space
in Figure~\protect\hyperlink{fig:fig2}{{[}fig:fig2{]}}.

The \textbf{dust} source presents an important variability when taking
into account the whole sites, but a deeper analysis shows 2 groups of
sites: AIX-RBX-VIF and the others. The first group presents high
intrinsic OP for both assays, whereas the others display half
(OP\textsuperscript{DTT}) or almost null (OP\textsuperscript{AA})
intrinsic OP (see table XXX). Then, the first conclusion is that 80 \%
of the sites agree on a common intrinsic OP for the dust source. The
high variability for VIF may be explained by different chemistry.
Indeed, the dust factor at VIF highly differs from the other dust
factors with a PD \textgreater{} 0.75 when compared to the other sites
(see SUP INFO). We don't have clear hypothesis yet for the two others
sites.

Despite the different PMF (and so number of sources) solution at each
site and the different OP signal, the rather low variability of the
intrinsic OP for a given source suggests that a given source of PM
behave similarly at each site with regards to the OP. It then supports
the idea that, at the national scale, the sources described above have a
determined and stable intrinsic OP. This result achieved the very first
requirement before a potential implementation of the OP in deterministic
chemistry transport model (CTM).

\hypertarget{variability-of-the-biogenic-and-organic-sources}{%
\subsubsection{Variability of the biogenic and organic
sources}\label{variability-of-the-biogenic-and-organic-sources}}

High variability is found for the \textbf{MSA rich} source with high
variability between sites but also at a given site, with a CV of 3.1 and
7.8 for the DTT and AA assay, respectively.

The secondary organic source appears to be the most variable source in
term of intrinsic OP, notably for the DTT assay. The MSA rich factor is
the one among the 8 major sources of PM that contributes the less to the
total PM mass and the PMF bootstrap result presents important
variability for the PM\textsubscript{10} apportioned by this source. In
our study, the secondary organic factor (SOA) is mainly traced by the
MSA, but its remaining composition is yet unknown. This PMF factor is
still not yet fully understood and few studies reported it so far. As a
result, we don't know for instance the loading of HULIS, quinone or
isoprene derived compounds in this factor, nor the amount of aging this
factor encountered at each site. Such unknown is reflected in the MSA
rich factor being the less similar factor between sites (see previous
similarity assessment section). Such uncertainties on the chemistry of
this factor may explain the diversity if intrinsic OP observed. Indeed,
it has been shown that SOA species may contribute to the
OP\textsuperscript{DTT} in the early stage of aging~(McWhinney et al.,
2013) or to the OP\textsuperscript{DCFH}~(Zhou et al., 2018), but due to
aging processes and photochemical degradation, aged SOA species may
contribute to a lesser extent to it~(Jiang and Jang, 2018; Wang et al.,
2017). But~Verma et al. (2009, 2015a) pointed that aging of SOA may
increase the OP. Moreover,~Tuet et al. (2019) shown that humidity and
seed particles have also an effect on SOA OP\textsuperscript{DTT}. Since
all these parameters vary in our study, it may explain the diversity of
chemistry and thus intrinsic OP\textsuperscript{DTT} we observed for
SOA. To a lesser extent, this variability is also found for the AA
assay, suggesting that the same phenomena affects the SOA
OP\textsuperscript{AA}. The sulfate-rich factor is also suspected to
account for some SOA since OC are present in non-negligible amounts.
Then, the variability of the sulfate-rich, and similarly for the
\textbf{aged sea-salt}, intrinsic OP may be explained by the same
phenomena.

Also the \textbf{primary biogenic source}, mainly traced by polyols,
presents some variability for the OP\textsuperscript{DTT}. Samaké et al.
(2017) highlighted that spore or bacteria does contribute to the
OP\textsuperscript{DTT} and OPAA activity, even if the microbial cells
are dead. However, the authors also present the inhibition of the DTT
loss rate in presence of metals and 1,4-naphtoquinone or Cu. Since both
results are observed, depending on which microbiota are living on the
sampled PM, intrinsic OP\textsuperscript{DTT} may be enhanced or
decreased. The variability of intrinsic OP\textsuperscript{DTT} observed
in Figure 5 may then reflect the local different microbiology living in
the PM or covariation of the primary biogenic source with another metal
or 1,4-naphtoquinone rich source for instance. Also,~Samaké et al.
(2019b) pointed that some secondary species may be incorporated in this
factor at some site, making it a mix of primary biogenic and SOA,
another hypothesis is the ``aging'' of this factor. It is then possible
that the SOA mixed in the primary biogenic may influence the intrinsic
OP in different way, similarly to the MSA rich factor.

\begin{longtable}[]{@{}llllllllll@{}}
\toprule
\endhead
\textbf{OP type} & \textbf{station} & \textbf{Aged salt} &
\textbf{Biomass burning} & \textbf{Dust} & \textbf{MSA rich} &
\textbf{Nitrate rich} & \textbf{Primary biogenic} & \textbf{Road
traffic} & \textbf{Sulfate rich}\tabularnewline
\textbf{DTT} & \textbf{All stations and BS} & 0.038(113) & 0.129(65) &
0.121(114) & 0.132(410) & 0.044(65) & 0.112(113) & 0.223(85) &
0.077(82)\tabularnewline
& AIX & -0.134(55) & 0.107(28) & 0.224(58) & 0.051(63) & 0.015(70) &
0.072(72) & 0.200(74) & -\tabularnewline
& CHAM & - & 0.092(6) & 0.087(18) & 0.240(49) & 0.065(28) & 0.131(17) &
0.406(60) & 0.082(17)\tabularnewline
& GRE-cb & 0.018(46) & 0.105(26) & 0.122(21) & 0.779(210) & 0.053(21) &
0.282(68) & 0.163(42) & -0.017(33)\tabularnewline
& GRE-fr\_2013 & 0.181(69) & 0.261(22) & 0.131(32) & 0.080(136) &
-0.021(60) & 0.165(47) & 0.206(33) & 0.186(30)\tabularnewline
& GRE-fr\_2017 & -0.083(163) & 0.102(47) & 0.094(39) & -0.173(323) &
0.008(18) & 0.237(125) & 0.183(38) & 0.054(35)\tabularnewline
& MRS-5av & 0.110(44) & 0.114(68) & 0.025(22) & 0.168(224) & 0.088(102)
& -0.046(90) & 0.243(53) & 0.025(35)\tabularnewline
& MNZ & - & 0.116(5) & 0.069(15) & 0.095(31) & 0.053(7) & 0.071(9) &
0.253(18) & 0.086(7)\tabularnewline
& NIC & 0.120(26) & 0.117(13) & 0.110(27) & - & 0.081(46) & 0.004(48) &
0.279(84) & 0.082(24)\tabularnewline
& NGT & 0.116(43) & 0.191(35) & - & 0.533(298) & 0.023(35) & -0.038(114)
& 0.159(35) & 0.198(46)\tabularnewline
& PAS & - & 0.129(11) & 0.109(15) & 0.161(75) & 0.161(55) & 0.186(31) &
0.256(44) & 0.114(21)\tabularnewline
& PdB & 0.032(32) & 0.179(14) & 0.140(19) & 0.123(80) & 0.017(22) &
0.161(26) & - & 0.098(35)\tabularnewline
& RBX & -0.006(50) & -0.018(43) & 0.251(16) & -0.096(72) & 0.066(9) &
0.004(42) & 0.189(28) & -0.106(46)\tabularnewline
& STG-cle & 0.137(37) & 0.155(17) & - & 0.370(71) & 0.084(18) &
0.154(35) & 0.197(18) & 0.098(21)\tabularnewline
& TAL & -0.044(30) & 0.147(19) & 0.103(21) & 0.063(43) & -0.036(32) &
0.142(45) & 0.272(96) & -\tabularnewline
& VIF & 0.004(68) & 0.144(28) & 0.105(344) & -0.540(870) & 0.002(19) &
0.148(38) & 0.116(32) & 0.106(25)\tabularnewline
\textbf{AA} & \textbf{All stations and BS} & 0.024(54) & 0.197(103) &
0.037(86) & -0.020(156) & 0.010(56) & 0.028(67) & 0.161(108) &
0.009(24)\tabularnewline
& AIX & -0.049(46) & 0.194(18) & 0.157(27) & 0.051(58) & 0.118(82) &
0.030(68) & 0.221(68) & -\tabularnewline
& CHAM & - & 0.174(12) & -0.048(15) & 0.001(37) & -0.032(39) & 0.025(15)
& 0.297(43) & -0.011(8)\tabularnewline
& GRE-cb & -0.007(27) & 0.169(36) & -0.015(16) & -0.169(103) & 0.030(12)
& 0.015(38) & 0.196(24) & 0.029(10)\tabularnewline
& GRE-fr\_2013 & 0.037(19) & 0.183(11) & 0.012(11) & -0.036(32) &
-0.004(14) & 0.032(14) & 0.129(9) & 0.002(13)\tabularnewline
& GRE-fr\_2017 & 0.103(83) & 0.368(26) & 0.017(8) & -0.016(100) &
-0.036(17) & -0.010(36) & 0.182(20) & 0.042(11)\tabularnewline
& MRS-5av & 0.016(7) & 0.101(12) & 0.004(7) & -0.029(34) & 0.004(9) &
0.001(18) & 0.023(15) & 0.001(6)\tabularnewline
& MNZ & - & 0.266(12) & 0.006(11) & 0.082(16) & -0.084(30) & 0.027(9) &
0.154(22) & 0.014(9)\tabularnewline
& NIC & 0.034(22) & 0.112(11) & -0.001(16) & - & -0.018(17) & -0.034(28)
& 0.354(47) & 0.013(10)\tabularnewline
& NGT & 0.069(31) & 0.428(60) & - & -0.078(234) & 0.063(35) & 0.225(95)
& 0.074(46) & -0.016(38)\tabularnewline
& PAS & - & 0.161(13) & -0.002(7) & -0.013(35) & 0.040(58) & 0.007(16) &
0.047(23) & -0.003(15)\tabularnewline
& PdB & 0.011(14) & 0.138(7) & 0.034(8) & -0.003(26) & -0.011(15) &
0.035(11) & - & 0.006(8)\tabularnewline
& RBX & 0.069(26) & 0.223(42) & 0.167(23) & 0.090(63) & 0.019(12) &
0.036(40) & 0.317(39) & 0.028(36)\tabularnewline
& STG-cle & -0.023(30) & 0.026(25) & - & 0.080(20) & 0.044(12) &
0.015(25) & 0.128(17) & -0.007(11)\tabularnewline
& TAL & 0.008(17) & 0.134(15) & 0.006(9) & -0.015(30) & 0.004(10) &
0.004(27) & 0.081(51) & -\tabularnewline
& VIF & 0.021(41) & 0.284(23) & 0.150(181) & -0.221(385) & 0.015(20) &
0.014(22) & 0.052(15) & 0.021(9)\tabularnewline
\bottomrule
\end{longtable}

\hypertarget{contribution-of-the-sources-to-the-op-1}{%
\subsection{Contribution of the sources to the
OP}\label{contribution-of-the-sources-to-the-op-1}}

Since the intrinsic OP of the 8 sources may be considered as stable
(except the MSA-rich), we can estimate the sources contribution to the
OP, similarly to the sources contribution to the PM concentration. In
this part, we estimate the contribution of the sources to the OP and PM
concentration at each sites, then present an aggregated view of the
seasonal contribution in Figure 6, and the daily contribution in Figure
7 and 8, taking into account all the sites. Since our dataset included
an important proportion of alpine sites and given the fact that all
sites are urbanized, the extrapolation to the whole France or other
region should be done cautiously. Also, only the contributions of the 8
previously selected sources are reported here. The full results with
details per site are presented in the website mentioned above.

\hypertarget{seasonality-of-the-contribution}{%
\subsubsection{Seasonality of the
contribution}\label{seasonality-of-the-contribution}}

\includegraphics[width=5.83333in,height=3.45694in]{media/image6.png}

Figure 6. Mean monthly contribution of the main 8 sources to the (a)
PM10 mass, (b) OPvDTT and (c) OPvAA taken into account each sources
contribution of every sits; and there respective normalized contribution
in (d) PM10 mass, (e) OPvDTT and (f) OPvAA (taking into account only
these 8 sources).

It is already known and well documented that PM\textsubscript{10} mass
concentration presents a seasonality, notably in alpine valley due to
the biomass burning source and meteorological effects. By construction,
the contribution to OPDTTv and OPAAv from each source is proportional to
the mass contribution of the source in question. However, the relative
importance of the contributions of the sources is altered by the fact
that they have different intrinsic OPs. The question, therefore, is what
are the sources of PM that contribute the most to the OP. The mean
monthly contribution of the 8 sources from all sites to the
PM\textsubscript{10} mass, OP\textsubscript{v}\textsuperscript{DTT} and
OP\textsubscript{v}\textsuperscript{AA} are presented in
Figure~\protect\hyperlink{fig:fig6}{{[}fig:fig6{]}} \textbf{(a)},
\textbf{(b)} and \textbf{(c)}, respectively.

As already shown by previous study in France~(Favez et al., 2017; Petit
et al., 2019; Srivastava et al., 2018; Waked et al., 2014; Weber et al.,
2019), the seasonal mean contribution to the PM\textsubscript{10} mass
show the importance of the biomass burning source, followed by the
secondary inorganic (sulfate-rich and nitrate-rich), the dust and road
traffic. But as a direct consequence of the different intrinsic OP, we
do observe a redistribution of the relative importance of the sources.
Namely, the nitrate-rich source that may contribute to a significant
amount to the PM\textsubscript{10} mass, notably in spring, barely
contribute to the OP\textsubscript{v}\textsuperscript{DTT} nor to the
OP\textsubscript{v}\textsuperscript{AA}. Conversely, the road traffic
contributes to about 15 \% during summer to the mean
PM\textsubscript{10} mass but is more than 50 \% of the mean
OP\textsubscript{v}\textsuperscript{AA} in the same period of time
(Figure~\protect\hyperlink{fig:fig6}{{[}fig:fig6{]}} \textbf{(d)},
\textbf{(e)} and \textbf{(f)}). However, the biomass burning sources is
still a major contributor to both the
OP\textsubscript{v}\textsuperscript{DTT} and
OP\textsubscript{v}\textsuperscript{AA} during winter months. We note
that the primary biogenic source still contributes to the
OP\textsubscript{v}\textsuperscript{DTT} but to a lesser extent to the
OP\textsubscript{v}\textsuperscript{AA}. Also the dust source continues
to be an important contributor to the
OP\textsubscript{v}\textsuperscript{DTT} but not to the
OP\textsubscript{v}\textsuperscript{AA}. These results tend to confirm
what previous studies already found. With regard to the seasonality of
OP, regulations should target the biomass burning emission in order to
decrease the PM\textsubscript{10} OP during winter by a large amount,
but also the road traffic that contributes homogeneously to both OP all
around the year.

\hypertarget{daily-mean-and-median-contribution}{%
\subsubsection{Daily mean and median
contribution}\label{daily-mean-and-median-contribution}}

\includegraphics[width=5.83333in,height=3.025in]{media/image7.png}

Figure 7. Averaged daily contribution of the sources to (a) the PM10
mass, (b) the OPvDTT and (c) the OPvAA. The bars represent the mean and
the error bars the 95 \% confidence interval of the mean.

\includegraphics[width=5.83333in,height=3.025in]{media/image8.png}

Figure 8. Median daily contribution of the sources to (a) the PM mass,
(b) the OPvDTT and (c) the OPvAA. The bars represent the mean and the
error bars the 95 \% confidence interval of the median.

The seasonal mean contribution of the source may not take into account
the rapid day-to-day variability of the OP observed in SI XXX. Moreover,
most epidemiological studies average the yearly exposition to a ``daily
averaged'' or ``daily median'' exposure~(World Health Organization,
2016). In this part, we try to express the population exposition to the
sources to both the mass and OP metric on a daily basis.

The daily contribution is shown in
Figure~\protect\hyperlink{fig:fig7}{{[}fig:fig7{]}} and
Figure~\protect\hyperlink{fig:fig8}{{[}fig:fig8{]}}, for the mean and
median daily contribution, respectively. The results highly differ if
considering the mean or the median contribution, and the two statistical
indicators do not answer the same question. In particular, mean
contribution is determined by the sum of individual measurements making
it highly sensitive to outliers. On the other hand, median contribution
is the middle value when the measurements in the actual dataset are
arranged in ascending order. The skewness of the distribution is not
surprising as some high PM\textsubscript{10} events (i.e., outliers in
the dataset) were present in our measurements. This is also specifically
anticipated in alpine areas (XX out of 15 sites) where the development
of atmospheric inversions causing increased pollutant concentrations are
highly favorable.

Therefore, the mean contribution is more related to the question
``\emph{What Which sources contribute the most to the
PM\textsubscript{10} OP or mass?}'' whereas the median contribution
addresses the question ``\emph{What is the} \emph{chronic}
\emph{population exposure to the PM\textsubscript{10} OP or mass?}''.
Since there is strong seasonality, day-to-day variability and non-
normally distributed contribution of the source to the
PM\textsubscript{10} mass, notably for the biomass burning, we do
observe important differences when looking at the mean or median
contribution. Therefore, we decided to report both statistics in this
study. Here again, we want to remind the reader that our dataset
includes several alpine site strongly affected by the residential
biomass burning heating, then the result might not reflect the overall
regional exposure.

We do observe (Figure~\protect\hyperlink{fig:fig7}{{[}fig:fig7{]}}) a
redistribution of the daily \textbf{mean} contribution sources' rank
between the PM\textsubscript{10} mass,
OP\textsubscript{v}\textsuperscript{DTT} and
OP\textsubscript{v}\textsuperscript{AA} similarly to the monthly mean
contribution shown above. Since the biomass burning source is an
important contributor to the PM\textsubscript{10} mass, it contributes
also significantly to both OP and is ranked as the first contributor to
both OP\textsubscript{v}\textsuperscript{DTT} and
OP\textsubscript{v}\textsuperscript{AA} mean daily contribution (mean
0.51 and 0.72, respectively). The road traffic source contribution, due
to its highest intrinsic OP in both assays, has almost the same daily
mean contribution than the biomass burning for the
OP\textsubscript{v}\textsuperscript{DTT}, and is the second contributor
to the daily mean OP\textsubscript{v}\textsuperscript{AA}, with half the
contribution of the biomass burning (mean 0.50 and 0.34, respectively).
The other source barely not contribute to the
OP\textsubscript{v}\textsuperscript{AA} (\textless0.1 ). For the
OP\textsubscript{v}\textsuperscript{DTT}, the dust is the third
contributor (mean 0.34), followed by the sulfate rich and primary
biogenic (0.23 and 0.16 , respectively). The nitrate-rich, aged seasalt
and MSA-rich present a low contribution (mean \textless0.1), due either
to their low intrinsic OP or to their low contribution to the PM mass.

However, for the daily \textbf{median} contribution, due to the high
seasonality of the biomass burning source and the consistent
contribution throughout the year of the road traffic, sulfate-rich and
dust sources, the ranks of the sources is even more drastically
redistributed between all the considered metric
(Figure~\protect\hyperlink{fig:fig8}{{[}fig:fig8{]}}) A) mass, B)
OP\textsubscript{v}\textsuperscript{DTT} and C)
OP\textsubscript{v}\textsuperscript{AA}). Moreover, the absolute value
of the contribution is also lowered compare to the mean daily
contribution, due to low frequencies of highly loaded PM events. The
major source contributing to the
OP\textsubscript{v}\textsuperscript{DTT} is now the road traffic (median
0.36) and contributes more than twice as much as the second source,
namely the dust one (median 0.16 ), followed by the sulfate-rich (median
0.13 ) and then the biomass burning (median 0.11 ). For the
OP\textsubscript{v}\textsuperscript{AA}, the two dominant sources are
the road-traffic (median 0.29 ) and the biomass burning (median 0.19 ).
The others sources being negligible in front of them (aged seasalt 0.016
, primary biogenic 0.015 and the others contributes less than 0.01 ).

The high differences between the mean and median contribution has strong
implication for air quality policy. Indeed, as previously shown, the
biomass burning may contribute to more than 50 \% of the high OP during
winter, and even more for some days. However, such events do not
represent the daily exposure of the population. Even if the regulation
should target those events to prevent acute exposure, we should also
take into account the long term exposure to a lower but constant level
of pollutant. With this respect, the biomass burning source impact is
drastically lowered but the road traffic emission become a major
concern. Moreover, our dataset includes several site in alpine valley
that bias our result toward a more important contribution of the biomass
burning source compare to the actual urbanized French cities. Then, our
conclusion strongly supports that at the regional scale, the important
source to target in order to decrease the chronic exposure to PM
pollutant is the road traffic one. However, site specific typologies
(notably alpine valley, or ``hot spot'' with specific sources) may also
target other sources, notably the biomass burning source in alpine
valley, to prevent acute exposure of the population.

\hypertarget{other-pmf-factors}{%
\subsection{Other PMF factors}\label{other-pmf-factors}}

So far, we have focused on sources with regional impact. However, we
would like to discuss the case of 2 sources that are often controversial
for political decision-makers or inhabitants but are less representative
of the urban background of French cities.

\hypertarget{heavy-fuel-oil}{%
\subsubsection{Heavy fuel oil}\label{heavy-fuel-oil}}

Among the others sources, not present at at least 2 third of the site,
the Heavy fuel oil (HFO) identified at MRS-5av and PdB, present an
intrinsic OP\textsuperscript{DTT} of 0.51(14) and 0.21(4), respectively,
and an intrinsic OP\textsuperscript{AA} of 0.04(2) and 0.11(3),
respectively. The intrinsic OP\textsuperscript{DTT} is then in average
higher than the road traffic one, making the HFO the second contributor
of the daily mean and median source contribution at MRS-5av for the
OP\textsuperscript{DTT} contribution and the fourth one for the
OP\textsubscript{v}\textsuperscript{AA} contribution (see website
http://getopstandop.u-ga.fr). For PdB, an important contribution, but
lower than for MRS-5av is found, for both the DTT and AA. Although only
2 sites presented an HFO factor, it suggests that the PM originated from
this source may significantly contribute to the total OP around harbor
cities.

\hypertarget{industrial}{%
\subsubsection{Industrial}\label{industrial}}

An industrial factor was identified at 6 sites. However, the
``industrial'' name for this factor cover a wide variety of chemistry,
as denoted in the similarity
Figure~\protect\hyperlink{fig:fig2}{{[}fig:fig2{]}}. They share a common
set of metal (Al, As, Cd, Mn, Mo, Pb, Rb, Zn), in different
concentration levels. Since we are not aware of any emission source
releasing these metals jointly into the atmosphere, we have named this
factor "Industrial", although the exact type of industry is not yet
known. This factor also barely contributes to the PM\textsubscript{10}
mass, with a mean concentration of around 1. The intrinsic
OP\textsuperscript{DTT} is high for GRE-cb and GRE-fr\_2017 (0.52(30)
and 0.37(27), respectively) as well as the intrinsic
OP\textsuperscript{AA} (0.82(29) and 0.61(17) , respectively), but is
close to 0 for the other sites and both OP. The high intrinsic OP seems
to indicate again the role of metals in the OP of PM, however since this
factor has strong uncertainties associated with the PMF results, and
then to the intrinsic OP, is hard to go into further conclusion at this
point.

\hypertarget{limitations-of-the-study}{%
\section{Limitations of the study}\label{limitations-of-the-study}}

In this study, we focused on major sources and trends, hence limit our
study to some aspect. Notably the PMF standardized approach allows
common source identification at the national scale but may also
``polish'' some site specificity. Also, the choice to focus on the main
sources of PM and to discuss the aggregated results allows discussion on
some local specificity, notably potential local sources that contribute
to the OP (for instance HFO or industry).

One main limitation is also the use of linear regression tools whereas
it has been shown that OP does not respond linearly with mass. The
residual analysis seems to agree with this experimental finding since
the highest OP samples is underestimated by the MLR model. The addition
of co-variation term or even the use of non-linear regression may be the
next step to better explain the OP of the sources and provide more
accurate policy for air quality.

Moreover, if the intrinsic OP results from the MLR can be extrapolated
to any given site with similar regional background of the urbanized area
used in this study, the source' contributions extrapolation should be
taken cautiously since our dataset displays an over-representation of
the alpine sites with regard to the whole France.

Conclusion

To our knowledge, this study gathers the most important database of OP
samples with concomitant observations of chemistry analysis,
source-apportionment through PMF and the measure of two OP assays (DTT
and AA) for 15 yearly time series over France spanning between 2012 to
2016 for a total of \textgreater1700 samples.

\begin{itemize}
\item
  We demonstrate that source apportionment of OP through a ``simple''
  multi linear regression without any constraint on the coefficient show
  good statistical result and is able to explain the observed
  OP\textsubscript{v}\textsuperscript{DTT} and
  OP\textsubscript{v}\textsuperscript{AA}.
\item
  The associated uncertainties of the MLR coefficient estimated thanks
  to bootstrapping the solution highlight the importance of the yearly
  representativeness of the input dataset.
\end{itemize}

\begin{itemize}
\item
  The intrinsic OP of the main regional sources presents values that are
  in the same range at each site, especially for the primary traffic,
  biomass burning, nitrate rich, dust \& sulfate rich PMF factor.
  Biogenic and MSA rich factors present higher discrepancy according to
  the site but also present the highest uncertainties at each site.
\item
  Different sensitivity for the 2 OP assays toward the given source is
  highlighted. The DTT appears to be the most balanced test, whereas the
  AA targets mainly the biomass burning and primary traffic factor.
\item
  In accordance with previous studies the primary road traffic and
  biomass burning factor has been shown as the main absolute OP
  contributors. On the opposite, the secondary inorganic sources
  (nitrate and sulfate rich) barely contribute to OP.
\item
  In order to assess the chronicle population exposure, the median
  contribution of sources to the
  OP\textsubscript{v}\textsuperscript{DTT} and
  OP\textsubscript{v}\textsuperscript{AA} are also reported and present
  important discrepancy with regard to the mean contribution. The
  importance of the road traffic source drastically increases, notably
  for the OP\textsubscript{v}\textsuperscript{DTT}, whereas the biomass
  burning contribution is lowered. Barely only the road traffic and
  biomass burning sources contribute to the daily median of the
  OP\textsubscript{v}\textsuperscript{AA}. Moreover, since there is no
  consensus toward the ``best'' OP assay for epidemiological or
  toxicological assessment, the sources that react to both OP should be
  of the first importance for regulation purposes. Then, the biomass
  burning and road traffic sources could be targeted to decrease the
  regional OP exposure.
\item
  Finally, the relatively stable intrinsic OP at a large geographical
  scale for the main PM sources paves the road to the implementation of
  the OP into regional chemistry transport model. This step would allow
  a quantitative estimation of the french population exposure OP,
  expending potential cross-over studies with epidemiology (today
  limited to this dataset sites)
\end{itemize}

Fundings

This work was partially funded by ANSES for OP measurements (ExPOSURE
program, grant 2016-CRD-31), IDEX UGA grant for innovation 2017
ROS-ONLINE and CDP IDEX UGA MOBILAIR (ANR-15-IDEX-02) Atmo Sud for the
chemical analyses of the samples from NIC, MRS-5av and PdB, Atmo AuRA
for the chemical analysis of the samples from GRE-fr, GRE-cb, VIF, CHAM,
PAS and MNZ, and the French Ministry of Environment, as part of the
National reference laboratory for air quality (LCSQA), for the samples
from GRE-fr, TAL and NIC. The study in CHAM, MNZ and PAS was funded by
ADEME. The PhD of Samuël Weber was funded by a grant from ENS Paris.
This study was also supported by direct funding by IGE (technician
salary), the LEFE CHAT (program 863353: ``Le PO comme proxy de l'impact
sanitaire''), and LABEX OSUG@2020 (ANR-10-LABX-56) (both for funding
analytical instruments).

The authors wish to thanks all the peoples from the different AASQA or
laboratories who gathered the filter and analyzed them (long list of
name).

Abrams, J. Y., Weber, R. J., Klein, M., Samat, S. E., Chang, H. H.,
Strickland, M. J., Verma, V., Fang, T., Bates, J. T., Mulholland, J. A.,
Russell, A. G. and Tolbert, P. E.: Associations between Ambient Fine
Particulate Oxidative Potential and Cardiorespiratory Emergency
Department Visits, Environmental Health Perspectives, 125(10),
doi:\href{https://doi.org/10.1289/EHP1545}{10.1289/EHP1545}, 2017.

Alleman, L. Y., Lamaison, L., Perdrix, E., Robache, A. and Galloo,
J.-C.: PM\textsubscript{10} metal concentrations and source
identification using positive matrix factorization and wind sectoring in
a French industrial zone, Atmospheric Research, 96(4), 612--625,
doi:\href{https://doi.org/10.1016/j.atmosres.2010.02.008}{10.1016/j.atmosres.2010.02.008},
2010.

Amato, F., Alastuey, A., Karanasiou, A., Lucarelli, F., Nava, S.,
Calzolai, G., Severi, M., Becagli, S., Gianelle, V. L., Colombi, C.,
Alves, C., Custódio, D., Nunes, T., Cerqueira, M., Pio, C.,
Eleftheriadis, K., Diapouli, E., Reche, C., Minguillón, M. C.,
Manousakas, M.-I., Maggos, T., Vratolis, S., Harrison, R. M. and Querol,
X.: AIRUSE-LIFE+: A harmonized PM speciation and source apportionment in
five southern European cities, Atmospheric Chemistry and Physics, 16(5),
3289--3309,
doi:\href{https://doi.org/10.5194/acp-16-3289-2016}{10.5194/acp-16-3289-2016},
2016.

Atkinson, R. W., Samoli, E., Analitis, A., Fuller, G. W., Green, D. C.,
Anderson, H. R., Purdie, E., Dunster, C., Aitlhadj, L., Kelly, F. J. and
Mudway, I. S.: Short-term associations between particle oxidative
potential and daily mortality and hospital admissions in London,
International Journal of Hygiene and Environmental Health, 219(6),
566--572,
doi:\href{https://doi.org/10.1016/j.ijheh.2016.06.004}{10.1016/j.ijheh.2016.06.004},
2016.

Ayres, J. G., Borm, P., Cassee, F. R., Castranova, V., Donaldson, K.,
Ghio, A., Harrison, R. M., Hider, R., Kelly, F., Kooter, I. M., Marano,
F., Maynard, R. L., Mudway, I., Nel, A., Sioutas, C., Smith, S.,
Baeza-Squiban, A., Cho, A., Duggan, S. and Froines, J.: Evaluating the
Toxicity of Airborne Particulate Matter and Nanoparticles by Measuring
Oxidative Stress PotentialA Workshop Report and Consensus Statement,
Inhalation Toxicology, 20(1), 75--99,
doi:\href{https://doi.org/10.1080/08958370701665517}{10.1080/08958370701665517},
2008.

Barraza-Villarreal, A., Sunyer, J., Hernandez-Cadena, L.,
Escamilla-Nuñez, M. C., Sienra-Monge, J. J., Ramírez-Aguilar, M.,
Cortez-Lugo, M., Holguin, F., Diaz-Sánchez, D., Olin, A. C. and Romieu,
I.: Air Pollution, Airway Inflammation, and Lung Function in a Cohort
Study of Mexico City Schoolchildren, Environmental Health Perspectives,
116(6), 832--838,
doi:\href{https://doi.org/10.1289/ehp.10926}{10.1289/ehp.10926}, 2008.

Bates, J. T., Weber, R. J., Abrams, J., Verma, V., Fang, T., Klein, M.,
Strickland, M. J., Sarnat, S. E., Chang, H. H., Mulholland, J. A.,
Tolbert, P. E. and Russell, A. G.: Reactive oxygen species generation
linked to sources of atmospheric particulate matter and
cardiorespiratory effects, Environmental Science \& Technology, 49(22),
13605--13612,
doi:\href{https://doi.org/10.1021/acs.est.5b02967}{10.1021/acs.est.5b02967},
2015.

Bates, J. T., Fang, T., Verma, V., Zeng, L., Weber, R. J., Tolbert, P.
E., Abrams, J. Y., Sarnat, S. E., Klein, M., Mulholland, J. A. and
Russell, A. G.: Review of Acellular Assays of Ambient Particulate Matter
Oxidative Potential: Methods and Relationships with Composition,
Sources, and Health Effects, Environmental Science \& Technology, 53(8),
4003--4019,
doi:\href{https://doi.org/10.1021/acs.est.8b03430}{10.1021/acs.est.8b03430},
2019.

Beck-Speier, I., Karg, E., Behrendt, H., Stoeger, T. and Alessandrini,
F.: Ultrafine particles affect the balance of endogenous pro- and
anti-inflammatory lipid mediators in the lung: In-vitro and in-vivo
studies, Particle and Fibre Toxicology, 9(1), 27,
doi:\href{https://doi.org/10.1186/1743-8977-9-27}{10.1186/1743-8977-9-27},
2012.

Belis, C. A., Pernigotti, D., Karagulian, F., Pirovano, G., Larsen, B.
R., Gerboles, M. and Hopke, P. K.: A new methodology to assess the
performance and uncertainty of source apportionment models in
intercomparison exercises, Atmospheric Environment, 119, 35--44,
doi:\href{https://doi.org/10.1016/j.atmosenv.2015.08.002}{10.1016/j.atmosenv.2015.08.002},
2015.

Belis, C. A., Favez, O., Mircea, M., Diapouli, E., Manousakas, M.-I.,
Vratolis, S., Gilardoni, S., Paglione, M., Decesari, S., Mocnik, G.,
Mooibroek, D., Salvador, P., Takahama, S., Vecchi, R., Paatero, P.,
European Commission and Joint Research Centre: European guide on air
pollution source apportionment with receptor models: Revised version
2019., 2019.

Belis, C. A., Pernigotti, D., Pirovano, G., Favez, O., Jaffrezo, J. L.,
Kuenen, J., Denier van Der Gon, H., Reizer, M., Riffault, V., Alleman,
L. Y., Almeida, M., Amato, F., Angyal, A., Argyropoulos, G., Bande, S.,
Beslic, I., Besombes, J. L., Bove, M. C., Brotto, P., Calori, G.,
Cesari, D., Colombi, C., Contini, D., De Gennaro, G., Di Gilio, A.,
Diapouli, E., El Haddad, I., Elbern, H., Eleftheriadis, K., Ferreira,
J., Vivanco, M. G., Gilardoni, S., Golly, B., Hellebust, S., Hopke, P.
K., Izadmanesh, Y., Jorquera, H., Krajsek, K., Kranenburg, R., Lazzeri,
P., Lenartz, F., Lucarelli, F., Maciejewska, K., Manders, A.,
Manousakas, M., Masiol, M., Mircea, M., Mooibroek, D., Nava, S.,
Oliveira, D., Paglione, M., Pandolfi, M., Perrone, M., Petralia, E.,
Pietrodangelo, A., Pillon, S., Pokorna, P., Prati, P., Salameh, D.,
Samara, C., Samek, L., Saraga, D., Sauvage, S., Schaap, M., Scotto, F.,
Sega, K., Siour, G., Tauler, R., Valli, G., Vecchi, R., Venturini, E.,
Vestenius, M., Waked, A. and Yubero, E.: Evaluation of receptor and
chemical transport models for PM10 source apportionment, Atmospheric
Environment: X, 5, 100053,
doi:\href{https://doi.org/10.1016/j.aeaoa.2019.100053}{10.1016/j.aeaoa.2019.100053},
2020.

Birch, M. E. and Cary, R. A.: Elemental Carbon-Based Method for
Monitoring Occupational Exposures to Particulate Diesel Exhaust, Aerosol
Science and Technology, 25(3), 221--241,
doi:\href{https://doi.org/10.1080/02786829608965393}{10.1080/02786829608965393},
1996.

Borlaza, L. J. S., Weber, S., Asslanian, K., Uzu, G., Jacob, V., Cañete,
T., Slama, R., Favez, O., Albinet, A., Guillaud, G., Thomasson, A. and
Jaffrezo, J.-L.: Source apportionment of PM10 using new organic tracers
at multiple urban sites in the Grenoble basin, n.d.

Bozzetti, C., Sosedova, Y., Xiao, M., Daellenbach, K. R., Ulevicius, V.,
Dudoitis, V., Mordas, G., Byčenkienė, S., Plauškaitė, K., Vlachou, A.,
Golly, B., Chazeau, B., Besombes, J.-L., Baltensperger, U., Jaffrezo,
J.-L., Slowik, J. G., Haddad, I. E. and Prévôt, A. S. H.: Argon
offline-AMS source apportionment of organic aerosol over yearly cycles
for an urban, rural, and marine site in northern Europe, Atmospheric
Chemistry and Physics, 17(1), 117--141,
doi:\href{https://doi.org/https://doi.org/10.5194/acp-17-117-2017}{https://doi.org/10.5194/acp-17-117-2017},
2017.

Brandt, J., Silver, J. D., Christensen, J. H., Andersen, M. S.,
Bønløkke, J. H., Sigsgaard, T., Geels, C., Gross, A., Hansen, A. B.,
Hansen, K. M., Hedegaard, G. B., Kaas, E. and Frohn, L. M.: Contribution
from the ten major emission sectors in Europe and Denmark to the
health-cost externalities of air pollution using the EVA model system an
integrated modelling approach, Atmospheric Chemistry and Physics,
13(15), 7725--7746,
doi:\href{https://doi.org/https://doi.org/10.5194/acp-13-7725-2013}{https://doi.org/10.5194/acp-13-7725-2013},
2013.

Brauer, M., Amann, M., Burnett, R. T., Cohen, A., Dentener, F., Ezzati,
M., Henderson, S. B., Krzyzanowski, M., Martin, R. V., Van Dingenen, R.,
van Donkelaar, A. and Thurston, G. D.: Exposure Assessment for
Estimation of the Global Burden of Disease Attributable to Outdoor Air
Pollution, Environmental Science \& Technology, 46(2), 652--660,
doi:\href{https://doi.org/10.1021/es2025752}{10.1021/es2025752}, 2012.

Bressi, M., Sciare, J., Ghersi, V., Mihalopoulos, N., Petit, J.-E.,
Nicolas, J. B., Moukhtar, S., Rosso, A., Féron, A., Bonnaire, N.,
Poulakis, E. and Theodosi, C.: Sources and geographical origins of fine
aerosols in Paris (France), Atmospheric Chemistry and Physics, 14(16),
8813--8839,
doi:\href{https://doi.org/10.5194/acp-14-8813-2014}{10.5194/acp-14-8813-2014},
2014.

Calas, A.: Pollution atmosphérique particulaire : Développement de
méthodologies non-invasives et acellulaires pour l'évaluation de
l'impact sanitaire, PhD thesis., 2017.

Calas, A., Uzu, G., Kelly, F. J., Houdier, S., Martins, J. M. F.,
Thomas, F., Molton, F., Charron, A., Dunster, C., Oliete, A., Jacob, V.,
Besombes, J.-L., Chevrier, F. and Jaffrezo, J.-L.: Comparison between
five acellular oxidative potential measurement assays performed with
detailed chemistry on PM\textsubscript{10} samples from the city of
Chamonix (France), Atmospheric Chemistry and Physics, 18(11),
7863--7875,
doi:\href{https://doi.org/10.5194/acp-18-7863-2018}{10.5194/acp-18-7863-2018},
2018.

Calas, A., Uzu, G., Besombes, J.-L., Martins, J. M. F., Redaelli, M.,
Weber, S., Charron, A., Albinet, A., Chevrier, F., Brulfert, G., Mesbah,
B., Favez, O. and Jaffrezo, J.-L.: Seasonal Variations and Chemical
Predictors of Oxidative Potential (OP) of Particulate Matter (PM), for
Seven Urban French Sites, Atmosphere, 10(11), 698,
doi:\href{https://doi.org/10.3390/atmos10110698}{10.3390/atmos10110698},
2019.

Canova, C., Minelli, C., Dunster, C., Kelly, F., Shah, P. L., Caneja,
C., Tumilty, M. K. and Burney, P.: PM10 Oxidative Properties and Asthma
and COPD, Epidemiology, 25(3), 467--468,
doi:\href{https://doi.org/10.1097/EDE.0000000000000084}{10.1097/EDE.0000000000000084},
2014.

Cavalli, F., Viana, M., Yttri, K. E., Genberg, J. and Putaud, J.-P.:
Toward a standardised thermal-optical protocol for measuring atmospheric
organic and elemental carbon: The EUSAAR protocol, Atmospheric
Measurement Techniques, 3(1), 79--89,
doi:\href{https://doi.org/https://doi.org/10.5194/amt-3-79-2010}{https://doi.org/10.5194/amt-3-79-2010},
2010.

CEN: Ambient air quality - Standard method for the measurement of Pb,
Cd, As and Ni in the PM\textsubscript{10} fraction of suspended
particulate matter, Technical Report, CEN, Brussels, Belgium., 2005.

CEN: Ambient air - Standard gravimetric measurement method for the
determination of the PM\textsubscript{10} or PM\textsubscript{2.5} mass
concentration of suspended particulate matter, Technical Report, CEN,
Brussels, Belgium., 2014.

CEN: Ambient air - Automated measuring systems for the measurement of
the concentration of particulate matter (PM\textsubscript{10};
PM\textsubscript{2.5}), Technical Report, CEN, Brussels, Belgium.,
2017a.

CEN: Ambient air - Measurement of elemental carbon (EC) and organic
carbon (OC) collected on filters, Technical Report, CEN, Brussels,
Belgium., 2017b.

CEN: Ambient air - Standard method for measurement of
NO\textsubscript{3}\textsuperscript{-},
SO\textsubscript{4}\textsuperscript{2-}, Cl\textsuperscript{-},
NH\textsubscript{4}\textsuperscript{+}, Na\textsuperscript{+},
K\textsuperscript{+}, Mg\textsuperscript{2+}, Ca\textsuperscript{2+} in
PM\textsubscript{2.5} as deposited on filters, Technical Report, CEN,
Brussels, Belgium., 2017c.

Cesari, D., Merico, E., Grasso, F. M., Decesari, S., Belosi, F.,
Manarini, F., De Nuntiis, P., Rinaldi, M., Volpi, F., Gambaro, A.,
Morabito, E. and Contini, D.: Source Apportionment of PM2.5 and of its
Oxidative Potential in an Industrial Suburban Site in South Italy,
Atmosphere, 10(12), 758,
doi:\href{https://doi.org/10.3390/atmos10120758}{10.3390/atmos10120758},
2019.

Charrier, J. G. and Anastasio, C.: On dithiothreitol (DTT) as a measure
of oxidative potential for ambient particles: Evidence for the
importance of soluble transition metals, Atmospheric Chemistry and
Physics Discussions, 12(5), 11317--11350,
doi:\href{https://doi.org/10.5194/acpd-12-11317-2012}{10.5194/acpd-12-11317-2012},
2012.

Charrier, J. G., McFall, A. S., Vu, K. K.-T., Baroi, J., Olea, C.,
Hasson, A. and Anastasio, C.: A bias in the ``mass-normalized'' DTT
response An effect of non-linear concentration-response curves for
copper and manganese, Atmospheric Environment, 144, 325--334,
doi:\href{https://doi.org/10.1016/j.atmosenv.2016.08.071}{10.1016/j.atmosenv.2016.08.071},
2016.

Chevrier, F.: Chauffage au bois et qualité de l'air en Vallée de l'Arve
: définition d'un système de surveillance et impact d'une politique de
rénovation du parc des appareils anciens., PhD thesis, Université
Grenoble Alpes, Grenoble., 2016.

Chevrier, F., Ježek, I., Brulfert, G., MOčnik, G., Marchand, N.,
Jaffrezo, J.-L. and Besombes, J.-L.: DECOMBIO-Contribution de la
combustion de la biomasse aux PM\textsubscript{10} en vallée de l'Arve:
Mise en place et qualification d'un dispositif de suivi, 2268-3798,
2016.

Cho, A. K., Sioutas, C., Miguel, A. H., Kumagai, Y., Schmitz, D. A.,
Singh, M., Eiguren-Fernandez, A. and Froines, J. R.: Redox activity of
airborne particulate matter at different sites in the Los Angeles Basin,
Environmental Research, 99(1), 40--47,
doi:\href{https://doi.org/10.1016/j.envres.2005.01.003}{10.1016/j.envres.2005.01.003},
2005.

Costabile, F., Alas, H., Aufderheide, M., Avino, P., Amato, F.,
Argentini, S., Barnaba, F., Berico, M., Bernardoni, V., Biondi, R.,
Casasanta, G., Ciampichetti, S., Calzolai, G., Canepari, S., Conidi, A.,
Cordelli, E., Di Ianni, A., Di Liberto, L., Facchini, M. C., Facci, A.,
Frasca, D., Gilardoni, S., Grollino, M. G., Gualtieri, M., Lucarelli,
F., Malaguti, A., Manigrasso, M., Montagnoli, M., Nava, S., Perrino, C.,
Padoan, E., Petenko, I., Querol, X., Simonetti, G., Tranfo, G.,
Ubertini, S., Valli, G., Valentini, S., Vecchi, R., Volpi, F., Weinhold,
K., Wiedensohler, A., Zanini, G., Gobbi, G. P. and Petralia, E.: First
Results of the ``Carbonaceous Aerosol in Rome and Environs (CARE)''
Experiment: Beyond Current Standards for PM\textsubscript{10},
Atmosphere, 8(12), 249,
doi:\href{https://doi.org/10.3390/atmos8120249}{10.3390/atmos8120249},
2017.

Costabile, F., Gualtieri, M., Canepari, S., Tranfo, G., Consales, C.,
Grollino, M. G., Paci, E., Petralia, E., Pigini, D. and Simonetti, G.:
Evidence of association between aerosol properties and in-vitro cellular
oxidative response to PM1, oxidative potential of PM2.5, a biomarker of
RNA oxidation, and its dependency on combustion sources, Atmospheric
Environment, 213, 444--455,
doi:\href{https://doi.org/10.1016/j.atmosenv.2019.06.023}{10.1016/j.atmosenv.2019.06.023},
2019.

Diémoz, H., Barnaba, F., Magri, T., Pession, G., Dionisi, D., Pittavino,
S., Tombolato, I. K. F., Campanelli, M., Ceca, L. S. D., Hervo, M.,
Liberto, L. D., Ferrero, L. and Gobbi, G. P.: Transport of Po Valley
aerosol pollution to the northwestern Alps Part 1: Phenomenology,
Atmospheric Chemistry and Physics, 19(5), 3065--3095,
doi:\href{https://doi.org/https://doi.org/10.5194/acp-19-3065-2019}{https://doi.org/10.5194/acp-19-3065-2019},
2019.

El Haddad, I., Marchand, N., Temime-Roussel, B., Wortham, H., Piot, C.,
Besombes, J.-L., Baduel, C., Voisin, D., Armengaud, A. and Jaffrezo,
J.-L.: Insights into the secondary fraction of the organic aerosol in a
Mediterranean urban area: Marseille, Atmospheric Chemistry and Physics,
11(5), 2059--2079,
doi:\href{https://doi.org/https://doi.org/10.5194/acp-11-2059-2011}{https://doi.org/10.5194/acp-11-2059-2011},
2011.

Fang, T., Guo, H., Verma, V., Peltier, R. E. and Weber, R. J.:
PM\textsubscript{2.5} water-soluble elements in the southeastern United
States: Automated analytical method development, spatiotemporal
distributions, source apportionment, and implications for heath studies,
Atmospheric Chemistry and Physics, 15(20), 11667--11682,
doi:\href{https://doi.org/10.5194/acp-15-11667-2015}{10.5194/acp-15-11667-2015},
2015.

Fang, T., Verma, V., Bates, J. T., Abrams, J., Klein, M., Strickland, M.
J., Sarnat, S. E., Chang, H. H., Mulholland, J. A., Tolbert, P. E.,
Russell, A. G. and Weber, R. J.: Oxidative potential of ambient
water-soluble PM \textsubscript{2.5} in the southeastern United States:
Contrasts in sources and health associations between ascorbic acid (AA)
and dithiothreitol (DTT) assays, Atmospheric Chemistry and Physics,
16(6), 3865--3879,
doi:\href{https://doi.org/10.5194/acp-16-3865-2016}{10.5194/acp-16-3865-2016},
2016.

Favez, O., Salameh, D. and Jaffrezo, J.-L.: Traitement harmonisé de jeux
de données multi-sites pour l'étude de sources de PM par Positive Matrix
Factorization (PMF), LCSQA, Verneuil-en-Halatte, France., 2017.

Gianini, M. F. D., Fischer, A., Gehrig, R., Ulrich, A., Wichser, A.,
Piot, C., Besombes, J.-L. and Hueglin, C.: Comparative source
apportionment of PM\textsubscript{10} in Switzerland for 2008/2009 and
1998/1999 by Positive Matrix Factorisation, Atmospheric Environment, 54,
149--158,
doi:\href{https://doi.org/10.1016/j.atmosenv.2012.02.036}{10.1016/j.atmosenv.2012.02.036},
2012.

Goix, S., Lévêque, T., Xiong, T.-T., Schreck, E., Baeza-Squiban, A.,
Geret, F., Uzu, G., Austruy, A. and Dumat, C.: Environmental and health
impacts of fine and ultrafine metallic particles: Assessment of threat
scores, Environmental Research, 133, 185--194,
doi:\href{https://doi.org/10.1016/j.envres.2014.05.015}{10.1016/j.envres.2014.05.015},
2014.

Goldberg, M.: A Systematic Review of the Relation Between Long-term
Exposure to Ambient Air Pollution and Chronic Diseases, Reviews on
Environmental Health, 23(4), 243--298,
doi:\href{https://doi.org/10.1515/REVEH.2008.23.4.243}{10.1515/REVEH.2008.23.4.243},
2011.

Golly, B., Waked, A., Weber, S., Samake, A., Jacob, V., Conil, S.,
Rangognio, J., Chrétien, E., Vagnot, M. P., Robic, P. Y., Besombes,
J.-L. and Jaffrezo, J.-L.: Organic markers and OC source apportionment
for seasonal variations of PM\textsubscript{2.5} at 5 rural sites in
France, Atmospheric Environment, 198, 142--157,
doi:\href{https://doi.org/10.1016/j.atmosenv.2018.10.027}{10.1016/j.atmosenv.2018.10.027},
2019.

Hodshire, A. L., Campuzano-Jost, P., Kodros, J. K., Croft, B., Nault, B.
A., Schroder, J. C., Jimenez, J. L. and Pierce, J. R.: The potential
role of methanesulfonic acid (MSA) in aerosol formation and growth and
the associated radiative forcings, Atmospheric Chemistry and Physics,
19(5), 3137--3160,
doi:\href{https://doi.org/https://doi.org/10.5194/acp-19-3137-2019}{https://doi.org/10.5194/acp-19-3137-2019},
2019.

Hu, S., Polidori, A., Arhami, M., Shafer, M. M., Schauer, J. J., Cho, A.
and Sioutas, C.: Redox activity and chemical speciation of size
fractioned PM in the communities of the Los Angeles-Long Beach harbor,
Atmos. Chem. Phys., 13, 2008.

Jaffrezo, J.-L., Aymoz, G., Delaval, C. and Cozic, J.: Seasonal
variations of the water soluble organic carbon mass fraction of aerosol
in two valleys of the French Alps, Atmospheric Chemistry and Physics,
5(10), 2809--2821, 2005.

Jain, S., Sharma, S. K., Srivastava, M. K., Chaterjee, A., Singh, R. K.,
Saxena, M. and Mandal, T. K.: Source Apportionment of
PM\textsubscript{10} Over Three Tropical Urban Atmospheres at
Indo-Gangetic Plain of India: An Approach Using Different Receptor
Models, Archives of Environmental Contamination and Toxicology,
doi:\href{https://doi.org/10.1007/s00244-018-0572-4}{10.1007/s00244-018-0572-4},
2018.

Janssen, N. A. H., Yang, A., Strak, M., Steenhof, M., Hellack, B.,
Gerlofs-Nijland, M. E., Kuhlbusch, T., Kelly, F., Harrison, R.,
Brunekreef, B., Hoek, G. and Cassee, F.: Oxidative potential of
particulate matter collected at sites with different source
characteristics, Science of The Total Environment, 472, 572--581,
doi:\href{https://doi.org/10.1016/j.scitotenv.2013.11.099}{10.1016/j.scitotenv.2013.11.099},
2014.

Janssen, N. A. H., Strak, M., Yang, A., Hellack, B., Kelly, F. J.,
Kuhlbusch, T. A. J., Harrison, R. M., Brunekreef, B., Cassee, F. R.,
Steenhof, M. and Hoek, G.: Associations between three specific
a-cellular measures of the oxidative potential of particulate matter and
markers of acute airway and nasal inflammation in healthy volunteers,
Occupational and Environmental Medicine, 72(1), 49--56,
doi:\href{https://doi.org/10.1136/oemed-2014-102303}{10.1136/oemed-2014-102303},
2015.

Jiang, H. and Jang, M.: Dynamic Oxidative Potential of Atmospheric
Organic Aerosol under Ambient Sunlight, Environmental Science \&
Technology, 52(13), 7496--7504,
doi:\href{https://doi.org/10.1021/acs.est.8b00148}{10.1021/acs.est.8b00148},
2018.

Jiang, J., Aksoyoglu, S., El-Haddad, I., Ciarelli, G., Denier van der
Gon, H. A. C., Canonaco, F., Gilardoni, S., Paglione, M., Minguillón, M.
C., Favez, O., Zhang, Y., Marchand, N., Hao, L., Virtanen, A., Florou,
K., O'Dowd, C., Ovadnevaite, J., Baltensperger, U. and Prévôt, A. S. H.:
Sources of organic aerosols in Europe: A modelling study using CAMx with
modified volatility basis set scheme, Atmospheric Chemistry and Physics
Discussions, 1--35,
doi:\href{https://doi.org/https://doi.org/10.5194/acp-2019-468}{https://doi.org/10.5194/acp-2019-468},
2019.

Karavalakis, G., Gysel, N., Schmitz, D. A., Cho, A. K., Sioutas, C.,
Schauer, J. J., Cocker, D. R. and Durbin, T. D.: Impact of biodiesel on
regulated and unregulated emissions, and redox and proinflammatory
properties of PM emitted from heavy-duty vehicles, The Science of the
Total Environment, 584-585, 1230--1238,
doi:\href{https://doi.org/10.1016/j.scitotenv.2017.01.187}{10.1016/j.scitotenv.2017.01.187},
2017.

Kelly, F. J. and Fussell, J. C.: Size, source and chemical composition
as determinants of toxicity attributable to ambient particulate matter,
Atmospheric Environment, 60, 504--526,
doi:\href{https://doi.org/10.1016/j.atmosenv.2012.06.039}{10.1016/j.atmosenv.2012.06.039},
2012.

Kelly, F. J. and Mudway, I. S.: Protein oxidation at the air-lung
interface, Amino Acids, 25(3-4), 375--396,
doi:\href{https://doi.org/10.1007/s00726-003-0024-x}{10.1007/s00726-003-0024-x},
2003.

Kranenburg, R., Segers, A. J., Hendriks, C. and Schaap, M.: Source
apportionment using LOTOS-EUROS: Module description and evaluation,
Geoscientific Model Development, 6(3), 721--733,
doi:\href{https://doi.org/https://doi.org/10.5194/gmd-6-721-2013}{https://doi.org/10.5194/gmd-6-721-2013},
2013.

Künzli, N., Mudway, I. S., Götschi, T., Shi, T., Kelly, F. J., Cook, S.,
Burney, P., Forsberg, B., Gauderman, J. W., Hazenkamp, M. E., Heinrich,
J., Jarvis, D., Norbäck, D., Payo-Losa, F., Poli, A., Sunyer, J. and
Borm, P. J. A.: Comparison of Oxidative Properties, Light Absorbance,
and Total and Elemental Mass Concentration of Ambient PM
\textsubscript{2.5} Collected at 20 European Sites, Environmental Health
Perspectives, 114(5), 684--690,
doi:\href{https://doi.org/10.1289/ehp.8584}{10.1289/ehp.8584}, 2006.

Lelieveld, J., Evans, J. S., Fnais, M., Giannadaki, D. and Pozzer, A.:
The contribution of outdoor air pollution sources to premature mortality
on a global scale, Nature, 525(7569), 367--371,
doi:\href{https://doi.org/10.1038/nature15371}{10.1038/nature15371},
2015.

Li, N., Hao, M., Phalen, R. F., Hinds, W. C. and Nel, A. E.: Particulate
air pollutants and asthma: A paradigm for the role of oxidative stress
in PM-induced adverse health effects, Clinical Immunology, 109(3),
250--265,
doi:\href{https://doi.org/10.1016/j.clim.2003.08.006}{10.1016/j.clim.2003.08.006},
2003.

Li, Q., Wyatt, A. and Kamens, R. M.: Oxidant generation and toxicity
enhancement of aged-diesel exhaust, Atmospheric Environment, 43(5),
1037--1042,
doi:\href{https://doi.org/10.1016/j.atmosenv.2008.11.018}{10.1016/j.atmosenv.2008.11.018},
2009.

Liu, Q., Baumgartner, J., Zhang, Y. and Schauer, J. J.: Source
apportionment of Beijing air pollution during a severe winter haze event
and associated pro-inflammatory responses in lung epithelial cells,
Atmospheric Environment, 126, 28--35,
doi:\href{https://doi.org/10.1016/j.atmosenv.2015.11.031}{10.1016/j.atmosenv.2015.11.031},
2016.

Ma, Y., Cheng, Y., Qiu, X., Cao, G., Fang, Y., Wang, J., Zhu, T., Yu, J.
and Hu, D.: Sources and oxidative potential of water-soluble humic-like
substances (HULIS\textsubscript{WS}) in fine particulate matter
(PM\textsubscript{2.5}) in Beijing, Atmospheric Chemistry and Physics,
18(8), 5607--5617,
doi:\href{https://doi.org/https://doi.org/10.5194/acp-18-5607-2018}{https://doi.org/10.5194/acp-18-5607-2018},
2018.

Marconi, M., Sferlazzo, D. M., Becagli, S., Bommarito, C., Calzolai, G.,
Chiari, M., di Sarra, A., Ghedini, C., Gómez-Amo, J. L., Lucarelli, F.,
Meloni, D., Monteleone, F., Nava, S., Pace, G., Piacentino, S., Rugi,
F., Severi, M., Traversi, R. and Udisti, R.: Saharan dust aerosol over
the central Mediterranean Sea: PM\textsubscript{10} chemical composition
and concentration versus optical columnar measurements, Atmospheric
Chemistry and Physics, 14(4), 2039--2054,
doi:\href{https://doi.org/10.5194/acp-14-2039-2014}{10.5194/acp-14-2039-2014},
2014.

Mbengue, S., Alleman, L. Y. and Flament, P.: Size-distributed metallic
elements in submicronic and ultrafine atmospheric particles from urban
and industrial areas in northern France, Atmospheric Research, 135-136,
35--47,
doi:\href{https://doi.org/10.1016/j.atmosres.2013.08.010}{10.1016/j.atmosres.2013.08.010},
2014.

McWhinney, R. D., Zhou, S. and Abbatt, J. P. D.: Naphthalene SOA: Redox
activity and naphthoquinone gasParticle partitioning, Atmospheric
Chemistry and Physics, 13(19), 9731--9744,
doi:\href{https://doi.org/10.5194/acp-13-9731-2013}{10.5194/acp-13-9731-2013},
2013.

Mircea, M., Calori, G., Pirovano, G. and Belis, C. A.: European guide on
air pollution source apportionment for particulate matter with source
oriented models and their combined use with receptor models,
Publications Office of the European Union, Luxembourg., 2020.

Moreno, T., Querol, X., Alastuey, A., de la Rosa, J., Sánchez de la
Campa, A. M., Minguillón, M., Pandolfi, M., González-Castanedo, Y.,
Monfort, E. and Gibbons, W.: Variations in vanadium, nickel and
lanthanoid element concentrations in urban air, Science of The Total
Environment, 408(20), 4569--4579,
doi:\href{https://doi.org/10.1016/j.scitotenv.2010.06.016}{10.1016/j.scitotenv.2010.06.016},
2010.

Ntziachristos, L., Froines, J. R., Cho, A. K. and Sioutas, C.:
Relationship between redox activity and chemical speciation of
size-fractionated particulate matter, Particle and Fibre Toxicology,
4(1), 5,
doi:\href{https://doi.org/10.1186/1743-8977-4-5}{10.1186/1743-8977-4-5},
2007.

Paatero, P.: The Multilinear Engine: A Table-Driven, Least Squares
Program for Solving Multilinear Problems, including the n-Way Parallel
Factor Analysis Model, Journal of Computational and Graphical
Statistics, 8(4), 854,
doi:\href{https://doi.org/10.2307/1390831}{10.2307/1390831}, 1999.

Paraskevopoulou, D., Bougiatioti, A., Stavroulas, I., Fang, T., Lianou,
M., Liakakou, E., Gerasopoulos, E., Weber, R., Nenes, A. and
Mihalopoulos, N.: Yearlong variability of oxidative potential of
particulate matter in an urban Mediterranean environment, Atmospheric
Environment, 206, 183--196,
doi:\href{https://doi.org/10.1016/j.atmosenv.2019.02.027}{10.1016/j.atmosenv.2019.02.027},
2019.

Pernigotti, D. and Belis, C. A.: DeltaSA tool for source apportionment
benchmarking, description and sensitivity analysis, Atmospheric
Environment, 180, 138--148,
doi:\href{https://doi.org/10.1016/j.atmosenv.2018.02.046}{10.1016/j.atmosenv.2018.02.046},
2018.

Pernigotti, D., Belis, C. A. and Spanò, L.: SPECIEUROPE: The European
data base for PM source profiles, Atmospheric Pollution Research, 7(2),
307--314,
doi:\href{https://doi.org/10.1016/j.apr.2015.10.007}{10.1016/j.apr.2015.10.007},
2016.

Perrone, M. G., Zhou, J., Malandrino, M., Sangiorgi, G., Rizzi, C.,
Ferrero, L., Dommen, J. and Bolzacchini, E.: PM chemical composition and
oxidative potential of the soluble fraction of particles at two sites in
the urban area of Milan, Northern Italy, Atmospheric Environment, 128,
104--113,
doi:\href{https://doi.org/10.1016/j.atmosenv.2015.12.040}{10.1016/j.atmosenv.2015.12.040},
2016.

Perrone, M. R., Bertoli, I., Romano, S., Russo, M., Rispoli, G. and
Pietrogrande, M. C.: PM2.5 and PM10 oxidative potential at a Central
Mediterranean Site: Contrasts between dithiothreitol- and ascorbic
acid-measured values in relation with particle size and chemical
composition, Atmospheric Environment, 210, 143--155,
doi:\href{https://doi.org/10.1016/j.atmosenv.2019.04.047}{10.1016/j.atmosenv.2019.04.047},
2019.

Petit, J.-E., Favez, O., Sciare, J., Canonaco, F., Croteau, P., Močnik,
G., Jayne, J., Worsnop, D. and Leoz-Garziandia, E.: Submicron aerosol
source apportionment of wintertime pollution in Paris, France by double
positive matrix factorization (PMF\textsuperscript{2}) using an aerosol
chemical speciation monitor (ACSM) and a multi-wavelength Aethalometer,
Atmospheric Chemistry and Physics, 14(24), 13773--13787,
doi:\href{https://doi.org/https://doi.org/10.5194/acp-14-13773-2014}{https://doi.org/10.5194/acp-14-13773-2014},
2014.

Petit, J.-E., Favez, O., Sciare, J., Crenn, V., Sarda-Estève, R.,
Bonnaire, N., Močnik, G., Dupont, J.-C., Haeffelin, M. and
Leoz-Garziandia, E.: Two years of near real-time chemical composition of
submicron aerosols in the region of Paris using an Aerosol Chemical
Speciation Monitor (ACSM) and a multi-wavelength Aethalometer,
Atmospheric Chemistry and Physics, 15(6), 2985--3005,
doi:\href{https://doi.org/10.5194/acp-15-2985-2015}{10.5194/acp-15-2985-2015},
2015.

Petit, J.-E., Pallarès, C., Favez, O., Alleman, L. Y., Bonnaire, N. and
Rivière, E.: Sources and Geographical Origins of PM10 in Metz (France)
Using Oxalate as a Marker of Secondary Organic Aerosols by Positive
Matrix Factorization Analysis, Atmosphere, 10(7), 370,
doi:\href{https://doi.org/10.3390/atmos10070370}{10.3390/atmos10070370},
2019.

Pietrogrande, M. C., Dalpiaz, C., Dell'Anna, R., Lazzeri, P., Manarini,
F., Visentin, M. and Tonidandel, G.: Chemical composition and oxidative
potential of atmospheric coarse particles at an industrial and urban
background site in the alpine region of northern Italy, Atmospheric
Environment, 191, 340--350,
doi:\href{https://doi.org/10.1016/j.atmosenv.2018.08.022}{10.1016/j.atmosenv.2018.08.022},
2018a.

Pietrogrande, M. C., Perrone, M. R., Manarini, F., Romano, S., Udisti,
R. and Becagli, S.: PM10 oxidative potential at a Central Mediterranean
Site: Association with chemical composition and meteorological
parameters, Atmospheric Environment, 188, 97--111,
doi:\href{https://doi.org/10.1016/j.atmosenv.2018.06.013}{10.1016/j.atmosenv.2018.06.013},
2018b.

Piot, C., Jaffrezo, J.-L., Cozic, J., Pissot, N., Haddad, I. E.,
Marchand, N. and Besombes, J.-L.: Quantification of levoglucosan and its
isomers by High Performance Liquid Chromatography \&ndash; Electrospray
Ionization tandem Mass Spectrometry and its applications to atmospheric
and soil samples, Atmospheric Measurement Techniques, 5(1), 141--148,
doi:\href{https://doi.org/https://doi.org/10.5194/amt-5-141-2012}{https://doi.org/10.5194/amt-5-141-2012},
2012.

Salameh, D., Detournay, A., Pey, J., Pérez, N., Liguori, F., Saraga, D.,
Bove, M. C., Brotto, P., Cassola, F., Massabò, D., Latella, A., Pillon,
S., Formenton, G., Patti, S., Armengaud, A., Piga, D., Jaffrezo, J. L.,
Bartzis, J., Tolis, E., Prati, P., Querol, X., Wortham, H. and Marchand,
N.: PM2.5 chemical composition in five European Mediterranean cities: A
1-year study, Atmospheric Research, 155, 102--117,
doi:\href{https://doi.org/10.1016/j.atmosres.2014.12.001}{10.1016/j.atmosres.2014.12.001},
2015.

Salameh, D., Pey, J., Bozzetti, C., El Haddad, I., Detournay, A.,
Sylvestre, A., Canonaco, F., Armengaud, A., Piga, D., Robin, D., Prevot,
A. S. H., Jaffrezo, J. L., Wortham, H. and Marchand, N.: Sources of
PM\textsubscript{2.5} at an urban-industrial Mediterranean city,
Marseille (France): Application of the ME-2 solver to inorganic and
organic markers, Atmospheric Research, 214, 263--274,
doi:\href{https://doi.org/10.1016/j.atmosres.2018.08.005}{10.1016/j.atmosres.2018.08.005},
2018.

Saleh, Y., Antherieu, S., Dusautoir, R., Y. Alleman, L., Sotty, J., De
Sousa, C., Platel, A., Perdrix, E., Riffault, V., Fronval, I., Nesslany,
F., Canivet, L., Garçon, G. and Lo-Guidice, J.-M.: Exposure to
Atmospheric Ultrafine Particles Induces Severe Lung Inflammatory
Response and Tissue Remodeling in Mice, International Journal of
Environmental Research and Public Health, 16(7), 1210,
doi:\href{https://doi.org/10.3390/ijerph16071210}{10.3390/ijerph16071210},
2019.

Samaké, A., Uzu, G., Martins, J. M. F., Calas, A., Vince, E., Parat, S.
and Jaffrezo, J.-L.: The unexpected role of bioaerosols in the Oxidative
Potential of PM, Scientific Reports, 7(1), 10978,
doi:\href{https://doi.org/10.1038/s41598-017-11178-0}{10.1038/s41598-017-11178-0},
2017.

Samaké, A., Jaffrezo, J.-L., Favez, O., Weber, S., Jacob, V., Canete,
T., Albinet, A., Charron, A., Riffault, V., Perdrix, E., Waked, A.,
Golly, B., Salameh, D., Chevrier, F., Oliveira, D. M., Besombes, J.-L.,
Martins, J. M. F., Bonnaire, N., Conil, S., Guillaud, G., Mesbah, B.,
Rocq, B., Robic, P.-Y., Hulin, A., Meur, S. L., Descheemaecker, M.,
Chretien, E., Marchand, N. and Uzu, G.: Arabitol, mannitol, and glucose
as tracers of primary biogenic organic aerosol: The influence of
environmental factors on ambient air concentrations and spatial
distribution over France, Atmospheric Chemistry and Physics, 19(16),
11013--11030,
doi:\href{https://doi.org/https://doi.org/10.5194/acp-19-11013-2019}{https://doi.org/10.5194/acp-19-11013-2019},
2019a.

Samaké, A., Jaffrezo, J.-L., Favez, O., Weber, S., Jacob, V., Albinet,
A., Riffault, V., Perdrix, E., Waked, A., Golly, B., Salameh, D.,
Chevrier, F., Oliveira, D. M., Bonnaire, N., Besombes, J.-L., Martins,
J. M. F., Conil, S., Guillaud, G., Mesbah, B., Rocq, B., Robic, P.-Y.,
Hulin, A., Meur, S. L., Descheemaecker, M., Chretien, E., Marchand, N.
and Uzu, G.: Polyols and glucose particulate species as tracers of
primary biogenic organic aerosols at 28 French sites, Atmospheric
Chemistry and Physics, 19(5), 3357--3374,
doi:\href{https://doi.org/https://doi.org/10.5194/acp-19-3357-2019}{https://doi.org/10.5194/acp-19-3357-2019},
2019b.

Sauvain, J.-J., Deslarzes, S. and Riediker, M.: Nanoparticle reactivity
toward dithiothreitol, Nanotoxicology, 2(3), 121--129,
doi:\href{https://doi.org/10.1080/17435390802245716}{10.1080/17435390802245716},
2008.

Seabold, S. and Perktold, J.: Statsmodels: Econometric and statistical
modeling with python, in Proceedings of the 9th Python in Science
Conference, vol. 57, p. 61., 2010.

Simon, H., Beck, L., Bhave, P. V., Divita, F., Hsu, Y., Luecken, D.,
Mobley, J. D., Pouliot, G. A., Reff, A., Sarwar, G. and Strum, M.: The
development and uses of EPA's SPECIATE database, Atmospheric Pollution
Research, 1(4), 196--206,
doi:\href{https://doi.org/10.5094/APR.2010.026}{10.5094/APR.2010.026},
2010.

Srivastava, D., Tomaz, S., Favez, O., Lanzafame, G. M., Golly, B.,
Besombes, J.-L., Alleman, L. Y., Jaffrezo, J.-L., Jacob, V., Perraudin,
E., Villenave, E. and Albinet, A.: Speciation of organic fraction does
matter for source apportionment. Part 1: A one-year campaign in Grenoble
(France), Science of The Total Environment, 624, 1598--1611,
doi:\href{https://doi.org/10.1016/j.scitotenv.2017.12.135}{10.1016/j.scitotenv.2017.12.135},
2018.

Steenhof, M., Gosens, I., Strak, M., Godri, K. J., Hoek, G., Cassee, F.
R., Mudway, I. S., Kelly, F. J., Harrison, R. M., Lebret, E.,
Brunekreef, B., Janssen, N. A. and Pieters, R. H.: In vitro toxicity of
particulate matter (PM) collected at different sites in the Netherlands
is associated with PM composition, size fraction and oxidative potential
- the RAPTES project, Particle and Fibre Toxicology, 8(1), 26,
doi:\href{https://doi.org/10.1186/1743-8977-8-26}{10.1186/1743-8977-8-26},
2011.

Strak, M., Janssen, N., Beelen, R., Schmitz, O., Karssenberg, D.,
Houthuijs, D., van den Brink, C., Dijst, M., Brunekreef, B. and Hoek,
G.: Associations between lifestyle and air pollution exposure: Potential
for confounding in large administrative data cohorts, Environmental
Research, 156, 364--373,
doi:\href{https://doi.org/10.1016/j.envres.2017.03.050}{10.1016/j.envres.2017.03.050},
2017a.

Strak, M., Janssen, N., Beelen, R., Schmitz, O., Vaartjes, I.,
Karssenberg, D., van den Brink, C., Bots, M. L., Dijst, M., Brunekreef,
B. and Hoek, G.: Long-term exposure to particulate matter, NO2 and the
oxidative potential of particulates and diabetes prevalence in a large
national health survey, Environment International, 108, 228--236,
doi:\href{https://doi.org/10.1016/j.envint.2017.08.017}{10.1016/j.envint.2017.08.017},
2017b.

Tuet, W. Y., Chen, Y., Fok, S., Gao, D., Weber, R. J., Champion, J. A.
and Ng, N. L.: Chemical and cellular oxidant production induced by
naphthalene secondary organic aerosol (SOA): Effect of redox-active
metals and photochemical aging, Scientific Reports, 7(1),
doi:\href{https://doi.org/10.1038/s41598-017-15071-8}{10.1038/s41598-017-15071-8},
2017.

Tuet, W. Y., Liu, F., de Oliveira Alves, N., Fok, S., Artaxo, P.,
Vasconcellos, P., Champion, J. A. and Ng, N. L.: Chemical Oxidative
Potential and Cellular Oxidative Stress from Open Biomass Burning
Aerosol, Environmental Science \& Technology Letters, 6(3), 126--132,
doi:\href{https://doi.org/10.1021/acs.estlett.9b00060}{10.1021/acs.estlett.9b00060},
2019.

US EPA: Positive Matrix Factorization Model for environmental data
analyses, www.epa.gov, 2017.

Verma, V., Ning, Z., Cho, A. K., Schauer, J. J., Shafer, M. M. and
Sioutas, C.: Redox activity of urban quasi-ultrafine particles from
primary and secondary sources, Atmospheric Environment, 43(40),
6360--6368,
doi:\href{https://doi.org/10.1016/j.atmosenv.2009.09.019}{10.1016/j.atmosenv.2009.09.019},
2009.

Verma, V., Fang, T., Guo, H., King, L., Bates, J. T., Peltier, R. E.,
Edgerton, E., Russell, A. G. and Weber, R. J.: Reactive oxygen species
associated with water-soluble PM2.5 in the southeastern United States:
Spatiotemporal trends and source apportionment., Atmospheric Chemistry
and Physics, 14(23), 12915--12930,
doi:\href{https://doi.org/10.5194/acp-14-12915-2014}{10.5194/acp-14-12915-2014},
2014.

Verma, V., Wang, Y., El-Afifi, R., Fang, T., Rowland, J., Russell, A. G.
and Weber, R. J.: Fractionating ambient humic-like substances (HULIS)
for their reactive oxygen species activity Assessing the importance of
quinones and atmospheric aging, Atmospheric Environment, 120, 351--359,
doi:\href{https://doi.org/10.1016/j.atmosenv.2015.09.010}{10.1016/j.atmosenv.2015.09.010},
2015a.

Verma, V., Fang, T., Xu, L., Peltier, R. E., Russell, A. G., Ng, N. L.
and Weber, R. J.: Organic Aerosols Associated with the Generation of
Reactive Oxygen Species (ROS) by Water-Soluble PM \textsubscript{2.5},
Environmental Science \& Technology, 49(7), 4646--4656,
doi:\href{https://doi.org/10.1021/es505577w}{10.1021/es505577w}, 2015b.

Vlachou, A., Daellenbach, K. R., Bozzetti, C., Chazeau, B., Salazar, G.
A., Szidat, S., Jaffrezo, J.-L., Hueglin, C., Baltensperger, U., Haddad,
I. E. and Prévôt, A. S. H.: Advanced source apportionment of
carbonaceous aerosols by coupling offline AMS and radiocarbon
size-segregated measurements over a nearly 2-year period, Atmospheric
Chemistry and Physics, 18(9), 6187--6206,
doi:\href{https://doi.org/https://doi.org/10.5194/acp-18-6187-2018}{https://doi.org/10.5194/acp-18-6187-2018},
2018.

Vlachou, A., Tobler, A., Lamkaddam, H., Canonaco, F., Daellenbach, K.
R., Jaffrezo, J.-L., Minguillón, M. C., Maasikmets, M., Teinemaa, E.,
Baltensperger, U., Haddad, I. E. and Prévôt, A. S. H.: Development of a
versatile source apportionment analysis based on positive matrix
factorization: A case study of the seasonal variation of organic aerosol
sources in Estonia, Atmospheric Chemistry and Physics, 19(11),
7279--7295,
doi:\href{https://doi.org/https://doi.org/10.5194/acp-19-7279-2019}{https://doi.org/10.5194/acp-19-7279-2019},
2019.

Wagstrom, K. M., Pandis, S. N., Yarwood, G., Wilson, G. M. and Morris,
R. E.: Development and application of a computationally efficient
particulate matter apportionment algorithm in a three-dimensional
chemical transport model, Atmospheric Environment, 42(22), 5650--5659,
doi:\href{https://doi.org/10.1016/j.atmosenv.2008.03.012}{10.1016/j.atmosenv.2008.03.012},
2008.

Waked, A., Favez, O., Alleman, L. Y., Piot, C., Petit, J.-E., Delaunay,
T., Verlinden, E., Golly, B., Besombes, J.-L., Jaffrezo, J.-L. and
Leoz-Garziandia, E.: Source apportionment of PM\textsubscript{10} in a
north-western Europe regional urban background site (Lens, France) using
positive matrix factorization and including primary biogenic emissions,
Atmospheric Chemistry and Physics, 14(7), 3325--3346,
doi:\href{https://doi.org/10.5194/acp-14-3325-2014}{10.5194/acp-14-3325-2014},
2014.

Wang, S., Ye, J., Soong, R., Wu, B., Yu, L., Simpson, A. and Chan, A. W.
H.: Relationship between Chemical Composition and Oxidative Potential of
Secondary Organic Aerosol from Polycyclic Aromatic Hydrocarbons,
Atmospheric Chemistry and Physics Discussions, 1--53,
doi:\href{https://doi.org/10.5194/acp-2017-1012}{10.5194/acp-2017-1012},
2017.

Wang, Z. S., Chien, C.-J. and Tonnesen, G. S.: Development of a tagged
species source apportionment algorithm to characterize three-dimensional
transport and transformation of precursors and secondary pollutants,
Journal of Geophysical Research: Atmospheres, 114(D21),
doi:\href{https://doi.org/10.1029/2008JD010846}{10.1029/2008JD010846},
2009.

Weber, S., Uzu, G., Calas, A., Chevrier, F., Besombes, J.-L., Charron,
A., Salameh, D., Ježek, I., Močnik, G. and Jaffrezo, J.-L.: An
apportionment method for the oxidative potential of atmospheric
particulate matter sources: Application to a one-year study in Chamonix,
France, Atmospheric Chemistry and Physics, 18(13), 9617--9629,
doi:\href{https://doi.org/10.5194/acp-18-9617-2018}{10.5194/acp-18-9617-2018},
2018.

Weber, S., Salameh, D., Albinet, A., Alleman, L. Y., Waked, A.,
Besombes, J.-L., Jacob, V., Guillaud, G., Mesbah, B., Rocq, B., Hulin,
A., Dominik-Sègue, M., Chrétien, E., Jaffrezo, J.-L. and Favez, O.:
Comparison of PM\textsubscript{10} Sources Profiles at 15 French Sites
Using a Harmonized Constrained Positive Matrix Factorization Approach,
Atmosphere, 10(6), 310,
doi:\href{https://doi.org/10.3390/atmos10060310}{10.3390/atmos10060310},
2019.

Weichenthal, S., Crouse, D. L., Pinault, L., Godri-Pollitt, K., Lavigne,
E., Evans, G., van Donkelaar, A., Martin, R. V. and Burnett, R. T.:
Oxidative burden of fine particulate air pollution and risk of
cause-specific mortality in the Canadian Census Health and Environment
Cohort (CanCHEC), Environmental Research, 146, 92--99,
doi:\href{https://doi.org/10.1016/j.envres.2015.12.013}{10.1016/j.envres.2015.12.013},
2016a.

Weichenthal, S. A., Lavigne, E., Evans, G. J., Godri Pollitt, K. J. and
Burnett, R. T.: Fine Particulate Matter and Emergency Room Visits for
Respiratory Illness. Effect Modification by Oxidative Potential,
American Journal of Respiratory and Critical Care Medicine, 194(5),
577--586,
doi:\href{https://doi.org/10.1164/rccm.201512-2434OC}{10.1164/rccm.201512-2434OC},
2016b.

World Health Organization: Ambient air pollution: A global assessment of
exposure and burden of disease, World Health Organization, Geneva,
Switzerland., 2016.

Yang, A., Janssen, N. A. H., Brunekreef, B., Cassee, F. R., Hoek, G. and
Gehring, U.: Children's respiratory health and oxidative potential of
PM\textsubscript{2.5}: The PIAMA birth cohort study, Occupational and
Environmental Medicine, 73(3), 154--160,
doi:\href{https://doi.org/10.1136/oemed-2015-103175}{10.1136/oemed-2015-103175},
2016.

Zhang, X., Staimer, N., Gillen, D. L., Tjoa, T., Schauer, J. J., Shafer,
M. M., Hasheminassab, S., Pakbin, P., Vaziri, N. D., Sioutas, C. and
Delfino, R. J.: Associations of oxidative stress and inflammatory
biomarkers with chemically-characterized air pollutant exposures in an
elderly cohort, Environmental Research, 150, 306--319,
doi:\href{https://doi.org/10.1016/j.envres.2016.06.019}{10.1016/j.envres.2016.06.019},
2016.

Zhou, J., Zotter, P., Bruns, E. A., Stefenelli, G., Bhattu, D., Brown,
S., Bertrand, A., Marchand, N., Lamkaddam, H., Slowik, J. G., Prévôt, A.
S. H., Baltensperger, U., Nussbaumer, T., El-Haddad, I. and Dommen, J.:
Particle-bound reactive oxygen species (PB-ROS) emissions and formation
pathways in residential wood smoke under different combustion and aging
conditions, Atmospheric Chemistry and Physics, 18(10), 6985--7000,
doi:\href{https://doi.org/https://doi.org/10.5194/acp-18-6985-2018}{https://doi.org/10.5194/acp-18-6985-2018},
2018.

Zhou, J., Elser, M., Huang, R.-J., Krapf, M., Fröhlich, R., Bhattu, D.,
Stefenelli, G., Zotter, P., Bruns, E. A., Pieber, S. M., Ni, H., Wang,
Q., Wang, Y., Zhou, Y., Chen, C., Xiao, M., Slowik, J. G., Brown, S.,
Cassagnes, L.-E., Daellenbach, K. R., Nussbaumer, T., Geiser, M.,
Prévôt, A. S. H., El-Haddad, I., Cao, J., Baltensperger, U. and Dommen,
J.: Predominance of secondary organic aerosol to particle-bound reactive
oxygen species activity in fine ambient aerosol, Atmospheric Chemistry
and Physics, 19(23), 14703--14720,
doi:\href{https://doi.org/https://doi.org/10.5194/acp-19-14703-2019}{https://doi.org/10.5194/acp-19-14703-2019},
2019.

\end{document}
